% Document de classe yathesis
\documentclass{yathesis}
%
% Chargement manuel de packages (pas déjà chargés par la classe yathesis)
\usepackage[T1]{fontenc}
\usepackage[utf8]{inputenc}
\usepackage{kpfonts}
\usepackage{booktabs}
\usepackage{siunitx}
\usepackage{pgfplots}
\usepackage{floatrow}
\usepackage{caption}
\usepackage{microtype}
\usepackage{varioref}
%\usepackage[xindy,quiet]{imakeidx}
%\usepackage[autostyle]{csquotes}
%\usepackage[backend=biber,safeinputenc]{biblatex}
\usepackage{hyperref}
%\usepackage[xindy,acronyms,symbols]{glossaries}
%
% (Facultatif) Génération de l'index (obligatoire si un package d'index, par
% exemple « imakeidx », est chargé)
% \makeindex
%
% (Facultatif) Spécification de la ou des ressources bibliographiques
% (obligatoire si le package « biblatex » est chargé)
% \addbibresource{auxiliaires/bibliographie.bib}
% \addbibresource{auxiliaires/}
%
% (Facultatif) Génération du glossaire (obligatoire si le package « glossaries »
% est chargé)
% \makeglossaries
%
% (Facultatif) Configuration des styles du glossaire et de la liste d'acronymes
% (à n'utiliser que si le package « glossaries » est chargé)
% \setglossarystyle{indexhypergroup}
% \setacronymstyle{long-sc-short}
%
% (Facultatif) Spécification de la ou des ressources terminologiques
% \loadglsentries{auxiliaires/}
% \loadglsentries{auxiliaires/}
% \loadglsentries{auxiliaires/}
%
% Les réglages figurant habituellement dans le préambule, notamment concernant
% la bibliographie et l'éventuel index, peuvent être saisis dans le fichier
% « thesis.cfg » (situé dans le sous-dossier « configuration ») qui est
% automatiquement importé par la classe yathesis.
%
% Importation manuelle du fichier de macros personnelles
% Macro pour mettre en forme les noms de fichiers
\newcommand{\fichier}[1]{\texttt{#1}}
% Macro pour mettre en forme les noms de packages LaTeX
\newcommand{\package}[1]{\textsf{#1}}
% Macro pour mettre en forme des locutions étrangères
\newcommand{\locution}[1]{\emph{#1}}

%
%%%%%%%%%%%%%%%%%%%%%%%%%%%%%%%%%%%%%%%%%%%%%%%%%%%%%%%%%%%%%%%%%%%%%%%%%%%%%%%
%%%%%%%%%%%%%%%%%%%%%%%%%%%%%%%%%%%%%%%%%%%%%%%%%%%%%%%%%%%%%%%%%%%%%%%%%%%%%%%
% Début du document
%%%%%%%%%%%%%%%%%%%%%%%%%%%%%%%%%%%%%%%%%%%%%%%%%%%%%%%%%%%%%%%%%%%%%%%%%%%%%%%
%%%%%%%%%%%%%%%%%%%%%%%%%%%%%%%%%%%%%%%%%%%%%%%%%%%%%%%%%%%%%%%%%%%%%%%%%%%%%%%
\begin{document}
%
%%%%%%%%%%%%%%%%%%%%%%%%%%%%%%%%%%%%%%%%%%%%%%%%%%%%%%%%%%%%%%%%%%%%%%%%%%%%%%%
% Caractéristiques du document
%%%%%%%%%%%%%%%%%%%%%%%%%%%%%%%%%%%%%%%%%%%%%%%%%%%%%%%%%%%%%%%%%%%%%%%%%%%%%%%
%
% Préparation des pages de couverture et de titre
%%%%%%%%%%%%%%%%%%%%%%%%%%%%%%%%%%%%%%%%%%%%%%%%%%%%%%%%%%%%%%%%%%%%%%%%%%%%%%%
% Les caractéristiques de la thèse sont saisies dans le fichier
% « characteristics.tex » (situé dans le dossier « configuration »).
%
% Production des pages de couverture et de titre
%%%%%%%%%%%%%%%%%%%%%%%%%%%%%%%%%%%%%%%%%%%%%%%%%%%%%%%%%%%%%%%%%%%%%%%%%%%%%%%
\maketitle
%
%%%%%%%%%%%%%%%%%%%%%%%%%%%%%%%%%%%%%%%%%%%%%%%%%%%%%%%%%%%%%%%%%%%%%%%%%%%%%%%
% Début de la partie liminaire de la thèse
%%%%%%%%%%%%%%%%%%%%%%%%%%%%%%%%%%%%%%%%%%%%%%%%%%%%%%%%%%%%%%%%%%%%%%%%%%%%%%%
%
% (Facultatif) Production de la page de clause de non-responsabilité
\makedisclaimer
%
% (Facultatif) Production de la page de mots clés
\makekeywords
%
% (Facultatif) Production de la page affichant les logo, nom et coordonnées du
% ou des laboratoires (ou unités de recherche) où la thèse a été préparée
\makelaboratory
%
% (Facultatif) Dédicace(s)
% Dédicace(s)
\dedication{À mon directeur bien-aimé !}
\dedication{À mon co-directeur bien-co-aimé aussi !}
\dedication{Je dédie également ce travail\\à tous ceux qui le méritent}
% Production de la page de dédicace(s)
\makededications

%
% (Facultatif) Épigraphe(s)
% Épigraphes(s)
\frontepigraph{Science sans conscience n'est que ruine de l'âme.}{François Rabelais}
\frontepigraph[english]{I can resist everything, except temptation!}{Oscar Wilde}
\frontepigraph{Il est plus facile de désintégrer un atome qu'un préjugé.}{Albert Einstein}
% Production de la page de d'épigraphe(s)
\makefrontepigraphs

%
% Résumés succincts
% Résumés (de 1700 caractères maximum, espaces compris) dans la
% langue principale (1re occurrence de l'environnement « abstract »)
% et, facultativement, dans la langue secondaire (2e occurrence de
% l'environnement « abstract »)
\begin{abstract}
  \lipsum[1-2]
\end{abstract}
\begin{abstract}
  \lipsum[3-4]
\end{abstract}
%
% Production de la page de résumés
\makeabstract

%
% (Facultatif) Chapitre de remerciements
\chapter{Remerciements}
\section{Une section de remerciements}
\lipsum[1]
\section{Une autre section de remerciements}
\lipsum[2-9]

%
% (Facultatif) Chapitre d'avertissement
% \chapter{Avertissement}
Thèse hilarante, comme le gaz du même nom !

%
% (Facultatif) Liste des acronymes
% \printacronyms
%
% (Facultatif) Liste des symboles
% \printsymbols
%
% (Facultatif) Chapitre d'avant-propos
% \chapter{Avant-propos}
\section{Une section d'avant-propos}
\lipsum[30-45]
\section{Une autre section d'avant-propos}
\lipsum[30-35]

%
% Sommaire
\tableofcontents[depth=chapter,name=Sommaire]
%
% (Facultatif) Liste des tableaux
\listoftables
%
% (Facultatif) Table des figures
\listoffigures
%
% (Facultatif) Table des listings (nécessite que le package « listings » soit
% chargé)
% \lstlistoflistings
%
%%%%%%%%%%%%%%%%%%%%%%%%%%%%%%%%%%%%%%%%%%%%%%%%%%%%%%%%%%%%%%%%%%%%%%%%%%%%%%%
% Début de la partie principale (du « corps ») de la thèse
%%%%%%%%%%%%%%%%%%%%%%%%%%%%%%%%%%%%%%%%%%%%%%%%%%%%%%%%%%%%%%%%%%%%%%%%%%%%%%%
\mainmatter
%
% Chapitre d'introduction (générale)
%%%%%%%%%%%%%%%%%%%%%%%%%%%%%%%%%%%%%%%%%%%%%%%%%%%%%%%%%%%%%%%%%%%%%%%%%%%%%%%
\chapter*{Introduction générale}
\lipsum[26]
\section{Une section d'introduction}
\lipsum[28]
\subsection{Une sous-section d'introduction}
\lipsum[29]
\subsubsection{Une sous-sous-section d'introduction}
\lipsum[30]
\paragraph{Un paragraphe d'introduction}
\lipsum[31]
\subparagraph{Un sous-paragraphe d'introduction}
\lipsum[32]
\subparagraph{Un autre sous-paragraphe d'introduction}
\lipsum[33]
\paragraph{Un autre paragraphe d'introduction}
\lipsum[34]
\subsubsection{Une autre sous-sous-section d'introduction}
\lipsum[35]
\subsection{Une autre sous-section d'introduction}
\lipsum[36]
\section{Une autre section d'introduction}
\lipsum[37]

%
% Chapitres ordinaires (avec parties éventuelles)
%%%%%%%%%%%%%%%%%%%%%%%%%%%%%%%%%%%%%%%%%%%%%%%%%%%%%%%%%%%%%%%%%%%%%%%%%%%%%%%
%
% Première partie éventuelle
% \part{...}
%
% Premier chapitre
% \include{corps/}
%
% Deuxième chapitre
% \include{corps/}
%
% Troisième chapitre
% \include{corps/}
%
%
% Deuxième partie éventuelle
% \part{...}
%
% Quatrième chapitre
% \include{corps/}
%
% Cinquième chapitre
% \include{corps/}
%
% Sixième chapitre
% \include{corps/}
%
% Chapitre  de conclusion (générale)
%%%%%%%%%%%%%%%%%%%%%%%%%%%%%%%%%%%%%%%%%%%%%%%%%%%%%%%%%%%%%%%%%%%%%%%%%%%%%%%
\chapter*{Conclusion générale}
\lipsum[26-27]
\section{Une section de conclusion}
\lipsum[28-29]
\subsection{Une sous-section de conclusion}
\lipsum[29-31]
\subsubsection{Une sous-sous-section de conclusion}
\lipsum[31-35]
\paragraph{Un paragraphe de conclusion}
\lipsum[36-38]
\subparagraph{Un sous-paragraphe de conclusion}
\lipsum[39-41]
\subparagraph{Un autre sous-paragraphe de conclusion}
\lipsum[39-41]
\paragraph{Un autre paragraphe de conclusion}
\lipsum[36-38]
\subsubsection{Une autre sous-sous-section de conclusion}
\lipsum[31-37]
\subsection{Une autre sous-section de conclusion}
\lipsum[29-31]
\section{Une autre section de conclusion}
\lipsum[28-43]

%
% Liste des références bibliographiques
%\printbibliography
%
%%%%%%%%%%%%%%%%%%%%%%%%%%%%%%%%%%%%%%%%%%%%%%%%%%%%%%%%%%%%%%%%%%%%%%%%%%%%%%%
% Début de la partie annexe éventuelle
%%%%%%%%%%%%%%%%%%%%%%%%%%%%%%%%%%%%%%%%%%%%%%%%%%%%%%%%%%%%%%%%%%%%%%%%%%%%%%%
% \appendix
%
% Premier chapitre annexe (éventuel)
% % \chapter{...}
% ...

%
% Deuxième chapitre annexe (éventuel)
% % \chapter{...}
% ...

%
%%%%%%%%%%%%%%%%%%%%%%%%%%%%%%%%%%%%%%%%%%%%%%%%%%%%%%%%%%%%%%%%%%%%%%%%%%%%%%%
% Début de la partie finale
%%%%%%%%%%%%%%%%%%%%%%%%%%%%%%%%%%%%%%%%%%%%%%%%%%%%%%%%%%%%%%%%%%%%%%%%%%%%%%%
\backmatter
%
% (Facultatif) Glossaire (si souhaité distinct de la liste des acronymes) :
% \printglossary
%
% (Facultatif) Index :
% \printindex
%
% Table des matières
\tableofcontents
%
% (Facultatif) Production de la 4e de couverture :
\makebackcover
%
\end{document}
