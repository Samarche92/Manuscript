\chapter{Steepest Descent Method}
\label{PhaseStationnaire}
The steepest descent method is an integral approximation technique where the integration contour is deformed into a contour $\gamma$ called the steepest descent contour, which passes near a saddle-point of the integrated function. Useful results are stated here without demonstration.

\paragraph{}
The integral to be estimated is of the form :
\begin{equation}
I(\lambda)=\int_{\gamma} f(z)e^{\lambda S(z)}\,dz
\end{equation}
where $\gamma$ is the steepest descent contour, $S$ and $f$ are analytical on all $\mathbb{C}^n$, except for eventually at a finite number of points, and $\lambda>0$. The steepest descent contour $\gamma$ must verify :
\begin{itemize}
\item $C$ and $\gamma$ must have the same endpoints,
\item $\gamma$ passes through at least one saddle point of $g$,
\item Im$(g)$ is constant on $\gamma$.
\end{itemize}

\paragraph{}
Le us denote $\mathbf{S_{xx}}(z)$ the hessian matrix of $S$, defined by :
\begin{equation}
\mathbf{S_{xx}}(z)=\left( \frac{\partial^2 S}{\partial x_i \partial x_j}(z) \right)_{1\leq i,j \leq n},
\end{equation}
then $z_ 0$ is a non-degenerate saddle point of $S$ if and only if :
\begin{eqnarray}
\left\{
\begin{array}{l}
\nabla S(z_0)=0 \\
\mbox{det} \; \mathbf{S_{xx}}(z_0) \neq 0
\end{array}
\right.
\end{eqnarray}

The following proposition is then true :

\begin{prop}
Assume
\begin{description}
  \item[(i)] $f$ and $S$ are holomorphic on an open, bounded and simply connected subset $W_x \subset \mathbb{C}^n$ such that $I_x=W_x \cap \mathbb{R}^n$ is connected,
  \item[(ii)]$ Re\left( S(z) \right)$ has a single maximum reached at exactly one point $z_0 \in I_x$,
  \item[(iii)] $z_0$ is a non-degenerate saddle point of $S$.
\end{description}
The following asymptotic evaluation then holds :
\begin{equation}
I(\lambda) \underset{\lambda \to +\infty}{=} \left( \frac{2\pi}{\lambda} \right)^{n/2} e^{\lambda S(z_0)} \lbrack f(z_0)+ \mathcal{O}(\lambda^{-1}) \rbrack \prod_{j=1}^n (-\mu_j)^{-1/2},
\label{steepformula}
\end{equation}
where $(\mu_j)_{1\leq j \leq n}$ are the eigen values of $\mathbf{S_{xx}}(z_0)$ and their square roots are defined by
$$|\mbox{arg} \sqrt{-\mu_j}| <\frac{\pi}{4} $$
\end{prop}

Note that any if any singularities are crossed during deformation of contour $C$ to contour $\gamma$, their contribution to the integral must be correctly taken into account.