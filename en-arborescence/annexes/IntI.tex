\chapter{Computation details for the coefficients of matrix $\mathbb{D}(a,b)$}
\label{matD}
In each chapter of this manuscript, a matrix $\mathbb{D}$ appears, the coefficients of which must be determined analytically. These coefficients are linear combinations of integrals noted $I_1^*$ to $I_5^*$. In this appendix, these integrals are defined and the details of their computation is given. Finally, expressions of the coefficients of matrix $\mathbb{D}$ in the acoustic and 2D and 3D elastic cases are given. In all the following, $\nuti \in \{ 1,\nu_T,\nuti_L,\nuti_T \}$.

\section{Integral $I_1^*$}
\label{calcI1}
Integral $I_1^*$ is defined by :
\begin{equation}
\label{defI1}
I_1^*=\int_{-\infty}^{+\infty} \frac{y}{(y+ib)(y-ia)\sqrt{\nuti^2+y^2}}\,dy
\end{equation}
When $a+b \neq 0$, we have the simple elements decomposition :
\begin{equation}
\frac{y}{(y+ib)(y-ia)}=\frac{1}{a+b} \left( \frac{b}{y+ib}+\frac{a}{y-ia} \right), 
\label{decomp2}
\end{equation}
which leads to
\begin{equation}
I_1^*= \frac{1}{a+b} \left( \int_{-\infty}^{+\infty} \frac{b}{(y+ib)\sqrt{\nuti^2+y^2}} \,dy +\int_{-\infty}^{+\infty} \frac{a}{(y-ia)\sqrt{\nuti^2+y^2}} \,dy \right)
\end{equation}
In each of these integrals, the following variable change is applied :
\begin{equation}
\begin{split}
 y&=2\nuti \frac{t}{1-t^2} \\
\sqrt{\nuti^2+y^2}&=\nuti\left( \frac{1+t^2}{1-t^2} \right)  \\
 dy&= 2\nuti \frac{1+t^2}{(1-t^2)^2}dt 
\end{split}
\label{changevar}
\end{equation}
Yielding
\begin{equation}
\int_{-\infty}^{+\infty} \frac{dy}{(y-ia)\sqrt{\nuti^2+y^2}}=\int_{-1}^1\frac{2\,dt}{2\nuti t-ia(1-t^2)}
\end{equation}
Finally, formula \eqref{intQ} is applied :
\begin{equation}
I_1^*=\frac{2ia}{(a+b)\nuti}\mbox{rog}(a/\nuti)-\frac{2ib}{(a+b)\nuti}\mbox{rog}(b/\nuti)
\label{valI1}
\end{equation}

\section{Integral $I_2^*$}
\label{calcI2}
Integral $I_2^*$ is defined by :
\begin{equation}
I_2^*=  \int_{-\infty}^{+\infty} \frac{y\sqrt{\nuti^2+y^2}}{(y+ib)(y-ia)}\, dy
\label{defI2}
\end{equation}
Note that
\begin{equation*}
\begin{split}
I_2^*&=\nuti^2 \int_{-\infty}^{+\infty}  \frac{y}{(y+ib)(y-ia)\sqrt{\nuti^2+y^2}}\,dy+ \int_{-\infty}^{+\infty} \frac{y^3}{(y+ib)(y-ia)\sqrt{\nuti^2+y^2}} \, dy\\
I_2^*&=\nuti^2 I_1^*+I_3^*,
\end{split}
\end{equation*}
where integrals $I_1^*$ and $I_3^*$ are defined by equations \eqref{defI1} and \eqref{defI3} and their expressions are given by \eqref{valI1} and \eqref{valI3} respectively.

\section{Integral $I_3^*$}
\label{calcI3}
Integral $I_3^*$ is defined by :
\begin{equation}
I_3^*=\int_{-\infty}^{+\infty} \frac{y^3}{(y+ib)(y-ia)\sqrt{\nuti^2+y^2}} \,dy
\label{defI3}
\end{equation}
The simple elements decomposition \eqref{decomp2} leads to
\begin{equation}
I_3^*=\frac{b}{a+b}\int_{-\infty}^{+\infty} \frac{y^2}{(y+ib)\sqrt{\nuti^2+y^2}} \, dy +\frac{a}{a+b}\int_{-\infty}^{+\infty} \frac{y^2}{(y-ia)\sqrt{\nuti^2+y^2}} \, dy
\label{decompI3}
\end{equation}

Once again, variable change \eqref{changevar} is applied
\begin{equation}
 \int_{-\infty}^{+\infty} \frac{y^2}{(y-ia)\sqrt{\nuti^2+y^2}} \,dy = \int_{-1}^1 \frac{8\nuti^2t^2}{(1-t^2)^2(2\nuti t -ia(1-t^2))} \, dt
\end{equation}

The integrated functions can be decomposed as such :
\begin{equation}
\frac{t^2}{(1-t^2)^2(2\nuti t -ia(1-t^2))}=\frac{\alpha}{1-t}+\frac{\beta}{1+t}+ \frac{\gamma}{(1-t)^2}+\frac{\delta}{(1+t)^2} +\frac{\lambda t + \rho }{2\nuti t-ia(1-t^2)},
\label{elemsimpl3}
\end{equation}
where the coefficients $\alpha,\beta,\gamma,\delta,\lambda,\rho$ will be determined in the sequel.

To determine $\gamma$, \eqref{elemsimpl3} is multiplied by $(1-t)^2$ and the result is evaluated at $t=1$. Similarly, $\delta$ is determined by multiplying \eqref{elemsimpl3} by $(1+t)^2$ and evaluating the result at $t=-1$ :
\begin{subequations}
\begin{equation}
\gamma=\frac{1}{8\nuti}
\end{equation}
\begin{equation}
\delta = -\frac{1}{8\nuti}
\end{equation}
\end{subequations}

The remaining terms are
\begin{equation}
\begin{split}
\frac{\alpha}{1-t}+\frac{\beta}{1+t}&+ \frac{\lambda t + \rho }{2\nuti t-ia(1-t^2)}\\  
~\\
&=\frac{t^2}{(1-t^2)^2(2\nuti t -ia(1-t^2))}-\frac{1}{8\nuti(1-t)^2}+\frac{1}{8\nuti(1+t)^2}  \\
~\\
&=\frac{8\nuti t^2-(1+t)^2(2\nuti t-ia(1-t^2))+(1-t)^2(2\nuti t-ia(1-t^2))}{8\nuti(1-t^2)^2(2\nuti_*-ia(1-t^2))}\\
~\\
&=\frac{iat}{2\nuti(1-t^2)(2\nuti t-ia(1-t^2))}
\end{split}
\end{equation}

To determine $\alpha$, \eqref{elemsimpl3} is multiplied by $(1-t)$ and the result is evaluated at $t=1$. Similarly, $\beta$ is determined by multiplying \eqref{elemsimpl3} by $(1+t)$ and evaluating the result at $t=-1$ :
\begin{subequations}
\begin{equation}
\alpha=\frac{ia}{8\nuti^2}
\end{equation}
\begin{equation}
\beta=\frac{ia}{8\nuti^2}
\end{equation}
\end{subequations}

The remaining term is
\begin{equation}
\begin{split}
 \frac{\lambda t + \rho }{2\nuti t-ia(1-t^2)} &=\frac{iat}{2\nuti(1-t^2)(2\nuti-ia(1-t^2))} -\frac{ia}{8\nuti^2(1-t)}-\frac{ia}{8\nuti^2(1+t)} \\
&= \frac{2\nuti iat-ia(2\nuti t-ia(1-t^2))}{4\nuti^2(1-t^2)(2\nuti t-ia(1-t^2))} \\
&=\frac{-a^2}{4\nuti ^2(2\nuti t-ia(1-t^2))}
\end{split}
\end{equation}
The final coefficients can now be determined by a simple identification :
\begin{subequations}
\begin{equation}
\lambda=0
\end{equation}
\begin{equation}
\rho=-\frac{a^2}{4\nuti^2}
\end{equation}
\end{subequations}

Finally, equation \eqref{intQ} can be applied :
\begin{equation}
\int_{-\infty}^{+\infty} \frac{y^2}{(y-ia)\sqrt{\nuti^2+y^2}} \, dy =2ia\log\left(\frac{1+t_{\nuti}}{1-t_{\nuti}}\right)-2i\frac{a^2}{\nuti}\rog{\frac{a}{\nuti}}
\label{y2_D}
\end{equation}
The value of $\int_{-\infty}^{+\infty} \dfrac{y^2}{(y+ib)\sqrt{\nuti^2+y^2}} \, dy $ can be obtained by taking the complex conjugate of  $\int_{-\infty}^{+\infty} \dfrac{y^2}{(y-ia)\sqrt{\nuti^2+y^2}} \, dy $ replacing $a$ with $b$ in the result. The final result can then be obtained using \eqref{decompI3} :
\begin{equation}
\begin{split}
I_3^*&=i(a-b)\int_{-1}^1 \left( \frac{1}{1-t}+\frac{1}{1+t} \right) \, dt+ 2\nuti \int_{-1}^1 \left( \frac{1}{(1-t)^2}-\frac{1}{(1+t)^2} \right) \, dt \\
 &-\frac{2a^3}{a+b}\int_{-1}^1\frac{dt}{2\nuti t-ia(1-t^2)}-\frac{2b^3}{a+b}\int_{-1}^1\frac{dt}{2\nuti t+ib(1-t^2)}\\
 ~\\
 &=2i(a-b)\log\left(\frac{1+t_{\nuti}}{1-t_{\nuti}}\right)-\frac{2ia^3}{\nuti(a+b)}\mbox{rog}(a/\nuti)+\frac{2ib^3}{\nuti(a+b)}\mbox{rog}(b/\nuti) 
\end{split}
\label{valI3}
\end{equation}
Note the appearance of the diverging term $\log\left(\dfrac{1+t_{\nuti}}{1-t_{\nuti}}\right)$. This term will be compensated by another in the final expression of the coefficients of matrix $\mathbb{D}$. In fact, \eqref{changevar} leads to :
\begin{equation}
\frac{2t_{\nuti}}{1-t_{\nuti}^2}=\frac{A}{\nuti} \mbox{ and } A\rightarrow + \infty \mbox{ when } t_{\nuti} \rightarrow 1
\end{equation}
and
\begin{equation}
1-t_{\nuti} =\frac{(1-t_{\nuti})(1+t_{\nuti})}{1+t_{\nuti}} \sim \frac{\nuti}{A}
\end{equation}
so that
\begin{equation}
\ln\left(\frac{1+t_{\nuti_L}}{1-t_{\nuti_L}}\right)- \ln\left(\frac{1+t_{\nuti_T}}{1-t_{\nuti_T}}\right)\sim \ln\left(\frac{\nuti_T}{\nuti_L}\right) 
\label{compensation}
\end{equation}

\section{Integral $I_4^*$}
\label{calcI4}
Integral $I_4^*$ is defined by :
\begin{equation}
I_4^*=\int_{-\infty}^{+\infty} \dfrac{dy}{(y+ib)(y-ia)\sqrt{\nuti^2+y^2}}
\label{defI4}
\end{equation}
For $a+b\neq0$, 
\begin{equation}
    \frac{1}{(y+ib)(y-ia)}=\frac{-i}{a+b}\left( \frac{1}{y-ia}-\frac{1}{y+ib}\right)
    \label{decomp1}
\end{equation}
Substituting the above decomposition in \eqref{defI4}, we get
\begin{equation}
I_4^*=\frac{i}{a+b} \int_{-\infty}^{+\infty} \left(\frac{1}{(y+ib)\sqrt{\nuti^2+y^2}}-\frac{1}{(y-ia)\sqrt{\nuti^2+y^2}} \right)\,dy
\end{equation}
These integrals have been computed in section \ref{calculIntQ}. Using \eqref{intQ}, we get :
\begin{equation}
I_4^*=\frac{2}{\nuti(a+b)}\left(\rog{a/\nuti}+\rog{b/\nuti} \right)
\label{valI4}
\end{equation}

\section{Integral $I_5^*$}
\label{calcI5}

Integral $I_5^*$ is defined by :
\begin{equation}
I_5^*=\int_{-\infty}^{+\infty} \frac{y^2}{(y+ib)(y-ia)\sqrt{\nuti^2+y^2}}\,dy
\end{equation}
Using decomposition \eqref{decomp1} we get
\begin{equation}
I_5^*=\frac{i}{a+b}\int_{-\infty}^{+\infty}\left( \frac{y^2}{(y+ib)\sqrt{\nuti^2+y^2}}-\frac{y^2}{(y-ia)\sqrt{\nuti^2+y^2}}\right)\,dy
\end{equation}
These integrals have been computed at section \ref{calcI3}. Applying formula \eqref{y2_D}, we get :
% En appliquant le changement de variables \eqref{changevar}, on a
% $$\int_{-\infty}^{+\infty}\frac{y}{(y-ia)\sqrt{\nuti_*^2+y^2}} \, dy=\int_{-1}^1 \frac{4\nuti t}{(1-t^2)(2\nuti t-ia(1-t^2))}\,dt$$
% On cherche la décomposition en éléments simples de l'intégrande :
% $$\frac{2\nuti t}{(1-t^2)(2\nuti t-ia(1-t^2)} = \frac{\alpha}{1-t}+\frac{\beta}{1+t}+\frac{\lambda t+\rho}{Q(t)}$$
% On a alors
% $$\alpha=\beta=1$$
% et 
% $$\lambda=0, \, \, \; \rho=2ia$$
% La formule \eqref{intQ} nous donne finalement :
\begin{equation}
I_5^*=2\log\left(\frac{1+t_{\nuti}}{1-t_{\nuti}}\right)-\frac{2}{\nuti(a+b)}\lbrack b^2\rog{b/\nuti}+a^2\rog{a/\nuti}\rbrack
\label{intI5}
\end{equation}

\section{Continuation and conclusion of the computation of the coefficients of $\mathbb{D}(a,b)$}
\label{fincalculs}
The final steps of the computation of the coefficients of matrices $\mathbb{D}_{lk}$ are given here. For each physical configuration presented in this manuscript, the coefficients can be expressed as linear combinations of integrals $I_1^*$ to $I_5^*$. These combinations are given case by case in the following.

\subsection{Acoustic case}
\label{finalDac}
In the second chapter of this manuscript, which deals with the diffraction of an acoustic wave, the expression of operator $\mathcal{D}(a,b)$ depends on whether the wedge is soft (Dirichlet boundary conditions) or hard (Neumann boundary conditions). Let us begin with the case of a soft wedge.  
\subsubsection{Dirichlet boundary conditions}
\label{finalDacDir}
In the case of Dirichlet boundary conditions, the expression of function $\mathcal{D}(a,b)$ is obtained by substituting \eqref{dmDir} in \eqref{ldbis} for $a>1$ and $b>1$ :
\begin{equation}
\mathcal{D}(a,b) = \int_{-\infty}^{+\infty} \dfrac{1}{ y+ib} \, \dfrac{1}{y -i a} \,\dfrac{1}{\zeta_0^0(iy)} \, dy . 
\end{equation}
According to \eqref{zeta_function}, the above expression can be simplified using the relation
\begin{equation}
\zeta_0^0(iy)= - \sqrt{1+y^2}.
\label{zeta0iy}
\end{equation}
By setting $\nuti=1$ in \eqref{defI4}, we find :
\begin{equation}
\mathcal{T}(a,b)=-I_4^1
\end{equation}
This integral is computed in section \ref{calcI4} and its value is given by \eqref{valI4}.
\subsubsection{Neumann boundary conditions}
\label{finalDacNeu}
In the case of Neumann boundary conditions, the expression of function $\mathcal{D}(a,b)$ is obtained by substituting \eqref{dmNeu} in \eqref{ldbis} for $a>1$ and $b>1$ :
\begin{equation}
\mathcal{D}(a,b) = \int_{-\infty}^{+\infty} \dfrac{dy}{ (y+ib)(y-ia)} \, dy .
\end{equation}
The result can be computed directly, by using Cauchy's residue theorem, yielding :
\begin{equation}
\mathcal{D}(a,b)=\dfrac{2\pi}{a+b}
\end{equation}

\subsection{2D elastic case}
\label{finalD2D}
\subsection{3D elastic case}
\label{finalD3D}