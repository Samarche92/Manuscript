\chapter{Computation details for the coefficients of matrix $\mathbb{D}(a,b)$}
\label{matD}
Ces intégrales interviennent dans le calcul des coefficients de la matrice $\mathbb{D}$. Nous donnons ici les développements permettant d'obtenir leurs valeurs.
\section{Calcul de $I_1^*$}
\label{calcI1}
Integral $I_1^*$ is defined by :
\begin{equation}
\label{defI1}
I_1^*=\int_{-\infty}^{+\infty} \frac{y}{(y+ib)(y-ia)\sqrt{\nuti_*^2+y^2}}\,dy
\end{equation}

Lorsque $a+b \neq 0$, on a la décomposition en éléments simples suivante :
\begin{equation}
\frac{y}{(y+ib)(y-ia)}=\frac{1}{a+b} \left( \frac{b}{y+ib}+\frac{a}{y-ia} \right) 
\label{decomp2}
\end{equation}
Ce qui donne : 
\begin{equation}
I_1^*= \frac{1}{a+b} \left( \int_{-\infty}^{+\infty} \frac{b}{(y+ib)\sqrt{\nuti_*^2+y^2}} \,dy +\int_{-\infty}^{+\infty} \frac{a}{(y-ia)\sqrt{\nuti_*^2+y^2}} \,dy \right)
\end{equation}
On effectue le changement de variables suivant :
\begin{equation}
\begin{split}
 y&=2\nuti_* \frac{t}{1-t^2} \\
\sqrt{\nuti_*^2+y^2}&=\nuti_*\left( \frac{1+t^2}{1-t^2} \right)  \\
 dy&= 2\nuti_* \frac{1+t^2}{(1-t^2)^2}dt 
\end{split}
\label{changevar}
\end{equation}
Ce qui donne :
\begin{equation*}
\int_{-\infty}^{+\infty} \frac{dy}{(y-ia)\sqrt{\nuti_*^2+y^2}}=\int_{-1}^1\frac{2\,dt}{2\nuti_*t-ia(1-t^2)}
\end{equation*}
On applique la formule (\ref{intQ}) :
\begin{equation}
I_1^*=\frac{2ia}{(a+b)\nuti_*}\mbox{rog}(a/\nuti_*)-\frac{2ib}{(a+b)\nuti_*}\mbox{rog}(b/\nuti_*)
\label{valI1}
\end{equation}

\section{Calcul de $I_3^*$}
\label{calcI3}
En utilisant la formule \eqref{decomp2} on a :
\begin{equation*}
\begin{split}
I_3^*&=\int_{-\infty}^{+\infty} \frac{y^3}{(y+ib)(y-ia)\sqrt{\nuti_*^2+y^2}} \,dy \\
&=\frac{b}{a+b}\int_{-\infty}^{+\infty} \frac{y^2}{(y+ib)\sqrt{\nuti_*^2+y^2}} \, dy +\frac{a}{a+b}\int_{-\infty}^{+\infty} \frac{y^2}{(y-ia)\sqrt{\nuti_*^2+y^2}} \, dy
\end{split}
\end{equation*}
On effectue le changement de variables \eqref{changevar} :
\begin{equation}
 \int_{-\infty}^{+\infty} \frac{y^2}{(y-ia)\sqrt{\nuti_*^2+y^2}} \,dy = \int_{-1}^1 \frac{8\nuti_*^2t^2}{(1-t^2)^2(2\nuti_* t -ia(1-t^2))} \, dt
\end{equation}

On cherche la décomposition en éléments simples de l'intégrande :
\begin{equation*}
\frac{t^2}{(1-t^2)^2(2\nuti_* t -ia(1-t^2))}=\frac{\alpha}{1-t}+\frac{\beta}{1+t}+ \frac{\gamma}{(1-t)^2}+\frac{\delta}{(1+t)^2} +\frac{\lambda t + \rho }{2\nuti_* t-ia(1-t^2)}
\end{equation*}

On a en multipliant à gauche et à droite successivement par $(1-t)^2$ puis par $(1+t)^2$ et en évaluant le résultat respectivement en $1$ puis en $-1$ :
$$ \gamma=\frac{1}{8\nuti_*} $$
$$\delta = -\frac{1}{8\nuti_*} $$
Notons que la contribution de ces élements simples à l'intégrale finale sera donc nulle. En effet :
$$ \frac{\gamma}{(1-t)^2}+\frac{\delta}{(1+t)^2}=\frac{1}{8\nuti_*}\left( \frac{1}{(1-t)^2}-\frac{1}{(1+t)^2}\right) $$
est une fonction impaire. Son intégrale sur le domaine $ (-1,1)$ qui est symétrique par rapport à 0 est donc nulle.

Il nous reste :
\begin{equation*}
\begin{split}
\frac{\alpha}{1-t}+\frac{\beta}{1+t}&+ \frac{\lambda t + \rho }{2\nuti_* t-ia(1-t^2)}\\  
~\\
&=\frac{t^2}{(1-t^2)^2(2\nuti_* t -ia(1-t^2))}-\frac{1}{8\nuti_*(1-t)^2}+\frac{1}{8\nuti_*(1+t)^2}  \\
~\\
&=\frac{8\nuti_*t^2-(1+t)^2(2\nuti_*t-ia(1-t^2))+(1-t)^2(2\nuti_*t-ia(1-t^2))}{8\nuti_*(1-t^2)^2(2\nuti_*-ia(1-t^2))}\\
~\\
&=\frac{iat}{2\nuti_*(1-t^2)(2\nuti_*t-ia(1-t^2))}
\end{split}
\end{equation*}

Comme précédemment, on a :
$$ \alpha=\frac{ia}{8\nuti_*^2}$$
$$ \beta=\frac{ia}{8\nuti_*^2}$$
Ce qui nous donne maintenant :
\begin{equation*}
\begin{split}
 \frac{\lambda t + \rho }{2\nuti_* t-ia(1-t^2)} &=\frac{iat}{2\nuti_*(1-t^2)(2\nuti_*-ia(1-t^2))} -\frac{ia}{8\nuti_*^2(1-t)}-\frac{ia}{8\nuti_*^2(1+t)} \\
&= \frac{2\nuti_* iat-ia(2\nuti_*t-ia(1-t^2))}{4\nuti_*^2(1-t^2)(2\nuti_*t-ia(1-t^2))} \\
&=\frac{-a^2}{4\nuti_*^2(2\nuti_*t-ia(1-t^2))}
\end{split}
\end{equation*}
Soit 
$$\lambda=0$$
et
$$\rho=-\frac{a^2}{4\nuti_*^2}$$
On peut donc appliquer \eqref{intQ} pour obtenir :
\begin{equation}
\int_{-\infty}^{+\infty} \frac{y^2}{(y-ia)\sqrt{\nuti_*^2+y^2}} \, dy =2ia\log\left(\frac{1+t_{\nuti_*}}{1-t_{\nuti_*}}\right)-2i\frac{a^2}{\nuti_*}\rog{\frac{a}{\nuti_*}}
\label{y2_D}
\end{equation}
On obtient la valeur de  $\int_{-\infty}^{+\infty} \frac{y^2}{(y+ib)\sqrt{\nuti_*^2+y^2}} \, dy $ nécessaire au calcul de $I_3$ en prenant le conjugué de $\int_{-\infty}^{+\infty} \frac{y^2}{(y-ia)\sqrt{\nuti_*^2+y^2}} \, dy $ et en remplaçant $a$ par $b$. On somme ensuite ces résultats et on obtient finalement :
\begin{equation*}
\begin{split}
I_3^*&=i(a-b)\int_{-1}^1 \left( \frac{1}{1-t}+\frac{1}{1+t} \right) \, dt+ 2\nuti_* \int_{-1}^1 \left( \frac{1}{(1-t)^2}-\frac{1}{(1+t)^2} \right) \, dt \\
 &-\frac{2a^3}{a+b}\int_{-1}^1\frac{dt}{2\nuti_*t-ia(1-t^2)}-\frac{2b^3}{a+b}\int_{-1}^1\frac{dt}{2\nuti_*t+ib(1-t^2)}\\
 ~\\
 &=2i(a-b)\log\left(\frac{1+t_{\nuti_*}}{1-t_{\nuti_*}}\right)-\frac{2ia^3}{\nuti_*(a+b)}\mbox{rog}(a/\nuti_*)+\frac{2ib^3}{\nuti_*(a+b)}\mbox{rog}(b/\nuti_*) 
\end{split}
\end{equation*}
Or on a, en utilisant (\ref{changevar}) :
$$ \frac{2t_{\nuti_*}}{1-t_{\nuti_*}^2}=\frac{A}{\nuti_*} \mbox{ et } A\rightarrow + \infty \mbox{ si et seulement si } t_{\nuti_*} \rightarrow 1$$
Le terme divergent se compense avec un autre dans le calcul final du coefficient. En effet :
$$1-t_{\nuti} =\frac{(1-t_{\nuti})(1+t_{\nuti})}{1+t_{\nuti}} \sim \frac{\nuti}{A}$$
donc
\begin{equation}
\ln\left(\frac{1+t_{\nuti_L}}{1-t_{\nuti_L}}\right)- \ln\left(\frac{1+t_{\nuti_T}}{1-t_{\nuti_T}}\right)\sim \ln\left(\frac{\nuti_T}{\nuti_L}\right) 
\label{compensation}
\end{equation}
\section{Calcul de $I_4^*$}
\label{calcI4}
Integral $I_4^*$ is defined by :
\begin{equation}
I_4^*=\int_{-\infty}^{+\infty} \dfrac{dy}{(y+ib)(y-ia)\sqrt{\nuti^2+y^2}}
\label{defI4}
\end{equation}
On a lorsque $a+b\neq0$ :
\begin{equation}
    \frac{1}{(y+ib)(y-ia)}=\frac{-i}{a+b}\left( \frac{1}{y-ia}-\frac{1}{y+ib}\right)
    \label{decomp1}
\end{equation}
ce qui donne
\begin{equation*}
I_4^*=\frac{i}{a+b} \int_{-\infty}^{+\infty} \left(\frac{1}{(y+ib)\sqrt{\nuti_*^2+y^2}}-\frac{1}{(y-ia)\sqrt{\nuti_*^2+y^2}} \right)\,dy
\end{equation*}
Ces intégrales ont déjà été calculées précédemment, en utilisant les resultats de \ref{calculIntQ} on obtient finalement :
\begin{equation}
I_4^*=\frac{2}{\nuti(a+b)}\left(\rog{a/\nuti_*}+\rog{b/\nuti_*} \right)
\label{valI4}
\end{equation}
\section{Calcul de $I_5^*$}
\label{calcI5}
On rappelle:
\begin{equation}
I_5^*=\int_{-\infty}^{+\infty} \frac{y^2}{(y+ib)(y-ia)\sqrt{\nuti_*^2+y^2}}\,dy
\end{equation}
La formule \eqref{decomp2} donne 
$$ I_5^*=\frac{i}{a+b}\int_{-\infty}^{+\infty}\left( \frac{y^2}{(y+ib)\sqrt{\nuti^2_*+y^2}}-\frac{y^2}{(y-ia)\sqrt{\nuti_*^2+y^2}}\right)\,dy $$
On reconnait le calcul mené précédemment. En appliquant la formule \eqref{y2_D}, on obtient :
% En appliquant le changement de variables \eqref{changevar}, on a
% $$\int_{-\infty}^{+\infty}\frac{y}{(y-ia)\sqrt{\nuti_*^2+y^2}} \, dy=\int_{-1}^1 \frac{4\nuti t}{(1-t^2)(2\nuti t-ia(1-t^2))}\,dt$$
% On cherche la décomposition en éléments simples de l'intégrande :
% $$\frac{2\nuti t}{(1-t^2)(2\nuti t-ia(1-t^2)} = \frac{\alpha}{1-t}+\frac{\beta}{1+t}+\frac{\lambda t+\rho}{Q(t)}$$
% On a alors
% $$\alpha=\beta=1$$
% et 
% $$\lambda=0, \, \, \; \rho=2ia$$
% La formule \eqref{intQ} nous donne finalement :
\begin{equation}
I_5^*=2\log\left(\frac{1+t_{\nuti^*}}{1-t_{\nuti^*}}\right)-\frac{2}{\nuti_*(a+b)}\lbrack b^2\rog{b/\nuti_*}+a^2\rog{a/\nuti_*}\rbrack
\label{intI5}
\end{equation}

\section{Continuation and conclusion of the computation of the coefficients of $\mathbb{D}(a,b)$}
\label{fincalculs}
The final steps of the computation of the coefficients of matrices $\mathbb{D}_{lk}$ are given here. For each physical configuration presented in this manuscript, the coefficients can be expressed as linear combinations of integrals $I_1^*$ to $I_5^*$. These combinations are given case by case in the following.

\subsection{Acoustic case}
\label{finalDac}
In the second chapter of this manuscript, which deals with the diffraction of an acoustic wave, the expression of operator $\mathcal{D}(a,b)$ depends on whether the wedge is soft (Dirichlet boundary conditions) or hard (Neumann boundary conditions). Let us begin with the case of a soft wedge.  
\subsubsection{Dirichlet boundary conditions}
\label{finalDacDir}
In the case of Dirichlet boundary conditions, the expression of function $\mathcal{D}(a,b)$ is obtained by substituting \eqref{dmDir} in \eqref{ldbis} for $a>1$ and $b>1$ :
\begin{equation}
\mathcal{D}(a,b) = \int_{-\infty}^{+\infty} \dfrac{1}{ y+ib} \, \dfrac{1}{y -i a} \,\dfrac{1}{\zeta_0^0(iy)} \, dy . 
\end{equation}
According to \eqref{zeta_function}, the above expression can be simplified using the relation
\begin{equation}
\zeta_0^0(iy)= - \sqrt{1+y^2}.
\label{zeta0iy}
\end{equation}
By setting $\nuti=1$ in \eqref{defI4}, we find :
\begin{equation}
\mathcal{T}(a,b)=-I_4^1
\end{equation}
This integral is computed in section \ref{calcI4} and its value is given by \eqref{valI4}.
\subsubsection{Neumann boundary conditions}
\label{finalDacNeu}
In the case of Neumann boundary conditions, the expression of function $\mathcal{D}(a,b)$ is obtained by substituting \eqref{dmNeu} in \eqref{ldbis} for $a>1$ and $b>1$ :
\begin{equation}
\mathcal{D}(a,b) = \int_{-\infty}^{+\infty} \dfrac{dy}{ (y+ib)(y-ia)} \, dy .
\end{equation}
The result can be computed directly, by using Cauchy's residue theorem, yielding :
\begin{equation}
\mathcal{D}(a,b)=\dfrac{2\pi}{a+b}
\end{equation}

\subsection{2D elastic case}
\label{finalD2D}
\subsection{3D elastic case}
\label{finalD3D}