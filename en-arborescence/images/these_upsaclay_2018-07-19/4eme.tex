%%%%%%%%%%%%%%%%%%%%%%%%%%%%%%%%%%%%%%%%%%%%%%%%%%%%%%%%%%%%%%%%%%%%%%%%%%%%%%%%%%%%%%%%%%%%%%%%%%%%%%%%%%%%%%%%%%%%%%%%%%%%%%%%%%%%%%%%%%%%%%%%%%%%%%%%%%%%%%%%%%%%%%%
%%%%%%%%%%%%%%%%%%%%%%%%%%%%%%%%%%%%%%%%%%%%%%%%%%%%%%%%%%%%%%%%%%%%%%%%%%%%%%%%%%%%%%%%%%%%%%%%%%%%%%%%%%%%%%%%%%%%%%%%%%%%%%%%%%%%%%%%%%%%%%%%%%%%%%%%%%%%%%%%%%%%%%%
%%% Modèle pour la 4ème de couverture des thèses préparées à l'Université Paris-Saclay, basé sur le modèle produit par Nikolas STOTT / Template for back cover of thesis made at Université Paris-Saclay, based on the template made by Nikolas STOTT
%%% Mis à jour par Aurélien ARNOUX (École polytechnique)/ Updated by Aurélien ARNOUX (École polytechnique)
%%% Les instructions concernant chaque donnée à remplir sont données en bloc de commentaire / Rules to fill this file are given in comment blocks
%%% ATTENTION Ces informations doivent tenir sur une seule page une fois compilées / WARNING These informations must contain in no more than one page once compiled
%%%%%%%%%%%%%%%%%%%%%%%%%%%%%%%%%%%%%%%%%%%%%%%%%%%%%%%%%%%%%%%%%%%%%%%%%%%%%%%%%%%%%%%%%%%%%%%%%%%%%%%%%%%%%%%%%%%%%%%%%%%%%%%%%%%%%%%%%%%%%%%%%%%%%%%%%%%%%%%%%%%%%%%
%%% Version du 19 juillet 2018 (Merci à Hadrien VROYLANDT (Univ. Paris-Sud) pour ses suggestions et corrections)
%%%%%%%%%%%%%%%%%%%%%%%%%%%%%%%%%%%%%%%%%%%%%%%%%%%%%%%%%%%%%%%%%%%%%%%%%%%%%%%%%%%%%%%%%%%%%%%%%%%%%%%%%%%%%%%%%%%%%%%%%%%%%%%%%%%%%%%%%%%%%%%%%%%%%%%%%%%%%%%%%%%%%%%

\documentclass[a4paper]{article}
\usepackage[utf8]{inputenc}
\usepackage{helvet}
\renewcommand{\familydefault}{\sfdefault}
\usepackage{geometry}
\geometry{
left=16mm,
top=30mm,
right=16mm,
bottom=30mm
}
\usepackage{xcolor}
\definecolor{bordeau}{rgb}{0.3515625,0,0.234375}
\usepackage[absolute,overlay]{textpos}
\usepackage{graphicx}
\usepackage{lipsum}
\usepackage{array}
\usepackage{caption}
\usepackage{multicol}
\setlength{\columnseprule}{0pt}
\setlength\columnsep{10pt}

\usepackage[french]{babel}

\label{form}
%%%%%%%%%%%%%%%%%%%%%%%%%%%%%%%%%%%%%%%%%%%%%%%%%%%%%%%%%%%%%%%%%%%%%%%%%%%%%%%%%%%%%%%%%%%%%%%%%%%%%%%%%%%%%%%%%%%%%%%%%%%%%%%%%%%%%%%%%%%%%%%%%%%%%%%%%%%%%%%%%%%%%%%
%%%%%%%%%%%%%%%%%%%%%%%%%%%%%%%%%%%%%%%%%%%%%%%%%%%%%%%%%%%%%%%%%%%%%%%%%%%%%%%%%%%%%%%%%%%%%%%%%%%%%%%%%%%%%%%%%%%%%%%%%%%%%%%%%%%%%%%%%%%%%%%%%%%%%%%%%%%%%%%%%%%%%%%
%%% Formulaire / Form
%%% Remplacer les paramètres des \newcommand par les informations demandées / Replace \newcommand parameters by asked informations
%%%%%%%%%%%%%%%%%%%%%%%%%%%%%%%%%%%%%%%%%%%%%%%%%%%%%%%%%%%%%%%%%%%%%%%%%%%%%%%%%%%%%%%%%%%%%%%%%%%%%%%%%%%%%%%%%%%%%%%%%%%%%%%%%%%%%%%%%%%%%%%%%%%%%%%%%%%%%%%%%%%%%%%
%%%%%%%%%%%%%%%%%%%%%%%%%%%%%%%%%%%%%%%%%%%%%%%%%%%%%%%%%%%%%%%%%%%%%%%%%%%%%%%%%%%%%%%%%%%%%%%%%%%%%%%%%%%%%%%%%%%%%%%%%%%%%%%%%%%%%%%%%%%%%%%%%%%%%%%%%%%%%%%%%%%%%%%

\newcommand{\logoEd}{EOBE}																		%% Logo de l'école doctorale. Indiquer le sigle / Doctoral school logo. Indicate the acronym : 2MIB; AAIF; ABIES; BIOSIGNE; CBMS; EDMH; EDOM; EDPIF; EDSP; EOBE; INTERFACES; ITFA; PHENIICS; SDSV; SDV; SHS; SMEMAG; SSMMH; STIC
\newcommand{\PhDTitleFR}{Modélisation de la diffusion 3D d'ondes élastiques par des structures complexes pour le calcul des échos de géométrie. Application à la simulation des CND par ultrasons.}													%% Titre de la thèse en français / Thesis title in french
\newcommand{\keywordsFR}{Elastodynamique, Diffraction, Méthodes asymptotiques}														%% Mots clés en français, séprarés par des , / Keywords in french, separated by ,
\newcommand{\abstractFR}{\lipsum[1-3]}															%% Résumé en français / abstract in french

\newcommand{\PhDTitleEN}{Modelling of the 3D scattering of elastic waves by complex structures for specimen echoes calculation. Application to ultrasonic NDT simulation.}													%% Titre de la thèse en anglais / Thesis title in english
\newcommand{\keywordsEN}{Elastodynamics, Diffraction, Asymptotic Methods}														%% Mots clés en anglais, séprarés par des , / Keywords in english, separated by ,
\newcommand{\abstractEN}{\lipsum[1-3]}															%% Résumé en anglais / abstract in english

\label{layout}
%%%%%%%%%%%%%%%%%%%%%%%%%%%%%%%%%%%%%%%%%%%%%%%%%%%%%%%%%%%%%%%%%%%%%%%%%%%%%%%%%%%%%%%%%%%%%%%%%%%%%%%%%%%%%%%%%%%%%%%%%%%%%%%%%%%%%%%%%%%%%%%%%%%%%%%%%%%%%%%%%%%%%%%
%%%%%%%%%%%%%%%%%%%%%%%%%%%%%%%%%%%%%%%%%%%%%%%%%%%%%%%%%%%%%%%%%%%%%%%%%%%%%%%%%%%%%%%%%%%%%%%%%%%%%%%%%%%%%%%%%%%%%%%%%%%%%%%%%%%%%%%%%%%%%%%%%%%%%%%%%%%%%%%%%%%%%%%
%%% Mise en page / Page layout      
%%% NE RIEN MODIFIER / DO NOT MODIFY
%%%%%%%%%%%%%%%%%%%%%%%%%%%%%%%%%%%%%%%%%%%%%%%%%%%%%%%%%%%%%%%%%%%%%%%%%%%%%%%%%%%%%%%%%%%%%%%%%%%%%%%%%%%%%%%%%%%%%%%%%%%%%%%%%%%%%%%%%%%%%%%%%%%%%%%%%%%%%%%%%%%%%%%
%%%%%%%%%%%%%%%%%%%%%%%%%%%%%%%%%%%%%%%%%%%%%%%%%%%%%%%%%%%%%%%%%%%%%%%%%%%%%%%%%%%%%%%%%%%%%%%%%%%%%%%%%%%%%%%%%%%%%%%%%%%%%%%%%%%%%%%%%%%%%%%%%%%%%%%%%%%%%%%%%%%%%%%

\begin{document}
\pagestyle{empty}

%%% Logo de l'école doctorale. Le nom du fichier correspond au sigle de l'ED / Doctoral school logo. Filename correspond to doctoral school acronym
%%% Les noms valides sont / Valid names are : 2MIB; AAIF; ABIES; BIOSIGNE; CBMS; EDMH; EDOM; EDPIF; EDSP; EOBE; INTERFACES; ITFA; PHENIICS; SDSV; SDV; SHS; SMEMAG; SSMMH; STIC
\begin{textblock*}{61mm}(16mm,3mm)
	\noindent\includegraphics[height=24mm]{media/ed/\logoEd.jpeg}
\end{textblock*}



%%%Titre de la thèse en français / Thesis title in french
\begin{center}
\fcolorbox{bordeau}{white}{\parbox{0.95\textwidth}{
{\bf Titre:} \PhDTitleFR 
\medskip

%%%Mots clés en français, séprarés par des ; / Keywords in french, separated by ;
{\bf Mots clés:} \keywordsFR 
\vspace{-2mm}

%%% Résumé en français / abstract in french
\begin{multicols}{2}
{\bf Résumé:} 
Le sujet de la thèse s’inscrit dans le cadre du développement de modèles pour la simulation du contrôle non-destructif (CND) par ultrasons. L'objectif à long terme est la mise au point, par une méthode de rayons, d’un outil complet de simulation des échos issus de la géométrie (surfaces d’entrée, de fond…) ou des structures internes des pièces inspectées. La thèse vise plus précisément à intégrer le phénomène de diffraction par les dièdres à un modèle existant dérivant de l’acoustique géométrique et qui prend uniquement en compte les réflexions sur les faces. 

Pour cela, la méthode dite des fonctions spectrales, développée initialement pour le cas d'un dièdre immergé, est développée et validée dans un premier temps dans le cas des ondes acoustiques pour des conditions aux limites de type Dirichlet ou Neumann. La méthode est ensuite étendue à la diffraction des ondes élastiques par des dièdres infinis à faces libres et d'angles quelconques, pour une incidence 2D puis pour une incidence 3D. Cette méthode est semi-analytique puisque les solutions recherchées s'écrivent sous la forme d'une somme d'une fonction singulière, qui est déterminée analytiquement à l'aide d'un algorithme récursif, et d'une fonction régulière, qui est approchée numériquement. 

Les codes correspondants sont validés par comparaison à une solution exacte dans le cas acoustique et par comparaison à d'autres codes (semi-analytiques et numériques) dans le cas élastique. Des validations expérimentales du modèle élastodynamique sont également proposées.
\end{multicols}
}}
\end{center}

\vspace*{0mm}

%%%Titre de la thèse en anglais / Thesis title in english
\begin{center}
\fcolorbox{bordeau}{white}{\parbox{0.95\textwidth}{
{\bf Title:} \PhDTitleEN 

\medskip

%%%Mots clés en anglais, séprarés par des ; / Keywords in english, separated by ;
{\bf Keywords:}  \keywordsEN %%3 à 6 mots clés%%
\vspace{-2mm}
\begin{multicols}{2}
	
%%% Résumé en anglais / abstract in english
{\bf Abstract:} 
This thesis falls into the framework of model development for simulation of ultrasonic non-destructive testing (NDT). The long-term goal is to develop, using ray methods, a complete simulation tool of specimen echoes (input, back-wall surfaces...) or echoes of inner structures of inspected parts. The thesis aims more specifically to integrate the phenomenon of diffraction by wedges to an existing model derived from geometrical acoustics, which only accounts for reflections on the wedge faces.

To this end, a method called the spectral functions method, which was initially developed for immersed wedges, is developed and validated as a first step in the case of acoustic waves with Dirichlet or Neumann boundary conditions. The method is then extended to elastic wave diffraction by infinite stress-free wedges of arbitrary angles, for 2D and 3D incidences. This method is semi-analytic since the unknown solutions are expressed as the sum of a singular function, determined analytically using a recursive algorithm, and a regular function which is approached numerically.

The corresponding codes are validated by comparison to an exact solution in the acoustic case and by comparison to other codes (semi-analytic and numerical) in the elastic case. Experimental validations of the elastodynamic model are also proposed.
\end{multicols}
}}
\end{center}


\begin{textblock*}{161mm}(10mm,270mm)
\color{bordeau}
{\bf\noindent Université Paris-Saclay	         }

\noindent Espace Technologique / Immeuble Discovery 

\noindent Route de l’Orme aux Merisiers RD 128 / 91190 Saint-Aubin, France 
\end{textblock*}

\begin{textblock*}{20mm}(182mm,255mm)
\includegraphics[width=20mm]{media/UPSACLAY-petit}
\end{textblock*}

\end{document}