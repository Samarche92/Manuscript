\chapter[][Sate of the Art]{State of the art}
\label{chap-biblio}

\section*{Introduction}

\section{Non-Uniform asymptotic methods}

There exist many different high-frequency approximations of the field echoed by the surfaces and interfaces of an inspected specimen. Some of these models lead to a discontinuous scattered field (which is not physical) and are therefore called non-uniform methods, as opposed to uniform models, which lead to continuous solutions.
\subsection{\acrfull{ge}}
The first approximation that can be applied to the study of wave propagation in a complex isotropic medium is the \acrfull{ge} model. It is a translation to elastodynamics of the geometrical optics theory. The field's propagation is described by ray tracing, each ray carrying a certain field value. At a given observation point, the field's value is the sum of the field values carried by each of the rays passing through this point. In the \acrshort{ge} theory, the incident, reflected and refracted rays are described. These rays are computed following Snell's laws of reflection and refraction :
\begin{equation}
    \frac{1}{c_{\alpha}}\cos\theta_{\alpha} = \frac{1}{c_{\beta}} \cos\theta_{\beta}
    \label{Snellrefl}
\end{equation}
In all this thesis, $\alpha=L,TH or TV$ represents the type of the incident wave (\acrshort{L} for Longitudinal, \acrshort{TH} for Transverse Horizontal and \acrshort{TV} for Transverse Vertical) and $\beta$ is the type of the reflected, transmitted of diffracted wave.

When the incident wave meets a surface containing an edge, the propagation medium can be decomposed into four zones (see Fig.~\ref{illuzones}) :
\begin{itemize}
    \item Zone II : the incident ray does not cross the scattering surface and propagates freely.
    \item Zone I : the incident rays are "shadowed" by the scattering surface and therefore do not illuminate this zone, called the shadow zone.
    \item Zone III :  the incident rays are reflected by the scattering surface and mode conversion may occur. This zone is illuminated by the incident rays and by the mode-converted reflected rays.
    \item Zone IV : the incident rays are reflected and mode conversion may occur. This zone is illuminated by incident rays and by L and T reflected waves.
\end{itemize}
The boundaries separating each of these zones are called the shadow boundaries. In the case where there is no mode conversion (determined by Snell's law of reflection \eqref{Snellrefl}), there is no Zone III. 

The displacement fields carried by the reflected and refracted rays are proportional to the field incident of the reflecting or refracting interfaces. This proportionality is contained in multiplicative coefficient called reflection or transmission coefficients respectively, which depend on the properties of the propagation medium and on the directions of incidence and observation.

In reality, part of the scattered wave propagates in Zone I. Indeed, part of the incident wave arrives on the edge and is diffracted. This diffracted wave propagates in all directions. The \acrshort{ge} model does not account for the diffracted wave, as they can not be predicted by ray tracing. To complete the \acrshort{ge} model, Keller \cite{GTD} has developed the \acrfull{gtd}.

\begin{figure}
    \centering
    \includegraphics[height=0.33\textheight]{images/chapter1/ShadowBoundary.png}
    \caption{Incident wave on an edge}
    \label{illuzones}
\end{figure}

\subsection{\acrfull{gtd}}
The \acrfull{gtd} was initially developed by Keller \cite{GTD} for optical waves and adapted to elastodynamics by Achenbach and Gautesen \cite{AchenbachGautesen, Achenbach}. this theory postulates the existence of diffracted waves emanating from the edge of the scattering surface. A incident ray on an edge generates a cone of rays, called Keller's cone of diffraction \cite{GTD}, represented on Fig.~\ref{KellerCone}. The cone's principal axis is the diffracting edge, its principal angle is determined by Snell's law of diffraction :
\begin{equation}
    \frac{1}{c_{\alpha}}\cos\Omega_{\alpha} = \frac{1}{c_{\beta}} \cos\Omega_{\beta}
\end{equation}

\begin{figure}
    \centering
    \includegraphics[width=\textwidth]{images/chapter1/KellerCone.png}
    \caption{Diffracted rays generated by an incident ray}
    \label{KellerCone}
\end{figure}

Rahmat-Samii \cite{ConePhoto} has observed this cone in a hotel room, see Fig.~\ref{PhotoCone}. A ray of light is incident on the corner of a table and generates a cone of diffracted rays, whose intersection with the door is a circle.

\begin{figure}
    \centering
    \includegraphics{images/chapter1/HotelCone.png}
    \caption{Observation of Keller's cone of diffraction}
    \label{PhotoCone}
\end{figure}

The \acrshort{gtd} is also a ray tracing method, meaning that at a given observation point $M$, the total field $\mathbf{u^{tot}}$ is the sum of the fields carried by each ray passing through $M$ :
\begin{equation}
    \mathbf{u^{tot}}(M)=\mathbf{u^{(GE)}}(M)+\sum_{\beta} \mathbf{u^{diff}_{\beta}}(M)
\end{equation}
where $u^{(GE)}$ is the \acrshort{ge} displacement field, composed of the incident, reflected and refracted fields and $u^{diff}_{\beta}$ is the diffracted field of type $\beta=L,TH,TV$. In this chapter, the bold font is used to denote vectors. The diffracted field's amplitude decreases as the distance $r$ from the point of impact $x_{\beta}^{\alpha}$ of the incident wave on the diffracting edge grows. As for the \acrshort{ge} field, the diffracted field is proportional to the field incident on the edge and this proportionality is characterized by a multiplicative coefficient called the diffraction coefficient, which depends on the propagation medium, on the geometry of the diffracting object and on the direction of observation. This is summarized by the following equation :
\begin{equation}
    \mathbf{u}_{\beta}^{diff}(M)=u_{\alpha}^{inc}(x_{\beta}^{\alpha})D_{\beta}^{\alpha}(M)\dfrac{e^{ik_{\beta}r}}{\sqrt{k_{\beta}R_{\beta}^{diff}}}\mathbf{e_{\beta}}(M)
\end{equation}
where $u_{\alpha}^{inc}$ is the value of the incident field, $D_{\beta}^{\alpha}$ is the diffraction coefficient, $k_{\beta}$ is the diffracted wave's wave number, $\mathbf{e_{\beta}}(M)$ is the unit polarisation vector of the diffracted wave at point M and $R_{\beta}^{diff}$ is a distance parameter which will be computed later, at section \ref{}.

This principle is called the locality principle, because it stipulates that the value of the field at any given point is fully determined by the field in the close vicinity of the point from which the ray carrying this field emanates. Computation of diffracted fields can therefore be reduced to a number of canonical problems, such as diffraction by a tip or a half plane. In the present thesis, the canonical problem of interest is diffraction by a wedge.

The \acrshort{gtd} is a high-frequency model which accounts for edge-diffracted waves. However, the resulting field is discontinuous at shadow boundaries and is therefore not physical. Some uniform corrections have been proposed to solve this problem, resulting in continuous fields. They are presented in the following. 

\section{Uniform corrections}
%% Maybe details only for UTD. Name PTD and UAT with principle behind it but not details (refer to litterature & AKD for these), give shortcomings and why need for UTD. Then briefly describe UTD. In each case, these are corrections of a GTD method -> need for good GTD

\subsection{\acrfull{ka}}
\subsection{\acrfull{ptd}}
\subsection{\acrfull{uat}}
\subsection{\acrfull{utd}}

\section{Existing \acrshort{gtd} models}
\subsection{The Sommerfeld Integral Method}
\subsection{The Laplace Transform Method}