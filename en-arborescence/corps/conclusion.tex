\chapter*{Conclusion}

The aim of this thesis is to propose and validate a generic and reliable elastodynamic wedge-diffraction model, valid for all wedge angles and for 3D incidences. So far, this had not yet been done in elastodynamics. This is done by extending a method called the \acrfull{sf} method and proposing the corresponding numerical resolution schemes. The principal results of this thesis are summarized in the following.

In the first chapter of this manuscript, a review of high-frequency wedge diffraction models is done. First, the two main non-uniform asymptotic methods are described : \acrfull{ge}, which only models reflected and refracted rays and the \acrfull{gtd}, which accounts models diffraction but diverges at observation directions close to specular reflections. Then, some uniform corrections of these models are presented. The \acrfull{ka}, which produces a uniform scattered field but models diffraction inaccurately, the \acrfull{ptd} which provides a good description of the scattered field in all directions but is computationally expensive, the \acrfull{uat} which also provides a good description of the scattered field but is difficult to implement and finally the \acrfull{utd} which is accurate, simple to implement and computationally cheap. For these reasons, \acrshort{utd} is the preferred uniform asymptotic model for wedge diffraction. Its accuracy relies on the existence of a reliable \acrshort{gtd} wedge diffraction model. With that in mind, the two main existing wedge diffraction models, the \acrfull{lt} method and the \acrfull{si} method, are presented briefly. Neither of these has been developed for elastic waves incident on wedges of angle higher than $\pi$.

In the second chapter of this manuscript, the spectral functions method is developed as a first step in the simpler case of an acoustic wave scattered by a soft (Dirichlet boundary conditions) or hard (Neumann boundary conditions) wedge. To do so, an integral formulation of the solution to the scattering problem is derived. This formulation is given with respect to two unknown functions called the spectral functions. A far-field asymptotic evaluation of this integral formulation leads to an expression of the \acrshort{gtd} diffraction coefficient as a functino of the spectral functions. The integral formulation is then injected into the problem's boundary conditions, yielding an integtral system of functional equations of which the spectral functions are the solution. This system is then solved semi-analytically. This means that the spectral functions are decomposed as the sum of two terms : a singular function, which is determined analytically thanks to a recursive algorithm, and a regular function, which is approached numerically thanks to a Galerkin collocation method. Finally, the accuracy of this numerical approximation is improved using a technique called the "propagation of the solution". The method is validated by comparing the \acrshort{gtd} coefficients obtained using the semi-analytical spectral functions method to the exact solution given by Sommerfeld.

In the t

The results obtained during this thesis led to two publications in peer-reviewed journals \cite{article, articleelasto} as well to two communications in international conferences with conference proceedings \cite{DD2018,AFPAC}.

Some suggestions for future work are given below :
\begin{itemize}
\item In the final chapter of this thesis, the regular part of the spectral functions diverges in certain cases. Further investigations need to be made in order to find the cause of these divergences and a new method of computation needs to be proposed.
\item A thorough numerical and/or experimental validation of the code implemented to treat the case of 3D diffraction of an elastic wave must be conducted.
\item The \acrlong{sf} method could be further extended to treat dihedral interfaces between two elastic materials. This would be the continuity of Lucien Rochery's internship.
\item The \acrshort{utd} model was developed by Audrey Kamta-Djakou \cite{AKDthese} using the \acrfull{si} pole propagation algorithm. It should be adapted to the \acrshort{sf} method so it can be applied to 3D incidences.
\item The elastodynamic diffraction coefficients present a slight discontinuity at critical angles, due to the presence of head waves. In the continuity of Fradkin et al. \cite{FradkinDarmon}, further investigations must be made in order to model the contribution of these waves correctly.
\item In the final chapter, it was shown that for transversal incident waves with a skew angle higher than the critical angle, an evanescent longitudinal wave is produced. The asymptotic contribution of this wave should be evaluated more precisely.
\item Following the ideas of Kamotskii \cite{Kamotski2}, the Spectral Functions method could also be adapted to treat scattering by multiple adjacent wedges.
\end{itemize}