\chapter*{Conclusion and future work}

The aim of this thesis is to propose and validate a generic and reliable elastodynamic diffraction model for infinite stress-free wedges, valid for all wedge angles and for 3D incidences, in order to be applied to high frequency simulation of ultrasonic geometry echoes. This is done by extending a method called the \acrfull{sf} method and proposing the corresponding numerical resolution schemes. The principal results of this thesis are summarized in the following.

In the first chapter of this manuscript, a review of high-frequency wedge diffraction models is done. First, the two main non-uniform asymptotic methods are described : \acrfull{ge}, which only model reflected and refracted rays and the \acrfull{gtd}, which accounts for diffraction but diverges at observation directions close to specular reflections. Then, some uniform solutions based on these models are presented. The \acrfull{ka}, which produces a uniform scattered field but models diffraction inaccurately, the \acrfull{ptd} which provides a good description of the scattered field in all directions but is computationally expensive for large scatterers, the \acrfull{uat} which also provides a good description of the scattered field but is difficult to implement and finally the \acrfull{utd} which is accurate, simple to implement and computationally cheap. For these reasons, \acrshort{utd} is the preferred uniform asymptotic model to model scattering from structures including wedges. Its accuracy relies on the existence of a reliable \acrshort{gtd} wedge diffraction model. With that in mind, the two main existing wedge diffraction models, the \acrfull{lt} method and the \acrfull{si} method, are presented briefly. The \acrshort{lt} method uses an integral formulation of the components of the displacement field in the entire space to derive a system of functional  equations of which the Laplace transform of the displacement field is the solution. The \acrshort{si} method uses Sommerfeld's exact expression of the elastodynamic potentials in the form of integrals to derive a different system of functional equations. In both cases, the corresponding systems are solved by decomposing the solutions as the sum of a singular function determined analytically and a regular function approached numerically. Neither method has been developed for an elastic wave incident on a wedge of angle higher than $\pi$.

In the second chapter of this manuscript, the \acrfull{sf} method is developed as a first step in the simpler case of an acoustic wave scattered by a soft (Dirichlet boundary conditions) or hard (Neumann boundary conditions) wedge of arbitrary angle. Similarly to the \acrshort{lt} and \acrshort{si} methods, the \acrshort{sf} method uses an integral formulation of the solution to derive a system of functional equations which is then solved semi-analytically by decomposing the solution as the sum of a singular function and a regular function. However, as opposed to the \acrshort{lt} and \acrshort{si} methods, the \acrshort{sf} method is valid for all wedge angles, including those higher than $\pi$. This is achieved by introducing a new angular variable $\phiti$ which depends on the wedge angle $\varphi$ but has a different expression for $\varphi \leq \pi$ and $\varphi>\pi$. In the acoustic version of the \acrshort{sf} method, the aforementioned integral formulation of the solution to the scattering problem is derived using a Fourier transform of the Helmoltz equation. This formulation is given with respect to two unknown functions called the spectral functions. A far-field asymptotic evaluation of this integral formulation leads to an expression of the \acrshort{gtd} diffraction coefficient as a function of the spectral functions. The integral formulation is then injected into the problem's boundary conditions, yielding an integral system of functional equations of which the spectral functions are the solution. This system is then solved semi-analytically. This means that the spectral functions are decomposed as the sum of two terms : a singular function, which is determined analytically thanks to a recursive algorithm, and a regular function, which is approached numerically thanks to a Galerkin collocation method. Finally, the accuracy of the numerical approximation of the regular part is improved using a technique called the "propagation of the solution". The method is successfully validated by comparing the \acrshort{gtd} diffraction coefficients obtained using the semi-analytical spectral functions method to the \acrshort{gtd} diffraction coefficients derived from the exact solution given by Sommerfeld. The results obtained using the spectral functions method and those obtained using the analytical formula are almost identical, except for slight discrepancies which appear in certain cases for observation angles close to the wedge faces.

In the third chapter of the manuscript, the spectral functions method is applied to the more complex problem of elastic wave diffraction by a stress-free wedge of arbitrary angle. The main steps of the method are the same as in the previous chapter but the corresponding computations are more complex, since the spectral functions are now two-dimensional vectors and the incident, reflected and edge-diffracted waves can be polarized longitudinally and transversally. These two propagation modes are coupled by the wedge boundary conditions, meaning that mode conversion occurs. For each given configuration, two diffraction coefficients are therefore computed : one for longitudinal diffracted waves and one for transversal diffracted waves. The absolute values of the diffraction coefficients obtained using the \acrfull{sf} code are compared to those obtained using the \acrshort{lt} code, for wedge angles lower than $\pi$ and the results are extremely close. However, the existing \acrshort{lt} code is only valid for wedge angles lower than $\pi$. For wedge angles higher than $\pi$, the absolute values of the diffraction coefficients obtained using the \acrshort{sf} code are compared to the diffraction coefficients extracted from the results of a finite elements code. In regions where the diffracted waves do not interfere with other waves and where the \acrshort{gtd} evaluation is theoretically valid, both codes give very similar results. Finally, the absolute values and angular phases of the diffraction coefficients computed with the \acrshort{sf} code are validated experimentally using the same measurements that were made to validate the \acrshort{lt} code and are compared once more to the results of the \acrshort{lt} code. The results of both codes are identical, except for a discrepancy near the wedge face in one case, and are very close to the experimental measurements.

In the fourth and final chapter of the manuscript, the spectral functions method is applied to the 3D case of elastic wave diffraction by a stress-free wedge, where the incident wave vector is not necessarily in the the plane normal to the wedge edge. In this case, the incident ray on the wedge edge produces a cone of diffracted rays called Keller's cone of diffraction for each scattered mode. The angle of this cone is determined by Snell's law of diffraction. According to Snell's law of diffraction, when the incident wave is transversal and the incident skew angle (i.e. the angle between the incident wave vector and the plane normal to the wedge edge) is higher than a certain angle called the critical angle, there is no diffracted longitudinal wave. The diffracted field then has imaginary branch points and extra care must be taken in dealing with these. The spectral functions method is developed in detail for the 3D case, for all types of incidences and for wedge angles higher and lower than $\pi$. An additional numerical approximation is proposed in order to compute the regular part of the spectral functions in the case of a transversal incident wave with a skew angle higher than the critical angle : the obtained results seem reasonable but have not been tested numerically or experimentally. The 3D spectral functions code is tested successfully in some particular cases. It produces identical results to the 2D code in 2D cases (the skew angle is set to $0$) and to the exact solution for the "acoustic limit" (the longitudinal and transversal wave velocities are set to mimic acoustic wave propagation) and the regular part is well evaluated in the case of an infinite plane (the wedge angle is equal to $\pi$ and there is no diffracted wave), notably after the critical angle when using the previously mentioned additional approximation.

The spectral functions method provides a high frequency approximation of the waves scattered by a wedge. It is semi-analytical and is therefore applicable to configurations where fully numerical methods such as finite elements or finite differences fail because they are too expensive computationally. It can be used to treat the scattering of acoustic waves as well as elastic longitudinal or transversal waves.

The main advantage of the spectral functions method, as opposed to other \acrshort{gtd} wedge-diffraction models, is that it has been developed for wedge angles lower and higher than $\pi$. A code developed for elastic waves can be used to treat the simpler case of acoustic waves. Finally, the method is generic and adaptable to more complex problems. However, the method also has some inconvenients. In some cases, it lacks precision for observation angles close to the wedge faces and it does not deal well with very small (i.e. smaller than $80^o$ in the elastic case) or very large (i.e. larger than $280^o$ in the elastic case) wedge angles. The \acrlong{lt} and \acrlong{si} methods deal better with small wedge andle and with observation angles close to the wedge faces.

The results obtained during this thesis led to three publications in peer-reviewed journals \cite{article, articleelasto, articleITD} as well to two communications in international conferences with peer-reviewed proceedings \cite{DD2018,AFPAC}.

Some suggestions for future work are given below :
\begin{itemize}
\item In the final chapter of this thesis, the regular part of the spectral functions diverges in the case of an incident transversal wave with a skew angle higher than the critical angle for edge diffraction (angle linked to Snell's law of diffraction on the L wave Keller diffraction cone). Further investigations need to be made in order to find the cause of these divergences and a new method of computation could be proposed.
\item A thorough numerical and/or experimental validation of the code implemented to treat the case of 3D diffraction of an elastic wave must be conducted and is currently in progress.
\item The spectral functions method could be further extended to treat dihedral interfaces between two elastic materials. This would be the continuity of Lucien Rochery's internship, which Michel Darmon and I supervised. During the course of this internship, the theoretical developments concerning the scattering of acoustic and elastic waves by wedges with impedance boundary conditions were launched.
\item The \acrshort{utd} model was developed by Audrey Kamta-Djakou \cite{AKDthese} using the \acrfull{si} pole propagation algorithm. It should be adapted to the \acrshort{sf} method so it can be applied to 3D incidences. I have begun working on the integration of the 3D \acrshort{sf} codes along with a \acrshort{utd} model in the \acrshort{ndt} simulation platform CIVA. In order to deal with the finite extension of wedge edges in CIVA, a possibility is to use an incremental model such as the Huygens method or the \acrfull{itd}, which I have helped develop and validate in elastodynamics \cite{articleITD}.
\item The elastodynamic diffraction coefficients present a slight discontinuity at critical angles of reflection (angle linked to Snell's law of reflection on each wedge face), due to the presence of head waves. In the continuity of Fradkin et al. \cite{FradkinDarmon} and of Darmon \cite{HDRMichel}, further investigations must be made in order to model the contribution of these waves correctly.
\item In the final chapter, it was shown that for transversal incident waves with a skew angle higher than the critical angle for edge diffraction, an evanescent longitudinal wave is produced. The contribution of this wave could be evaluated or modeled.
\item Following the ideas of Kamotskii \cite{Kamotski2}, the Spectral Functions method could also be adapted to treat scattering by adjacent wedges for which other methods \cite{Bernardpoly,BorovikovKinber} could also be studied.
\end{itemize}