\chapter*{Conclusion}

The aim of this thesis is to propose and validate a generic and reliable elastodynamic wedge-diffraction model, valid for all wedge angles and for 3D incidences. This is done by extending a method called the \acrfull{sf} method and proposing the corresponding numerical resolution schemes. The principal results of this thesis are summarized in the following.

In the first chapter of this manuscript, a review of high-frequency wedge diffraction models is done. First, the two main non-uniform asymptotic methods are described : \acrfull{ge}, which only model reflected and refracted rays and the \acrfull{gtd}, which accounts for diffraction but diverges at observation directions close to specular reflections. Then, some uniform corrections of these models are presented. The \acrfull{ka}, which produces a uniform scattered field but models diffraction inaccurately, the \acrfull{ptd} which provides a good description of the scattered field in all directions but is computationally expensive, the \acrfull{uat} which also provides a good description of the scattered field but is difficult to implement and finally the \acrfull{utd} which is accurate, simple to implement and computationally cheap. For these reasons, \acrshort{utd} is the preferred uniform asymptotic model for wedge diffraction. Its accuracy relies on the existence of a reliable \acrshort{gtd} wedge diffraction model. With that in mind, the two main existing wedge diffraction models, the \acrfull{lt} method and the \acrfull{si} method, are presented briefly. Neither of them has been developed for an elastic wave incident on a wedge of angle higher than $\pi$.

In the second chapter of this manuscript, the spectral functions method is developed as a first step in the simpler case of an acoustic wave scattered by a soft (Dirichlet boundary conditions) or hard (Neumann boundary conditions) wedge of arbitrary angle. To do so, an integral formulation of the solution to the scattering problem is derived. This formulation is given with respect to two unknown functions called the spectral functions. A far-field asymptotic evaluation of this integral formulation leads to an expression of the \acrshort{gtd} diffraction coefficient as a function of the spectral functions. The integral formulation is then injected into the problem's boundary conditions, yielding an integral system of functional equations of which the spectral functions are the solution. This system is then solved semi-analytically. This means that the spectral functions are decomposed as the sum of two terms : a singular function, which is determined analytically thanks to a recursive algorithm, and a regular function, which is approached numerically thanks to a Galerkin collocation method. Finally, the accuracy of the numerical approximation of the regular part is improved using a technique called the "propagation of the solution". The method is validated by comparing the \acrshort{gtd} diffraction coefficients obtained using the semi-analytical spectral functions method to the \acrshort{gtd} diffraction coefficients derived from the exact solution given by Sommerfeld.

In the third chapter of the manuscript, the spectral functions method is applied to the more complex problem of elastic wave diffraction by a stress-free wedge of arbitrary angle. The main steps of the method are the same as in the previous chapter but the corresponding computations are more complex, since the spectral functions are now two-dimensional vectors and the incident, reflected and edge-diffracted waves can be polarized longitudinally and transversally. These two propagation modes are coupled by the wedge boundary conditions, meaning that mode conversion occurs. For each given configuration, two diffraction coefficients are therefore computed : one for longitudinal diffracted waves and one for transversal diffracted waves. The code developed using the \acrfull{sf} method is validated for wedge angles lower than $\pi$ by comparison to the \acrfull{lt} code. However, the existing \acrshort{lt} code is only valid for wedge angles lower than $\pi$. For wedge angles higher than $\pi$, the \acrshort{sf} code is validated by comparison to a finite elements code. Finally, the results are also validated experimentally using the same measurements that were made to validate the \acrshort{lt} code.

In the fourth and final chapter of the manuscript, the spectral functions method is applied to the 3D case of elastic wave diffraction by a stress-free wedge, where the incident wave vector is not necessarily in the the plane normal to the wedge edge. In this case, the incident ray on the wedge edge produces a cone of diffracted rays called Keller's cone of diffraction. The angle of this cone is determined by Snell's law of diffraction. According to Snell's law of diffraction, when the incident wave is transversal and the incident skew angle (i.e. the angle between the incident wave vector and the plane normal to the wedge edge) is higher than a certain angle called the critical angle, there is no diffracted longitudinal wave. The spectral functions then have imaginary branch points and extra care must be taken in dealing with these. The spectral functions method is developed in detail for the 3D case, for all types of incidences and for wedge angles higher and lower than $\pi$. An additional numerical approximation is proposed in order to compute the regular part of the spectral functions in the case of a transversal incident wave with a skew angle higher than the critical angle. The corresponding code is tested successfully in the particular cases of 2D incidences (the skew angle is set to $0$), of the "acoustic limit" (the longitudinal and transversal wave velocities are set to mimic acoustic wave propagation) and of an infinite plane (the wedge angle is equal to $\pi$ and there is no diffracted wave).

The results obtained during this thesis led to two publications in peer-reviewed journals \cite{article, articleelasto} as well to two communications in international conferences with conference proceedings \cite{DD2018,AFPAC}.

Some suggestions for future work are given below :
\begin{itemize}
\item In the final chapter of this thesis, the regular part of the spectral functions diverges in certain cases. Further investigations need to be made in order to find the cause of these divergences and a new method of computation must be proposed.
\item A thorough numerical and/or experimental validation of the code implemented to treat the case of 3D diffraction of an elastic wave must be conducted.
\item The spectral functions method could be further extended to treat dihedral interfaces between two elastic materials. This would be the continuity of Lucien Rochery's internship.
\item The \acrshort{utd} model was developed by Audrey Kamta-Djakou \cite{AKDthese} using the \acrfull{si} pole propagation algorithm. It should be adapted to the \acrshort{sf} method so it can be applied to 3D incidences.
\item The elastodynamic diffraction coefficients present a slight discontinuity at critical angles, due to the presence of head waves. In the continuity of Fradkin et al. \cite{FradkinDarmon}, further investigations must be made in order to model the contribution of these waves correctly.
\item In the final chapter, it was shown that for transversal incident waves with a skew angle higher than the critical angle, an evanescent longitudinal wave is produced. The asymptotic contribution of this wave should be determined.
\item Following the ideas of Kamotskii \cite{Kamotski2}, the Spectral Functions method could also be adapted to treat scattering by multiple adjacent wedges.
\end{itemize}