\chapter*{Introduction}
Inspection of industrial components, from the production phase to their use is essential to their reliability. Indentification and proper quantification of defects is necessary in order to avoid adverse events for end-users and is especially crucial in areas such as aeronautics, civil engineering, nuclear energy or the automobile industry. 

\acrfull{ndt}, also known as \acrfull{nde}, regroups all the inspection techniques that preserve the inspected specimen's integrity. There exists a wide range of \acrshort{ndt} methods. This thesis focuses on ultrasonic testing, an approach in which ultrasounds are emitted into a specimen and the waves scattered inside the specimen are analyzed in order to detect anomalies. These waves, which propagate through solid mediums without causing structural damage or changes, are elastic waves. The signal collected by the receiving transducer, which corresponds to the wave scattered by the specimen's boundaries and inhomogenities, contains information on the component's state and must therefore be analyzed.

The feasibility of ultrasonic inspections is predicted thanks to numerical modeling. Numerical modeling also helps with the analysis of the received signals. This is why CEA-LIST (Commissariat à l’Énergie Atomique et aux Énergies nouvelles - Laboratoire d’Intégration des Systèmes et Technologies) offers an \acrshort{ndt} simulation tool via the CIVA software platform. This software uses semi-analytical models to reduce computation time, which can be of limited validity. CIVA is therefore in constant evolution to extend its scope.
%%



