\chapter*{Introduction}

The term \acrfull{ndt}, also known as \acrfull{nde}, regroups all the inspection techniques that preserve the inspected specimen's integrity. This is particularly useful in areas such as civil engineering, aeronautics, nuclear energy or the automobile and railway industries, as it enables users to test industrial components during their production phase or during the course of their lifetime. There exists a wide range of \acrshort{ndt} methods, the most frequently used being visual testing, eddy-current, magnetic-particle, liquid penetrant, radiographic and finally ultrasonic testing. 

This thesis focuses on ultrasonic testing, an approach in which ultrasounds are emitted into a specimen and the waves scattered inside the specimen are analyzed in order to detect anomalies. These waves, which propagate through solid mediums without causing structural damage or changes, are elastic waves. The signal collected by the receiving transducer, which corresponds to the wave scattered by the specimen's boundaries and inhomogenities, contains information about the component's state and must therefore be analyzed.

The feasibility of ultrasonic inspections is predicted using simulation and modeling. Simulation and modeling also helps with the analysis of the received signals. To this end, CEA-LIST (Commissariat à l’Énergie Atomique et aux Énergies Alternatives - Laboratoire d’Intégration des Systèmes et Technologies) offers an \acrshort{ndt} simulation tool via the CIVA software platform. UT inspections deal with high frequency ($f \approx 2-5$ MHz) ultrasonic waves therefore, simulation of realistic inspections by finite elements and finite differences can be time consuming because such methods require a small mesh step for a precise description of the scattered wave (on the order of $1/5^{th}$ of the wavelength). Semi-analytical methods are consequently preferred for high frequency problems, in order to reduce computation time. 

Ultrasonic inspection of a specimen generates echoes from the entry and back-wall surfaces of this specimen. If these surfaces contain dihedral corners, it is then necessary to provide a correct model of the interaction between the ultrasonic beam and these wedges. These interactions may be linked to two different phenomena : reflection from the wedge faces and diffraction of the incident rays by the wedge edge. Both must be correctly taken into account by the model.

During the course of a previous thesis (Audrey Kamta-Djakou's PhD thesis \cite{AKDthese}), the specular model, which models reflection but not diffraction and therefore is not spatially continuous, was combined to an edge-diffraction model. The resulting model is called the \acrfull{utd} and provides a spatially uniform high frequency representation of the scattered field. This model presents the advantages of being computationally cheap (as opposed to finite elements or to the \acrfull{ka} which is currently implemented in CIVA) and of modeling diffraction correctly (as opposed to \acrshort{ka}). In order for the \acrshort{utd} model to be valid, a preexisting trustworthy model is necessary to accurately predict diffraction from a wedge edge. A method called the \acrfull{lt} method \cite{GautesenFradkin} was also studied in Kamta-Djakou's thesis but has the disadvantage of only being developed for wedge angles lower than $\pi$.

The aim of this thesis is to propose and validate a generic and reliable elastodynamic wedge-diffraction model, valid for all wedge angles and for 3D incidences. So far, this has not yet been done in elastodynamics. This is done by extending a method called the \acrfull{sf} method and proposing the corresponding numerical resolution schemes. 

This manuscript is divided into 4 chapters.

In chapter 1, a review of high frequency wedge scattering models is presented. The first section of this chapter describes non-uniform asymptotic methods (non-uniform in the sense that the resulting scattered field is not spatially continuous), namely \acrfull{ge}, which models specular reflection but not diffraction and is therefore discontinuous at shadow boundaries and the \acrfull{gtd}, which models reflection and diffraction but diverges in directions of specular reflections and at shadow boundaries. The second section presents uniform solutions based on these non-uniform models : the \acrfull{ka}, the \acrfull{ptd}, the \acrfull{uat} and finally the \acrfull{utd}. All these models (except for \acrshort{ka}) require a reliable preexisting wedge-diffraction \acrshort{gtd} solution in order to be accurate. The third and final section therefore presents the two main existing \acrshort{gtd} wedge diffraction models : the \acrfull{si} method and the \acrfull{lt} method.

In chapter 2, the \acrfull{sf} method is developed as a first step for the simpler case of an acoustic wave scattered by a soft (Dirichlet boundary conditions) or hard (Neumann boundary conditions) wedge. This is done by first, determining an integral formulation of the scattering problem. This formulation is given with respect to two unknown functions called the spectral functions. A system of functional equations of which these spectral functions are the solution is then determined using the problem's boundary conditions. This system permits the decomposition of the spectral functions into two terms : a singular function, determined analytically using a recursive algorithm and a regular function which is approached numerically using a Galerkin collocation method. The accuracy of this numerical approximation is improved by a method called "propagation of the solution". Finally, the solution computed using the \acrlong{sf} method is validated by comparison to the exact solution given by Sommerfeld.

Chapter 3 deals with extension of the \acrlong{sf} method to the 2D case of an elastic wave scattered by a stress-free wedge. The outline of the method is similar, but the unknown spectral functions are now two-dimensional vectors and the incident, reflected and diffracted waves can be longitudinal or transversal and mode conversion can occur. All of these phenomena are accounted for by the \acrlong{sf} method, which has the advantage of being valid for all wedge angles (as opposed to the previously existing \acrfull{lt} and \acrfull{si} methods). The resulting code is validated for wedge angles lower than $\pi$ by comparison to the \acrshort{lt} code and for wedge angles higher than $\pi$ by comparison to a finite elements code. In addition, experimental validation is also carried out, thanks to previously made experimental measurements.

Finally, in chapter 4, the \acrlong{sf} method is developed for 3D cases, meaning for cases where the incident wave is no longer in the plane normal to the wedge edge. The spectral functions are now three-dimensional vectors and the possible wave polarizations are longitudinal, transverse horizontal and transverse vertical. An additional difficulty is created in the case of an incident transversal wave, when the skew angle (the angle between the incident ray and the plane normal to the edge) is higher than a certain angle called the critical angle (which depends on the propagation medium), extra care must be taken to deal with the branch points (for a multi-valued function, a branch point is a point where different definitions coincide) of the spectral functions, as some of them now lie on the imaginary axis (whereas up till now, the branch points were all real numbers) and can therefore be crossed by the complex contour deformations encountered in the method. Nevertheless, the method is developed in detail and for all wedge angles and tested numerically in various ways. In cases where imaginary branch points appear, an additional numerical approximation method is proposed.