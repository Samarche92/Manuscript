\chapter[][3D Elastic Case]{The spectral functions method for 3D elastic wave diffraction by a stress-free wedge}
\label{chap-3D}

\section*{Introduction}

\section{Problem statement}

\begin{figure}[h]
\centering
\begin{subfigure}[b]{0.65\textwidth}
	\includegraphics[width=\textwidth]{diedre3D.png}
	\caption{Dièdre d'angle $\varphi$ et l'onde incidente $\textbf{k}^{inc}$}
	\label{diedre}
\end{subfigure}
\hfill
\begin{subfigure}[b]{0.3\textwidth}
	\includegraphics[width=\textwidth]{Spherical_Coordinates.png}
	\caption{Système de coordonnées utilisé}
	\label{coords}
\end{subfigure}
\caption{Géométrie du problème}
\label{diedre_coords}
\end{figure}

On s'intéresse à la diffraction d'une onde élastique par un dièdre délimité par les faces $\mathcal{S}_1$ et $\mathcal{S}_2$ libres de contraintes. La géométrie du problème est donnée sur le schéma de la figure \ref{diedre_coords}. On définit alors l'intérieur du dièdre 

$$\Omega=\{ (r\cos \theta \cos \delta, r \sin \theta \cos \delta, r \sin \delta)\, \backslash \, \theta \in \rbrack 0, \varphi \lbrack, \, \delta \in \rbrack -\frac{\pi}{2}, \frac{\pi}{2} \lbrack \} $$
et sa surface
$$ \mathcal{S}=\mathcal{S}_1 \cup \mathcal{S}_2 $$

On suppose que le dièdre est irradié par une onde plane incidente de la forme
$$ \mathbf{u}^{inc}(\mathbf{x},t)=\mathbf{A}_{\alpha}e^{i(\mathbf{k}_{\alpha}^{inc}\cdot \mathbf{x}-\omega t)}$$
Où $\mathbf{A}_{\alpha}$ est le vecteur d'amplitude de l'onde incidente et $\mathbf{k}_{\alpha}^{inc}$ est le vecteur d'onde incident. On note $\alpha=L,TH,TV$ le type de l'onde incidente (L pour une onde longitudinale, TH pour une onde tranverse horizontale ou TV pour une onde transverse verticale). Dans le repère $(\mathbf{x}_1,\mathbf{y}_1, \mathbf{z}_1)$, le vecteur d'onde incident est donné par :
$$\mathbf{k}_{\alpha}^{inc}=\frac{\omega}{c_{\alpha}} \begin{pmatrix}
\cos\theta_{inc} \cos \delta_{inc} \\ \sin\theta_{inc} \cos \delta_{inc} \\
\sin \delta_{inc}
\end{pmatrix} $$
La grandeur $c_L=\sqrt{(\underline{\lambda}+2\underline{\mu})/\rho}$ est la célérité des ondes longitudinales (c'est-à-dire les ondes dont l'amplitude est parallèle à la direction de propagation donnée par le vecteur d'onde) dans le matériau considéré st $c_T=\sqrt{\underline{\mu}/\rho}$ celle des ondes transversales (dont l'amplitude dans le plan normal à la direction de propagation) avec $\underline{\lambda}, \underline{\mu}$ les coefficients de Lamé. 
\paragraph{}
Le vecteur d'amplitude peut être dirigé suivant trois directions différentes et deux à deux orthogonales, selon la polarisation de l'onde incidente. Celle-ci peut-être longitudinale (notée L), transverse verticale (notée TV) ou transverse horizontale (notée TH). On note alors $\hat{i}_*$ avec $*=L, TH, TV$ le vecteur unitaire correspondant à chacune de ces polarisations. Il est donné par Achenbach \cite{Achenbach}, formule (1.39) :
\begin{equation}
\hat{i}_L = \begin{pmatrix}
\cos\theta_{inc} \cos \delta_{inc} \\ \sin\theta_{inc} \cos \delta_{inc} \\
\sin \delta_{inc}
\end{pmatrix}
\hfill
\hfill
\hat{i}_{TV} = \begin{pmatrix}
-\cos\theta_{inc} \sin \delta_{inc} \\ -\sin\theta_{inc} \sin \delta_{inc} \\
\cos \delta_{inc}
\end{pmatrix}
\hfill
\hfill
\hat{i}_{TH} = \begin{pmatrix}
-\sin\theta_{inc} \\ \cos\theta_{inc} \\
0
\end{pmatrix}
\label{ivec}
\end{equation}
On a alors
$$ \mathbf{A}_L \in \mbox{Vect}(\hat{i}_L) $$
$$ \mathbf{A}_T \in \mbox{Vect}(\hat{i}_{TV}, \hat{i}_{TH})$$
La notation Vect désigne l'espace engendré par un vecteur ou une famille de vecteurs.
\paragraph{}
%Dans toute la suite, on suppose que le repère utilisé est le repère $(\mathbf{x}_1,\mathbf{y}_1, \mathbf{z}_1)$, sauf mention explicite du contraire.
 Rappellons l'équation de l'élasticité linéaire pour un matériau homogène isotrope dont est solution le champ de déplacement $\mathbf{u}$ :
\begin{equation}
\underline{\mu} \Delta \mathbf{u} + (\underline{\lambda}+\underline{\mu})\nabla \nabla \mathbf{u} = \rho \frac{\partial^2 \mathbf{u}}{\partial t^2}
\label{Elasticitelin}
\end{equation}
Sur les faces du dièdre, ce champ vérifie la condition de surface libre (absence de contraintes normales) qui s'écrit :
\begin{equation}
(\underline{\lambda} \nabla \mathbf{u} .\mathbf{\mathbb{I}_3}+2\underline{\mu} \mathbf{\varepsilon} (\mathbf{u})).\mathbf{n}=0
\end{equation}
On précise que $\mathbf{\mathbb{I}_3}$ est la matrice identité d'ordre 3, $n$ est la normale intérieure au dièdre (soit $n=\mathbf{y}_1$ sur $\mathcal{S}_1$  et $n=\mathbf{y}_2$ sur $\mathcal{S}_2$) et $\underline{\lambda}, \underline{\mu}$ sont les coefficients de Lamé, propres au matériau considéré. On rappelle également l'expression du tenseur des déformations :
$$ \mathbf{\varepsilon}_{ij}(\mathbf{u})=\frac{1}{2} (\frac{\partial u_i}{\partial x_j}+\frac{\partial u_j}{\partial x_i}) $$
%\paragraph{}
%Kamotski et Lebeau \cite{Lebeau2} ont démontré l'existence et l'unicité d'une solution à ce problème dans le cas bidimensionnel. Cette démonstration reste valable dans le cas présent. Nous allons détailler ici une procédure de calcul de cette solution.
\paragraph{}
 Désormais, les caractères gras seront réservés aux matrices afin de simplifier les notations. Les solutions cherchées sont harmoniques en temps, on ommettra dans la suite le facteur $e^{-i\omega t}$ commun à tous les termes de la solution. De plus, il n'y a aucun obstacle à la propagation dans la direction z, le facteur $e^{i\frac{\omega}{c_{\alpha}}\sin\delta_{inc}z}$ est donc lui aussi commun à tous les termes intervenant  dans l'écriture de la solution. 
% On rappelle la loi de Snell :
% \begin{equation}
% \frac{1}{c_{\alpha}}\sin\delta_{\alpha}=\frac{1}{c_{\beta}}\sin\delta_{\beta}
% \label{Snell}
% \end{equation}
\paragraph{}
On suppose que le champ total u est la somme du champ incident $u^{inc}$ et d'un champ diffracté $u_0$.
\begin{equation}
u=u_0+u^{inc}
\end{equation}
On adimensionne le problème en effectuant le changement de variables :
\begin{equation}
u_0(x,y,z)=v\left( \frac{\omega}{c_L} x, \frac{\omega}{c_L} y \right)e^{i\nu_{\alpha}\sin\delta_{\alpha}z}
\label{adiming}
\end{equation}
Les coefficients de Lamé adimensionnés sont donnés par :
$$ \lambda=\frac{\underline{\lambda}}{\rho c_L^2}, \; \; \; \; \mu=\frac{\underline{\mu}}{\rho c_L^2}$$
On définit également les paramètres sans dimension suivants :
\begin{equation}
\nu_L=1 \hspace{2em} \nu_T=\frac{c_L}{c_T} 
\end{equation}
Pour simplifier les notations, dans toute la suite on notera :
\begin{equation}
\tau=\nu_{\alpha}\sin\delta_{\alpha}
\label{deftau}
\end{equation}
Il est important de noter que l'on a $\tau \in \lbrack -\nu_{\alpha}, \nu_{\alpha} \rbrack$. La dépendance en z de $u_0$ est donc contenue dans le facteur $e^{i\tau z}$ qui sera omis dans toute la suite.

Le problème adimensionné que l'on cherche à résoudre s'écrit finalement:

\begin{eqnarray}
(\mathcal{P}^*) \hspace{2em} \left\{
\begin{array}{lr}
(E+1)v=0 & (\Omega) \\
Bv=-Bv_{\alpha}^{inc} & (\mathcal{S})
\end{array}
\right.
\label{Padim}
\end{eqnarray}
Avec,en notant $(v_1,v_2,v_3)$ les coordonnées de v :
\begin{equation}
Ev=\mu (\Delta v -\tau^2 v)+(\lambda+\mu)
\begin{pmatrix}
\frac{\partial^2 v_1}{\partial x^2}+\frac{\partial^2 v_2}{\partial x \partial y} + i\tau\frac{\partial v_3}{\partial x} \\
\frac{\partial^2 v_1}{\partial x \partial y}+\frac{\partial^2 v_2}{\partial y^2}+ i\tau\frac{\partial v_3}{\partial y}\\
i\tau\left( \frac{\partial v_1}{\partial x}+\frac{\partial v_2}{\partial y}\right)-\tau^2 v_3
\end{pmatrix}
\label{Eadim}
\end{equation}
et
\begin{equation}
Bv=\begin{pmatrix}
\mu \left(\frac{\partial v_x}{\partial y}+\frac{\partial v_y}{\partial x}\right) \\
\frac{\partial v_y}{\partial y}+\lambda\left( \frac{\partial v_x}{\partial x}+i\tau v_z\right)\\
\mu \left(\frac{\partial v_z}{\partial y}+i\tau v_y\right)
\end{pmatrix}
\label{Badim}
\end{equation}
Où on note $E$ et $B$  l'opérateur d'élasticité adimmensionné et de contraintes normales respectivement. La première équation du système correspond donc à la version adimensionnée des équations de l'élasticité linéaire vérifiées et la seconde correspond à la condition de surface libre adimmensionnée.