\chapter[][3D Elastic Case]{3D elastic wave diffraction by a stress-free wedge of arbitrary angle}
\label{chap-3D}

\section*{Introduction}

In the previous chapters of this manuscript, the spectral functions method has been presented in the case of an acoustic wave incident on an infinite wedge with Dirichlet or Neumann boundaries and in the case of an elastic wave incident on a stress-free wedge. Both of these problems have been treated in the case of 2D incidences, meaning that the incident ray is in the plane normal to the wedge edge. In this chapter, we will extend the spectral functions method to the case of 3D incidences. This extension will be done for elastic waves incident on stress-free wedges and it will be shown that the 3D code developed for the elastic case can be applied to the limit case of an acoustic wave incident on a wedge with Dirichlet boundary conditions, as well as to the 2D elastic problem.

The problem of 3D wedge diffraction has been studied over the past century in acoustics, electromagnetics and in to a lesser degree in elastodynamics. The problem was introduced notably by Sommerfeld \cite{Sommerfeld}, who gave an exact expression of the solution to the scattering problem of a scalar plane wave by a wedge with Dirichlet or Neumann boundaries in the form of a contour integral. This integral can be used to obtain an analytical expression of the \acrfull{gtd} diffraction coefficient both in electromagnetics and in acoustics \cite{Bouche,Bo}. Independently, Macdonald \cite{Macdo} has expressed the solution to the same problem as an infinite series. Proof that the Sommerfeld and the Macdonald approaches are equivalent was developed by Carslaw \cite{Carslaw}.

In the case of an incident acoustic wave, Rawlins \cite{Rawlins} determined an expression of the solution as a real integral for a spherical acoustic wave diffracted by a wedge with Dirichlet or Neumann boundaries when the aperture angle is an integer multiple of $\frac{\pi}{n}$, where $n$ is also an integer. In the case of an electromagnetic wave, Rojas \cite{Rojas} derived a uniform asymptotic solution for a plane wave incident on an impedant wedge when the wedge angle is a multiple of $\frac{\pi}{2}$. By generalizing the Malyuzhinets technique \cite{SMtechnique}, Bernard \cite{Bernard} reduced the 3D problem of a plane electromagnetic wave diffracted by an impedant wedge of arbitrary angle to a scalar functional equation with only one unknown and provides examples of numerical resolution of this equation in the case where the relative impedance of the wedge faces is $1$ and in the case of a half or full plane with arbitrary impedance. Finally, an application of the Wiener-Hopf technique to the case of electromagnetic plane wave diffraction by impenetrable wedges of arbitrary angles was developed by Daniele in 2D \cite{Daniele} and extended to 3D cases by Daniele and Lombardi \cite{DanieleLombardi}.

In elastodynamics, a \acrshort{gtd} solution to the 3D problem of plane wave diffraction by a stress-free half plane was developed by Achenbach and Gautesen \cite{Achenbach,AchenbachGautesen,GautesenNote} and Gautesen \cite{GautesenRayleigh4,GautesenRayleigh3} proposed a semi-analytical scheme of resolution of the far-field scattering problem of a skew incident Rayleigh wave diffracted by a quarter-space (i.e. a wedge of angle $\frac{\pi}{2}$ or $\frac{3\pi}{2}$). To our knowledge, no resolution scheme has been developed for a skew incident longitudinal or transversal plane elastic wave diffracted by an arbitrary-angled wedge. Therefore, it is the aim of this chapter.

In the first part of this chapter, the problem is presented. In the second part, an integral formulation of the solution is derived, depending on two unknown functions, called the spectral functions. The 3D diffraction coefficient is defined and expressed with respect to these spectral functions. In the third part, the semi-analytical evaluation of these functions is detailed. Finally, the corresponding code is tested numerically in the fourth part.
\section{Problem statement}

\begin{figure}[h]
\centering
	\includegraphics[width=\textwidth]{images/chapter4/wedge_3D.png}
\caption{Geometry of the problem}
\label{diedre_coords}
\end{figure}

Let us consider the problem of an elastic wave diffracted by a stress-free infinite wedge delimited by faces $\mathcal{S}_1$ and $\mathcal{S}_2$. The geometry of the problem is shown on Fig.~\ref{diedre_coords}. Vector $\mathbf{x}=(r,\theta,\delta)$ is an observation point in the propagation domain indexed by its spherical coordinates and the domain $\Omega$ is the inside of the wedge, defined by :
\begin{equation}
\Omega=\{ (r\cos \theta \cos \delta, r \sin \theta \cos \delta, r \sin \delta)\, \backslash \, \theta \in \rbrack 0, \varphi \lbrack, \, \delta \in \rbrack -\frac{\pi}{2}, \frac{\pi}{2} \lbrack \}
\end{equation}
%et sa surface
%$$ \mathcal{S}=\mathcal{S}_1 \cup \mathcal{S}_2 $$

The incident wave is a plane wave of the form
\begin{equation}
\mathbf{u}^{inc}(\mathbf{x},t)=\mathbf{A}_{\alpha}e^{i(\mathbf{k}_{\alpha}^{inc}\cdot \mathbf{x}-\omega t)}
\end{equation}
where $\mathbf{A}_{\alpha}$ is the amplitude vector of the incident wave and $\mathbf{k}_{\alpha}^{inc}$ is the incident wave vector. The type of the incident wave is denoted $\mathbf{k}_{\alpha}^{inc}$ (L for a longitudinal wave, TH for transverse horizontal and TV for transverse vertical). $(O; \mathbf{e}_{x_1}, \mathbf{e}_{y_1}, \mathbf{e}_{z_1})$ is a Cartesian coordinate system associated to face $\mathcal{S}_1$ and $(O; \mathbf{e}_{x_2}, \mathbf{e}_{y_2}, \mathbf{e}_{z_2})$ is a Cartesian coordinate system associated to face $\mathcal{S}_2$. In the following, all vectors are expressed in the coordinate system $\mathbf{x}=(x'_1,y'_1,z'_1)_{(\mathbf{e}_{x_1}, \mathbf{e}_{y_1}, \mathbf{e}_{z_1})}$, unless explicitly mentioned otherwise. In this system, the incident wave vector is given by :
\begin{equation}
\mathbf{k}_{\alpha}^{inc}=\frac{\omega}{c_{\alpha}} \begin{pmatrix}
\cos\theta_{inc} \cos \delta_{inc} \\ \sin\theta_{inc} \cos \delta_{inc} \\
\sin \delta_{inc}
\end{pmatrix}
\end{equation}
As always, $c_L$ is the velocity of longitudinal waves and $c_T$ is the velocity of transverse waves.

The amplitude vector can be directed by three different and two-by-two orthogonal vectors, depending on the incident wave's polarization. These unit polarization vectors are noted $\hat{i}_*$, where $*=L, TH, TV$ and are given by Achenbach \cite{Achenbach} :
\begin{equation}
\hat{i}_L = \begin{pmatrix}
\cos\theta_{inc} \cos \delta_{inc} \\ \sin\theta_{inc} \cos \delta_{inc} \\
\pm \sin \delta_{inc}
\end{pmatrix}
\hfill
\hfill
\hat{i}_{TV} = \begin{pmatrix}
\mp\cos\theta_{inc} \sin \delta_{inc} \\ \mp\sin\theta_{inc} \sin \delta_{inc} \\
\cos \delta_{inc}
\end{pmatrix}
\hfill
\hfill
\hat{i}_{TH} = \begin{pmatrix}
-\sin\theta_{inc} \\ \cos\theta_{inc} \\
0
\end{pmatrix}
\label{C4:ivec}
\end{equation}
where the top sign gives the polarization of an incident wave and the bottom sign gives the polarization of a diffracted wave.

For a homogeneous, isotropic material, the linear elasticity equation solved by the displacement field $\mathbf{u}$ is 
\begin{equation}
\underline{\mu} \Delta \mathbf{u} + (\underline{\lambda}+\underline{\mu})\nabla \nabla \mathbf{u} = \rho \frac{\partial^2 \mathbf{u}}{\partial t^2}
\label{C4:Elasticitelin}
\end{equation}
On each of the wedge faces, the displacement field verifies the zero-stress boundary conditions, expressed as :
\begin{equation}
(\underline{\lambda} \nabla \mathbf{u} .\mathbf{\mathbb{I}_3}+2\underline{\mu} \mathbf{\varepsilon} (\mathbf{u})).\mathbf{n}=0
\label{C4:stressfree}
\end{equation}
where $\mathbf{\mathbb{I}_3}$ is the identity matrix of the third order, $n$ is the inward facing normal to the wedge face ($n=\mathbf{y}_1$ on $\mathcal{S}_1$  and $n=\mathbf{y}_2$ on $\mathcal{S}_2$) and $\underline{\lambda}, \underline{\mu}$ are the Lamé coefficients of the considered elastic medium. The expression of the deformations tensor is :
\begin{equation}
\mathbf{\varepsilon}(\mathbf{u})=\frac{1}{2} \begin{pmatrix}
2\dfrac{\partial u_1}{\partial x'_1} & \dfrac{\partial u_1}{\partial y'_1}+\dfrac{\partial u_2}{\partial x'_1}&\dfrac{\partial u_1}{\partial z'_1}+\dfrac{\partial u_3}{\partial x'_1} \\
\dfrac{\partial u_1}{\partial y'_1}+\dfrac{\partial u_2}{\partial x'_1}&2\dfrac{\partial u_2}{\partial y'_1}&\dfrac{\partial u_2}{\partial z'_1}+\dfrac{\partial u_3}{\partial y'_1} \\
\dfrac{\partial u_1}{\partial z'_1}+\dfrac{\partial u_3}{\partial x'_1} &\dfrac{\partial u_2}{\partial z'_1}+\dfrac{\partial u_3}{\partial y'_1} & 2\dfrac{\partial u_3}{\partial z'_1}
\end{pmatrix}
\label{C4:mateps}
\end{equation}
Kamotski and Lebeau \cite{KamotskiLebeau} have proven existence and uniqueness of the solution to this problem in the 2D the case. We will suppose that their demonstration is still valid in the 3D case.

From hereon after, bold characters will be reserved to matrices in order to simplify notations. The solutions being time harmonic, the factor $e^{-i\omega t}$ will be implied but omitted everywhere. Furthermore, since there is no obstacle to propagation in the $z$ direction, $e^{i\frac{\omega}{c_{\alpha}}\sin\delta_{inc}z^{\prime}_1}$ is also a common factor to all the terms which appear in the solution.

The total field is written as the sum of an incident field $u^{inc}$ and a scattered field $u_0$
\begin{equation}
u=u_0+u^{inc}
\label{C4:scat}
\end{equation}

The dimensionless problem is obtained by applying the following variable change :
\begin{subequations}
\begin{equation}
x=\dfrac{\omega}{c_L}x', \hspace{1em} y=\dfrac{\omega}{c_L}y', \hspace{1em} z=\dfrac{\omega}{c_L}z'
\end{equation}
\begin{equation}
u_0(x',y',z')=v(x,y)e^{i\nu_{\beta}\sin\delta_{\beta}z}
\end{equation}
\label{C4:adiming}
\end{subequations}
where $\delta_{\beta}$ is the angle of Snell's cone of diffraction (determined by Snell's law of diffraction, given by \eqref{C4:Snelldiff}), the dimensionless Lamé parameters $\lambda,\mu$ are given in the previous chapter by \eqref{LameAdim} and parameters $\nu_L$ and $\nu_T$ are also defined in the previous chapter, by \eqref{nuLnuT}. Since $e^{i\nu_{\alpha}\sin\delta_{inc}z}$ is a common factor to all the terms of the solution, we can deduce Snell's law of diffraction :
\begin{equation}
\nu_{\alpha}\sin\delta_{inc}=-\nu_{\beta}\sin\delta_{\beta}
\label{C4:Snelldiff}
\end{equation}
To simplify notations, the following parameter $\tau$ is defined by :
\begin{equation}
\tau=\nu_{\alpha}\sin\delta_{\alpha}
\label{deftau}
\end{equation}
Note that we therefore always have $\tau \in \lbrack -\nu_{\alpha}, \nu_{\alpha} \rbrack$. $u_0$'s $z$-dependency is entirely contained in the factor $e^{i\tau z}$ which will be implied but omitted in all the following.

Substituting \eqref{C4:scat} and \eqref{C4:adiming} into \eqref{C4:Elasticitelin} and \eqref{C4:stressfree} and using \eqref{C4:mateps} yields the dimensionless problem
\begin{eqnarray}
(\mathcal{P}^*) \hspace{2em} \left\{
\begin{array}{lr}
(E+1)v=0 & (\Omega) \\
Bv=-Bv_{\alpha}^{inc} & (\mathcal{S})
\end{array}
\right.
\label{C4:Padim}
\end{eqnarray}
where $(v_1,v_2,v_3)$ are the components of vector $v$ :
\begin{equation}
Ev=\mu (\Delta v -\tau^2 v)+(\lambda+\mu)
\begin{pmatrix}
\frac{\partial^2 v_1}{\partial x^2}+\frac{\partial^2 v_2}{\partial x \partial y} + i\tau\frac{\partial v_3}{\partial x} \\
\frac{\partial^2 v_1}{\partial x \partial y}+\frac{\partial^2 v_2}{\partial y^2}+ i\tau\frac{\partial v_3}{\partial y}\\
i\tau\left( \frac{\partial v_1}{\partial x}+\frac{\partial v_2}{\partial y}\right)-\tau^2 v_3
\end{pmatrix}
\label{C4:Eadim}
\end{equation}
and
\begin{equation}
Bv=\begin{pmatrix}
\mu \left(\frac{\partial v_x}{\partial y}+\frac{\partial v_y}{\partial x}\right) \\
\frac{\partial v_y}{\partial y}+\lambda\left( \frac{\partial v_x}{\partial x}+i\tau v_z\right)\\
\mu \left(\frac{\partial v_z}{\partial y}+i\tau v_y\right)
\end{pmatrix}
\label{C4:Badim}
\end{equation}
where $E$ and $B$ are respectively the dimensionless linear elasticity operator and normal stress operator and $\lambda,\mu$ are the dimensionless Lamé parameters, defined by \eqref{LameAdim}. The dimensionless incident field is
\begin{multline}
\rm v_L^{inc}(r,\theta)=\begin{pmatrix}
\cos \theta_{inc}\cos\delta_{inc} \\
\sin \theta_{inc}\cos\delta_{inc} \\
\sin\delta_{inc}
\end{pmatrix} e^{ir\nu_L \cos(\theta-\theta_{inc})\cos\delta_{inc}}
\\
\rm v_{TH}^{inc}(r,\theta)=\begin{pmatrix}
-\sin \theta_{inc} \\
\cos \theta_{inc} \\
0
\end{pmatrix} e^{ir\nu_T \cos(\theta-\theta_{inc})\cos\delta_{inc}} 
\\
\rm v_{TV}^{inc}(r,\theta)=\begin{pmatrix}
-\cos\theta_{inc}\sin\delta_{inc} \\
-\sin\theta_{inc}\sin\delta_{inc} \\
\cos\delta_{inc}
\end{pmatrix} e^{ir\nu_T \cos(\theta-\theta_{inc})\cos\delta_{inc}} 
\label{C4:vinc}
\end{multline}
 The first equation of system \eqref{C4:Padim} is the dimensionless version of the linear elasticity equation and the second equation is the dimensionless version of the stress-free boundary conditions.

\section{Integral formulation of the solution}
As for the previous cases, the first step in solving problem $(\mathcal{P}^{\alpha})$ is to formulate the solution as an integral. 
\subsection{Limiting absorption principle}
The limiting absorption principle is applied to $(\mathcal{P}^{\alpha})$. This means that it is considered as a special case $(\varepsilon=0)$ of the problem
\begin{eqnarray}
(\mathcal{P}^*_{\varepsilon}) \hspace{2em} \left\{
\begin{array}{lr}
(E+e^{-2i\epsilon})v^{\varepsilon}=0 & (\Omega) \\
Bv^{\varepsilon}=-Bv_*^{inc} & (\mathcal{S})
\end{array}
\right.
\label{C4:Pabs}
\end{eqnarray}
Following Kamotski and Lebeau \cite{KamotskiLebeau}, we will once again suppose that the solution can be expressed as the sum of two contributions, corresponding to each of the wedge faces :
\begin{equation}
v^{\varepsilon}=v_1^{\varepsilon}+v_2^{\varepsilon}
\label{C4:v1+v2}
\end{equation}
where functions $v_j^{\varepsilon}$ are now defined on all of  $\mathbb{R}^3$ by
\begin{equation}
v_j^{\varepsilon}=-(E+e^{-2i\varepsilon})^{-1} \begin{bmatrix}
\begin{pmatrix}
\alpha_j \\
\beta_j \\
\gamma_j
\end{pmatrix}
\otimes \delta_{\mathcal{S}_j}
\end{bmatrix}
\label{C4:vjdef}
\end{equation}
Distributions $\alpha_j,\beta_j, \gamma_j $ are unknown and are belong to the special class $\mathcal{A}$ defined in chapter 2 of this thesis, Def.~\ref{defClassA}. We can now define the outgoing solution of $(\mathcal{P}^{\alpha})$ analogously to the 2D case :
\begin{definition}
	 v is called an outgoing solution of equation \eqref{C4:Padim} if v is a solution of the form
	\begin{equation}
	\label{C4:decomposition}
	v=v_1|_{\Omega}+v_2|_{\Omega}
	\end{equation}
	where, for $j=1,2$ :
	\begin{equation}
	\label{C4:inv_potentiels}
	v_j=-\lim_{\varepsilon \to 0} (E+e^{-2i\varepsilon})^{-1} \begin{bmatrix}
\begin{pmatrix}
\alpha_j \\
\beta_j \\
\gamma_j
\end{pmatrix}
\otimes \delta_{\mathcal{S}_j}
\end{bmatrix}
	\end{equation}
	where $\alpha_j,\beta_j,\gamma_j \in \mathcal{A}$ and where $\delta_{\mathcal{S}_1}$ and $\delta_{\mathcal{S}_2}$ are the Dirac distributions associated to the wedge faces $\mathcal{S}_1$ and $\mathcal{S}_2$ respectively.
\end{definition}
The following theorem was proved by Kamotski and Lebeau \cite{KamotskiLebeau} in the 2D case. We will suppose that their proof can be adapted to the 3D case and that the theorem is still true.
\begin{theorem}
Equation \eqref{C4:Padim} admits a unique outgoing solution.
\end{theorem}
Now that the outgoing solution has been defined, we will derive an integral formulation of this solution.

\subsection{Integral formulation}
The two-sided Fourier transform and its inverse transform are defined in chapter 2 of this manuscript by \eqref{fullfourierdef}. The first step in determining an integral formulation of the solution is to apply the two-sided Fourier transform to \eqref{C4:vjdef}. This is possible because all the distributions that appear in this equation are tempered distributions and they therefore admit a Fourier transform. We then have :
\begin{equation}
\hat{v}^{\varepsilon}_j(\xi,\eta)=(\mathbf{M}-e^{-2i\varepsilon}\mathbf{\mathbb{I}_3})^{-1}\Sigma_j(\xi)
\label{C4:matMvjeps}
\end{equation}
where $\Sigma_j, j=1,2$  are the unknown spectral functions, defined by :
\begin{equation}
\Sigma_j(\xi,)=\begin{pmatrix}
\hat{\alpha_j}(\xi)\\ \hat{\beta_j}(\xi) \\ \hat{\gamma_j}(\xi)
\end{pmatrix}
\label{C4:sigmacomponents}
\end{equation}
and where \textbf{M} is the two-sided Fourier transform of operator $E$. Its expression is :
\begin{equation}
\mathbf{M}(\xi,\eta)=
\begin{pmatrix}
\xi^2+\mu(\eta^2+\tau^2) & (\lambda+\mu)\xi \eta & (\lambda+\mu)\xi\tau \\
 (\lambda+\mu)\xi \eta & \eta^2+\mu(\xi^2+\tau^2) & (\lambda+\mu)\eta\tau \\
(\lambda+\mu)\xi\tau & (\lambda+\mu)\eta\tau & \tau^2+\mu(\xi^2+\eta^2) 
\end{pmatrix}
\label{C4:matM}
\end{equation}
Substituting $\lambda$ by $1-2\mu$ and $\mu$ by $1/\nu_T^2$, \eqref{C4:matM} yields
\begin{multline}
(\mathbf{M}-e^{-2i\varepsilon}\mathbb{I}_3)^{-1}=\\
\frac{\begin{pmatrix}
\xi^2+\nu_T^2(\eta^2+\tau^2-e^{-2i\varepsilon}) & (1-\nu_T^2)\xi \eta & (1-\nu_T^2)\xi \tau \\
(1-\nu_T^2)\xi \eta & \eta^2+\nu_T^2(\xi^2+\tau^2-e^{-2i\varepsilon}) & (1-\nu_T^2)\tau \eta \\
(1-\nu_T^2)\xi \tau & (1-\nu_T^2)\tau \eta & \tau^2+\nu_T^2(\eta^2+\xi^2-e^{-2i\varepsilon})
\end{pmatrix}}{(\xi^2+\eta^2+\tau^2-e^{-2i\varepsilon})(\xi^2+\eta^2+\tau^2-\nu_T^2e^{-2i\varepsilon})} 
\end{multline}

Finally, the integral formulation of $v_j$ is obtained by inverting the two-sided Fourier transform applied in \eqref{C4:matMvjeps} :
\begin{equation}
v_j^{\varepsilon}(x_j,y_j)=\frac{1}{4\pi^2}\int_{\mathbb{R}^2} e^{i x_j\xi}\left( \int_{-\infty}^{+\infty}e^{iy_j\eta} (\mathbf{M}-e^{-2i\varepsilon}\mathbf{\mathbb{I}_3})^{-1} \,d\eta \right) \Sigma_j(\xi) \,d\xi
 \label{C4:invdouble}
\end{equation}

The poles of $(\mathbf{M}-e^{-2i\varepsilon}\mathbf{\mathbb{I}_3})^{-1}$ (which are the poles of the integrand of the inner integral on $\eta$ in \eqref{C4:invdouble}) are located in $\eta=\pm \zeta_*^{\varepsilon}(\xi)$, where $*=L,T$ and
\begin{equation}
\zeta_*^{\epsilon}(\xi)=\sqrt{e^{-2i\varepsilon}\nu^2_*-(\xi^2+\tau^2)}
\end{equation}
Let us define $\nuti_*, *=L,T$ by
\begin{equation}
\nuti_*^{\varepsilon}=\sqrt{e^{-2i\varepsilon}\nu_*^2-\tau^2}
\label{defnutieps}
\end{equation}
If the incident wave is longitudinal, then, according to \eqref{deftau}, $\tau=\sin\delta_{inc}$ and
\begin{subequations}
\begin{equation}
\nuti_L^0=\sqrt{1-\sin^2\delta_{inc}}=\cos\delta_{inc} \quad \in \mathbb{R}
\end{equation}
\begin{equation}
\nuti_T^0=\sqrt{\nu_T^2-\sin^2\delta_{inc}} \quad \in \mathbb{R},
\end{equation}
\end{subequations}
since $\nu_L=1$ and $\nu_T=\frac{c_L}{c_T}>1$. However, if the incident wave is transverse, then $\tau=\nu_T\sin\delta_{inc}$ and we have
\begin{equation}
\nuti_T^0=\sqrt{\nu_T^2-\nu_T^2\sin^2\delta_{inc}}=\nu_T\cos\delta_{inc} \quad \in \mathbb{R},
\end{equation}
and for $\nuti_L^0$, two cases may occur :
\begin{flalign}
	&\bullet \mbox{ if } |\sin\delta_{inc}| \leq \frac{\nu_L}{\nu_T}, \mbox{then } \nuti_L^0=\sqrt{\nu_L-\nu_T^2\sin^2\delta_{inc}} \quad \in \mathbb{R} \label{noncr}\\
&\bullet \mbox{ if } |\sin\delta_{inc}| >\frac{\nu_L}{\nu_T},\mbox{ then } \nuti_L^0=\sqrt{1-\nu_T^2\sin^2\delta_{inc}}=i\sqrt{\nu_T^2\sin^2\delta_{inc}-1} \quad \in i\mathbb{R} \label{crit}
\end{flalign}
As we will see, in the case described by \eqref{noncr}, the computations made in the 3D case are analogous to those described in the previous chapter. On the other hand, in the case described by \eqref{crit}, the presence of an imaginary branch point modifies and complicates all the occurring complex integral contour deformations. In any case,
\begin{equation}
\zeta_*^{\varepsilon}(\xi)=\sqrt{\tilde{\nu}_*^{\varepsilon^2}-\xi^2}
\label{C4:defzeta}
\end{equation}
%La racine carrée complexe étant définie au signe près, $\sqrt{z}$ peut prendre deux valeurs distinctes pour $z$ fixé. C'est ce qu'on appelle les branches de coupure de la fonction. Les points de branchement sont les valeurs pour lesquelles ces branches se croisent (ici il s'agit de $\xi=\pm \tilde{\nu}_*^{\epsilon}$). Afin de bien définir $\zeta_*^{\epsilon}(\xi)$, on choisit la branche dont la partie imaginaire est positive. 

The inner integral of \eqref{C4:invdouble} is computed using Cauchy's residue theorem :
\begin{equation}
v_j^{\varepsilon}(x_j,y_j)=\frac{i}{4\pi}e^{2i\varepsilon}\int_{\mathbb{R}} e^{ix_j\xi}\sum_{*=L,T}e^{i|y_j|\zeta_*^{\varepsilon}(\xi)}\mathbf{M_*^{\varepsilon}}(\xi,\mbox{sgn }y_j)\Sigma_j(\xi)\,d\xi
\label{C4:vjeps}
\end{equation}
where $t=\mbox{sgn}( y_j)$ and $\mathbf{M_*^{\varepsilon}}(\xi,t), *=L,T$ are defined by
\begin{subequations}
\begin{equation}
\mathbf{M_L^{\varepsilon}}(\xi,t)=\begin{pmatrix}
\frac{\xi^2}{\zeta_L^{\varepsilon}} &t\xi & \frac{\xi\tau}{\zeta_L^{\varepsilon}} \\
t\xi & \zeta_L^{\varepsilon} & t\tau \\
\frac{\xi\tau}{\zeta_L^{\varepsilon}}& t\tau & \frac{\tau^2}{\zeta_L^{\varepsilon}}
\end{pmatrix}
\label{C4:MLeps}
\end{equation}
\begin{equation}
\mathbf{M_T^{\varepsilon}}(\xi,t)=\begin{pmatrix}
\zeta_T^{\varepsilon} +\frac{\tau^2}{\zeta_T^{\varepsilon}}& -t\xi &-\frac{\xi\tau}{\zeta_T^{\varepsilon}}\\
-t\xi & \frac{\xi^2+\tau^2}{\zeta_T^{\varepsilon}}&-t\tau \\
-\frac{\xi\tau}{\zeta_T^{\varepsilon}}&-t\tau&\zeta_T^{\varepsilon} +\frac{\xi^2}{\zeta_T^{\varepsilon}}
\end{pmatrix}
\label{C4:MTeps}
\end{equation}
\label{C4:M*eps}
\end{subequations}

These matrices can be computed in two different manners, providing a way to test this result. The first way to compute these matrices is by direct computation of the residues of matrix $(\mathbf{M}-e^{-2i\varepsilon}\mathbf{\mathbb{I}_3})^{-1}$, where matrix $\mathbf{M}$ is given by \eqref{C4:matM}, at poles $\pm \zeta^{\varepsilon}_*(\xi), *=L,T$ when applying the Cauchy theorem to compute the inner integral of \eqref{C4:invdouble} to obtain \eqref{C4:vjeps}. The second option for computing these matrices starts by computing the eigen vectors and eigen values of $\mathbf{M}$. The three eigen vectors of $\mathbf{M}$ and the corresponding eigen values are :
\begin{subequations}
\begin{equation}
    \mathbf{M}\begin{pmatrix}
    \xi \\ \eta \\ \tau 
    \end{pmatrix}
    =(\xi^2+\eta^2+\tau^2)\begin{pmatrix}
    \xi \\ \eta \\ \tau 
    \end{pmatrix}
\end{equation}
\begin{equation}
   \mathbf{M}\begin{pmatrix}
    -\eta \\ \xi \\ 0
    \end{pmatrix}
    =\frac{\xi^2+\eta^2+\tau^2}{\nu_T^2}\begin{pmatrix}
    -\eta \\ \xi \\ 0
    \end{pmatrix} 
\end{equation}
\begin{equation}
   \mathbf{M}\begin{pmatrix}
    -\xi\tau \\ \eta\tau \\ \xi^2+\eta^2
    \end{pmatrix}
    =\frac{\xi^2+\eta^2+\tau^2}{\nu_T^2}\begin{pmatrix}
    -\xi\tau \\ \eta\tau \\ \xi^2+\eta^2
    \end{pmatrix} 
\end{equation}
\end{subequations}
These three vectors are linearly independent and constitute a vector basis of $\mathbb{C}^3$. This means that any vector of $\mathbb{C}^3$ can be expressed as a linear combination of these three vectors. Notably :
\begin{equation}
\Sigma_j=
    \begin{pmatrix}
    \hat{\alpha}_j\\ \hat{\beta}_j\\ \hat{\gamma}_j
    \end{pmatrix}
    = \frac{\xi\hat{\alpha}_j+\eta\hat{\beta}_j+\tau\hat{\gamma}_j}{\xi^2+\eta^2+\tau^2}\begin{pmatrix}
    \xi \\ \eta \\ \tau 
    \end{pmatrix} + \frac{\xi\hat{\beta}_j-\eta\hat{\alpha}_j}{\xi^2+\eta^2} \begin{pmatrix}
    -\eta \\ \xi \\ 0
    \end{pmatrix} + \frac{(\xi^2+\eta^2)\hat{\gamma}_j-\tau(\xi\hat{\alpha}_j+\eta\hat{\beta}_j)}{(\xi^2+\eta^2)(\xi^2+\eta^2+\tau^2)} \begin{pmatrix}
    -\xi\tau \\ \eta\tau \\ \xi^2+\eta^2
    \end{pmatrix}
\end{equation}
This allows us to write the term $(\mathbf{M}-e^{-2i\varepsilon}\mathbb{I}_3)^{-1} \Sigma_j$ as the sum of three contributions :
\begin{multline}
(\mathbf{M}-e^{-2i\epsilon}\mathbb{I}_3)^{-1} \Sigma_j=\frac{\xi\hat{\alpha}_j+\eta\hat{\beta}_j+\tau\hat{\gamma}_j}{\xi^2+\eta^2+\tau^2}\lbrack(\xi^2+\eta^2+\tau^2)-e^{-2i\varepsilon} \rbrack^{-1} \begin{pmatrix}
    \xi \\ \eta \\ \tau 
    \end{pmatrix} \\
+\frac{\xi\hat{\beta}_j-\eta\hat{\alpha}_j}{\xi^2+\eta^2} \lbrack \frac{\xi^2+\eta^2+\tau^2}{\nu_T^2}-e^{-2i\varepsilon} \rbrack^{-1} \begin{pmatrix}
    -\eta \\ \xi \\ 0
    \end{pmatrix}\\
    +\frac{(\xi^2+\eta^2)\hat{\gamma}_j-\tau(\xi\hat{\alpha}_j+\eta\hat{\beta}_j)}{(\xi^2+\eta^2)(\xi^2+\eta^2+\tau^2)} \lbrack \frac{\xi^2+\eta^2+\tau^2}{\nu_T^2}-e^{-2i\varepsilon} \rbrack^{-1} \begin{pmatrix}
    -\xi\tau \\ \eta\tau \\ \xi^2+\eta^2
    \end{pmatrix}
    \label{invspectre}
\end{multline}
Expression \eqref{invspectre} therefore simplifies the evaluation of the residues of the integral on $\eta$ in \eqref{C4:invdouble} at poles $\pm \zeta^{\varepsilon}_*(\xi)$. This second computation method thus yields
\begin{equation}
v_j^{\varepsilon}(x_j,y_j)=\frac{i}{4\pi}e^{2i\varepsilon}\int_{\mathbb{R}} e^{ix_j\xi}\sum_{*=L,TH,TV}e^{i|y_j|\zeta_*^{\varepsilon}(\xi)}\mathbf{M_*^{\varepsilon}}(\xi,\mbox{sgn }y_j)\Sigma_j(\xi)\,d\xi
\label{C4:vjeps2}
\end{equation}
where $\mathbf{M_L^{\varepsilon}}(\xi,t),$ is given by \eqref{C4:MLeps} and
\begin{subequations}
\begin{equation}
\mathbf{M_{TV}^{\varepsilon}}(\xi,t)=\begin{pmatrix}
\frac{\xi^2\tau^2}{\zeta_T^{\varepsilon}(\xi^2+\zeta^2)}&\frac{t\xi\tau^2}{\xi^2+\zeta_T^{\varepsilon 2}}&\frac{-\xi\tau}{\zeta_T^{\varepsilon}}\\
\frac{t\xi\tau^2}{\xi^2+\zeta_T^{\varepsilon 2}}&\frac{\zeta_T^{\varepsilon}\tau^2}{\xi^2+\zeta_T^{\varepsilon 2}}&-t\tau\\
\frac{-\xi\tau}{\zeta_T^{\varepsilon}}&-t\tau&\frac{\xi^2+\zeta_T^{\varepsilon 2}}{\zeta_T^{\varepsilon}}
\end{pmatrix}
\label{C4:MTVeps}
\end{equation}
\begin{equation}
\mathbf{M_{TH}^{\varepsilon}}(\xi,t)=\left(1+\frac{\tau^2}{\xi^2+\zeta_T^{\varepsilon 2}}\right)\begin{pmatrix}
\zeta_T^{\varepsilon}&-t\xi&0\\
-t\xi&\frac{\xi^2}{\zeta_T^{\varepsilon}}&0\\
0&0&0
\end{pmatrix}
\label{C4:MTHeps}
\end{equation}
\label{C4:M*eps2}
\end{subequations}
Note that $ \mathbf{M_T^{\varepsilon}}=\mathbf{M_{TH}^{\varepsilon}}+\mathbf{M_{TV}^{\varepsilon}}$. Expressions \eqref{C4:vjeps} and \eqref{C4:vjeps2} are equivalent.

Integral \eqref{C4:vjeps} or \eqref{C4:vjeps2} is well defined for Im$\zeta_*^{\varepsilon}>0$, so that the exponential in the integral decreases with the distance $y_j$. The branch cut for the square root in the definition of $\zeta_*^{\varepsilon}$ \eqref{C4:defzeta} is therefore defined by:
\begin{equation}
\zeta_*^{\varepsilon}=\left\{
\begin{matrix}
i \sqrt{\xi^2-\nuti_*^{\epsilon 2}} & \rm if & |\xi|\geq |\nuti_*^{\epsilon}|
\\
-\sqrt{\nuti_*^{\epsilon 2}-\xi^2} & \rm if & |\xi|< |\nuti_*^{\epsilon}|
\end{matrix}
\right.
\label{C4:zetabranchcut}
\end{equation}
For all values of $\varepsilon \in \rbrack 0, \pi \lbrack $, the integration contour never crosses the branch points of $\zeta_*^{\varepsilon}$, which are located at $\pm \tilde{\nu}_*^{\varepsilon}$ ($\nuti_*^{\varepsilon}$ is given by \eqref{defnutieps}), outside of the real axis.

According to Croisille et Lebeau \cite{CroisilleLebeau}, convergence in the 2D case is verified for $\varepsilon \rightarrow 0$. We will suppose that this is still the case in 3D. The integration contour $\mathbb{R}$ is deformed into contour $\Gamma_{0}$, visible on Fig.~\ref{C4:Gamma0_noncr} in the case where $\nuti_L^0\in\mathbb{R}$, described in \eqref{noncr} and on Fig.~\ref{C4:Gamma0_cr} in the other case, described in \eqref{crit}. Physically, if we define a critical angle for diffraction $\delta_c$ by $\sin\delta_c=\frac{\nu_L}{\nu_T}$, then the case described in \eqref{noncr} corresponds to the case where the incident skew angle (the angle between the incident wave vector and the plane normal to the wedge edge) is lower than the critical angle. On the contrary, the case described by \eqref{crit} corresponds to the case where the incident skew angle is higher than this critical angle and, according to Snell's law of diffraction \eqref{C4:Snelldiff}, there is no diffracted longitudinal wave. In both cases, the branch points of the integrand are avoided.

In all the following, superscript $\varepsilon=0$ will be omitted in order to alleviate notations. Finally:
\begin{equation}
v_j(x_j,y_j)=\frac{i}{4\pi}\int_{\Gamma_0} e^{ix_j\xi}\sum_{*=L,T}e^{i|y_j|\zeta_*(\xi)}\mathbf{M_*}(\xi,\mbox{sgn }y_j)\Sigma_j(\xi)\,d\xi
\label{C4:vj0}
\end{equation}

%The branch cut chosen for the definition of $\zeta_*, \, *=L,T$ is the one for which the square root's imaginary part is positive, in order to satisfy the following radiation condition at infinity :
%\begin{equation}
%\underset{y_j\rightarrow +\infty}{\rm lim}||v_j(x_j,y_j)||=0
%\end{equation}
%The branch cut is defined by 
% \begin{eqnarray}
% \zeta_*(\xi)=
% \left\{
% \begin{array}{lr}
% i\sqrt{\xi^2-\tilde{\nu}_*^2}& \mbox{if } |\xi| \geq |\nuti_*| \\
% -\sqrt{\tilde{\nu}_*^2-\xi^2}& \mbox{if } |\xi| \leq |\nuti_*|
% \end{array}
% \right.
% \label{defzeta}
% \end{eqnarray}

\begin{figure}
\centering
\begin{subfigure}[b]{0.45\textwidth}
\begin{tikzpicture}
\node at (0,0) {$\times$};
\node at (0.35,0.35) {$0$};
\node at (2.25,0) {$\times$}; % Pole
\node at (1.5,0) {$\times$}; %pole
\node at (1.5,-0.45) {$\tilde{\nu}_L$};
\node at (2.25,-0.45) {$\tilde{\nu}_T$};
\node at (-1.5,0) {$\times$};
\node at (-2.25,0) {$\times$};
\node at (-1.5,0.45) {$-\tilde{\nu}_L$}; %pole
\node at (-2.25,0.45) {$-\tilde{\nu}_T$}; %pole
\node at (3.25,0.38) {$(\Gamma_0)$};
\node at (3.7,-0.38) {$\xi_1$};
\node at (0.38,2.25) {$\xi_2$};
%\draw[ thick, ->] (-2.5,-0.5) arc (180:235:1);
%\node at (-2.7,-0.9) {$F_1$};
%\draw[ thick, ->] (2.5,0.5) arc (0:45:1); %ici c'est les fleches 
%\node at (2.7,0.9) {$F_2$};
\draw[thick,black,yshift=0pt,decoration={markings,
mark=at position 1 with {\arrow{stealth}}},
postaction={decorate}](0,-2.5) -- (0,2.5);
\draw[thick,black,yshift=0pt,
decoration={ markings,  % This schema allows for fine-tuning the positions of arrows 
      mark=at position 0.1 with {\arrow{latex}},
      mark=at position 0.6 with {\arrow{latex}},
      mark=at position 0.9 with {\arrow{latex}},
      mark=at position 1 with {\arrow{stealth}}},
      postaction={decorate}]
      (-4,0) -- (-2.5,0)  arc (-180:0:0.25) -- (-1.75,0)  arc (-180:0:0.25)  -- (1.25,0)arc (180:0:0.25)  -- (2,0)arc (180:0:0.25) -- (3.75,0); % ca c'est l'axe
\end{tikzpicture}
\caption{Contour $\Gamma_0$ in the case where $\nuti_L \in \mathbb{R}$}
\label{C4:Gamma0_noncr}
\end{subfigure}
\hfill
\centering
\begin{subfigure}[b]{0.45\textwidth}
\begin{tikzpicture}
\node at (0,0) {$\times$};
\node at (0.35,0.35) {$0$};
\node at (2.25,0) {$\times$}; %pole
\node at (2.25,-0.45) {$\tilde{\nu}_T$};
\node at (-2.25,0) {$\times$};
\node at (-2.25,0.45) {$-\tilde{\nu}_T$}; %pole
\node at (3.5,0.38) {$(\Gamma_0)$};
\node at (0,1.5) {$\times$};
\node at (0.45, 1.5) {$\nuti_L$};
\node at (0,-1.5) {$\times$};
\node at (0.45,-1.5) {$-\nuti_L$};
\node at (3.9,-0.38) {$\xi_1$};
\node at (0.38,2.25) {$\xi_2$};
\draw[thick,black,yshift=0pt,decoration={markings,
mark=at position 0.99 with {\arrow{stealth}}},
postaction={decorate}](0,-2.5) -- (0,2.5);
\draw[thick,black,yshift=0pt,
decoration={ markings,  % This schema allows for fine-tuning the positions of arrows 
      mark=at position 0.1 with {\arrow{latex}},
      mark=at position 0.6 with {\arrow{latex}},
      mark=at position 0.9 with {\arrow{latex}},
      mark=at position 0.99 with {\arrow{stealth}}},
     postaction={decorate}]
      (-3.75,0) -- (-2.5,0)  arc (-180:0:0.25) -- (2,0)arc (180:0:0.25) -- (4,0); % ca c'est l'axe
\end{tikzpicture}
\caption{Contour $\Gamma_0$ in the case where $\nuti_L \in i\mathbb{R}$}
\label{C4:Gamma0_cr}
\end{subfigure}
\caption{Contour $\Gamma_0$ in the complex plane $\xi=\xi_1+i\xi_2$}
\label{C4:Gamma0}
\end{figure}

Integral formulation \eqref{C4:vj0} is an expression of the solution in terms of the unknown spectral function $\Sigma_j$. In the next section, a far-field approximation of this integral is derived and the diffraction coefficient is defined.

\subsection{Far field approximation}
$x=(x'_1,y'_1,z'_1)=(r\cos\theta\cos\delta_{\beta},r\sin\theta\cos\delta_{\beta},-r\sin\delta_{\beta})$ is an observation point, indexed by its spherical coordinates, visible on Fig.~\ref{diedre_coords}. In order for the diffracted field to be observable at this point, $x$ is located on one of Keller's cones of diffraction, visible on Fig.~\ref{C4:Kellercone}. The observation skew angle $\delta_{\beta}$ is set by Snell's law of diffraction \eqref{C4:Snelldiff}.

\begin{figure}
\centering
\includegraphics[width=\textwidth]{images/chapter4/Keller_annotated.png}
        \caption{Keller's cone of diffraction}
        \label{C4:Kellercone}
\end{figure}

According to \eqref{C4:adiming},the scattered field at point $P$ is :
\begin{equation}
u_0(x'_1,y'_1,z'_1)=v(\frac{\omega}{c_L}r\cos\theta\cos\delta_{\beta},\frac{\omega}{c_L}r\sin\theta\cos\delta_{\beta})e^{-ik_{\beta}\sin\delta_{\beta}z'_1}
\end{equation}
The far field parameter is $R=\frac{\omega r}{c_L}$. The aim is to determine the asymptotic behavior of $v(R\cos\theta\cos\delta_{\beta},R\sin\theta\cos\delta_{\beta})$ when $R\rightarrow +\infty$. The first step is to apply the following change of variables in integral $v_1$, given by \eqref{C4:vjeps} :
\begin{equation}
\begin{split}
\xi&=\tilde{\nu}_*\cos\lambda \\
d\xi&=-\tilde{\nu}_*\sin\lambda\, d\lambda
\end{split}
\label{C4:changevar2}
\end{equation}
yielding, when $\noncr$
\begin{equation}
v_1(r,\theta,\delta_{\beta})=\frac{i}{4\pi} \int_{C_0}\sum_{*=L,T}\tilde{\nu}_*^2 e^{i\tilde{\nu}_*R\cos\delta_{\beta}\cos(\lambda+\bar{\theta})}\mathbf{ P_*}(\lambda,t)\Sigma_1(\tilde{\nu}_*\cos\lambda) \, d \lambda
\label{C4:v1C0}
\end{equation}
where $\bar{\theta}$ has been defined in the second chapter of this manuscript by \eqref{obs} and
\begin{equation}
\mathbf{P_L}(\lambda,t)=
\begin{pmatrix}
\cos^2\lambda & -t\cos\lambda\sin\lambda &\frac{\tau}{\tilde{\nu}_L} \cos\lambda \\
-t\cos\lambda\sin\lambda & \sin^2\lambda&-t\frac{\tau}{\tilde{\nu}_L}\sin\lambda \\
\frac{\tau}{\tilde{\nu}_L} \cos\lambda&-t\frac{\tau}{\tilde{\nu}_L}\sin\lambda&\frac{\tau^2}{\tilde{\nu}_L^2}
\end{pmatrix}
\end{equation}
and
\begin{equation}
\mathbf{P_T}(\lambda,t)=
\begin{pmatrix}
\sin^2\lambda+\frac{\tau^2}{\tilde{\nu}_T^2} & t\cos\lambda\sin\lambda &-\frac{\tau}{\tilde{\nu}_T}\cos\lambda \\
t\cos\lambda\sin\lambda & \cos^2\lambda+\frac{\tau^2}{\tilde{\nu}_T^2}&t\frac{\tau}{\tilde{\nu}_T}\sin\lambda \\
-\frac{\tau}{\tilde{\nu}_T}\cos\lambda&t\frac{\tau}{\tilde{\nu}_T}\sin\lambda&1
\end{pmatrix}
\end{equation}
$t=\mbox{sgn} \sin\theta$ and contour $C_0$ is visible on Fig.~\ref{C4:steepestcontour}. Note that contour $C_0$ does not fit exactly on the $\frac{\pi}{2}$-spaced grid in represented in the complex plane. This corresponds to adding an infinitely small imaginary part to the axis $\Gamma_0$ represented in Fig.~\ref{C4:Gamma0}, in order to avoid the branch points $\xi=\pm \nuti_*$ located at $\lambda=0$ and $\lambda=\pi$, where $\lambda$ is given by \eqref{C4:changevar2}.

In the case defined by \eqref{crit}, let us define $\eta_L$ in the following manner :
\begin{equation}
\nuti_L=i\eta_L, \quad \eta_L \in \mathbb{R}
\label{defetaL}
\end{equation}
Variable change \eqref{C4:changevar2} then yields :
\begin{equation}
\begin{split}
\xi&=\nuti_L\cos\lambda=i\eta_L\cos(\lambda_1+i\lambda_2)\\
&=\eta_L(\sin\lambda_1\sinh\lambda_2+i\cos\lambda_1\cosh\lambda_2)
\end{split}
\label{changevar2L}
\end{equation}
The integration contour on $\xi$, $\Gamma_0$ follows the real axis, except near the branch point $\pm \nu_*$. The new contour integration contour $C_0^L$ on $\lambda$ (determined by variable change \eqref{changevar2L}), thus verifies :
\begin{equation}
\rm Im(\xi)=0 \iff \cos\lambda_1=0 \iff \lambda_1=\pm \frac{\pi}{2}
\label{pisur2}
\end{equation}
When $\xi$ travels along $\Gamma_0$, going from $-\infty$ to $+\infty$, then, according to \eqref{changevar2L} and \eqref{pisur2}, $\lambda_2$ goes from $-\infty$ to $+\infty$ or vice-versa, depending on the sign of $\lambda_1$. For example, if $\lambda_1=-\frac{\pi}{2}$, then 
\begin{equation}
\xi \to -\infty \iff \sin\lambda_1\sinh\lambda_2+i\cos\lambda_1\cosh\lambda_2 = -\sinh\lambda_2 \rightarrow -\infty \iff \lambda_2 \rightarrow +\infty
\end{equation}
 The final step to determining contour $C_0^L$ is therefore to determine the sign of $\lambda_1$. Suppose that $\lambda_1=-\frac{\pi}{2}$, then applying variable change \eqref{C4:changevar2} in integral formulation \eqref{C4:vjeps} produces an evanescent term :
\begin{equation}
\begin{split}
v_1(r,\theta,\delta_{\beta})&=-\frac{i}{4\pi} \int_{C_0^L}\eta_L^2 e^{-\eta_L R\cos\delta\cos(\lambda+\bar{\theta})}\mathbf{ P_L}(\lambda,t)\Sigma_1(i\eta_L\cos\lambda) \, d\lambda\\
&+\frac{i}{4\pi} \int_{C_0}\tilde{\nu}_T^2 e^{i\nuti_T R\cos\delta\cos(\lambda+\bar{\theta)})}\mathbf{ P_T}(\lambda,t)\Sigma_1(\tilde{\nu}_T\cos\lambda) \, d\lambda
\end{split}
\label{C4:v1C0evan}
\end{equation}
where contour $C_0^L$ is visible on Fig.~\ref{C4:steepestcontour}. The exponential term in the first integral of \eqref{C4:v1C0evan} is $e^{-\eta_L R\cos\delta\cos(\lambda+\bar{\theta})}$. In order to determine the behaviour of this term when $R\rightarrow +\infty$, the sign of the other terms in the exponential must be determined. Kowing $\eta_L>0$ and $\cos\delta>0$, it remains to determine the sign of $\cos(\lambda+\bar{\theta})$. For $\lambda \in C_0^L$, we have :
\begin{equation}
\lambda=-\dfrac{\pi}{2}+i\lambda_2
\end{equation}
%where the notation $\dfrac{\pi}{2}^+$ refers to a real number which tends to $\frac{\pi}{2}$ with superior values and $\lambda_2=\rm Im(\lambda)$. 
This yields
\begin{equation}
\cos(\lambda+\bar{\theta})=\cos(\bar{\theta}-\dfrac{\pi}{2})\cosh\lambda_2-i\sin(\bar{\theta}-\dfrac{\pi}{2})\sinh\lambda_2
\end{equation}
Having $\bar{\theta} \in \lbrack 0,\pi \rbrack$, we have $\cos(\bar{\theta}-\dfrac{\pi}{2})=\sin\bar{\theta}\geq 0$ and 
\begin{equation}
|e^{-\eta_L R\cos\delta\cos(\lambda+\bar{\theta})}|=e^{-\eta_L R\cos\delta\sin\bar{\theta}\cosh\lambda_2}
\end{equation}
The amplitude of the integrand in the first integral of \eqref{C4:v1C0evan} decreases exponentially as the distance from the edge grows. On the contrary, if $\lambda_1=\frac{\pi}{2}$, then we would have :
\begin{equation}
\cos(\lambda+\bar{\theta})=\cos(\bar{\theta}+\dfrac{\pi}{2})\cosh\lambda_2-i\sin(\bar{\theta}+\dfrac{\pi}{2})\sinh\lambda_2
\end{equation}
and
\begin{equation}
|e^{-\eta_L R\cos\delta\cos(\lambda+\bar{\theta})}|=e^{\eta_L R\cos\delta\sin\bar{\theta}\cosh\lambda_2}
\end{equation}
The amplitude of the integrand in the first integral of \eqref{C4:v1C0evan} would then increase exponentially as the distance from the edge grows, which is physically impossible. Therefore, $C_0^L$, visible on Fig.~\ref{C4:steepestcontour}, is defined by
\begin{equation}
C_0^L= -\dfrac{\pi}{2}-i\mathbb{R}
\end{equation}
%This means that it is an evanescent term.

\begin{figure}
\centering
\begin{tikzpicture}
	\draw[step=1.5cm,gray,very thin,dashed](-2.5,-2.7)grid(3.7,3.7);
	\draw[thin, decoration={ markings,  
		mark=at position 1 with {\arrow{stealth}}},
	postaction={decorate}] (-2.5,0)  -- (3.7,0) node[above]{$\lambda_1$};
	\draw[thin, decoration={ markings,  
		mark=at position 1 with {\arrow{stealth}}},
	postaction={decorate}](0,-2.5)--(0,3.5) node[left]{$\lambda_2$};

	\node at (2.8,-0.2) {$\pi$};
	\node at (-0.2,0.2) {$0$};
	\node at (0.92,0.5) {$\lambda_s$};
	\node at (3,3.1) {$C_0$};
	\node at (1.9,3.1) {$\gamma_{\theta}$};
	\node at (-2.0,3.3) {$C_0^L$};
	\node at (2.1,-0.005) {$\times$};
	\node at (2.1,-0.4) {$\lambda_c$};
	
	\draw[black,very thick, decoration={ markings,  
		mark=at position 0.2 with {\arrow{latex}},
		mark=at position 0.8 with {\arrow{latex}}},
	postaction={decorate}] (-1.5,3.5)--(-1.5,-2.5);
	
	\node at (-1.9,-0.5) {$-\dfrac{\pi}{2}$};
	
	\draw[black,very thick][domain=0:3.5] plot({2*pi/7 + acos(1/cosh(\x))*pi/180},\x);
	\draw[black,very thick, decoration={ markings,  
		mark=at position 0.2 with {\arrow{latex}}},
	postaction={decorate}][domain=0:-2.8] plot({2*pi/7 - acos(1/cosh(\x))*pi/180},\x);
	
	\draw[very thick,black,%xshift=0pt,
	decoration={ markings,  
		mark=at position 0.8 with {\arrow{latex}}},
	postaction={decorate}]
	(0.2,0.2) arc (0:-90:0.4) -- (-0.2,-3); 
	
	\draw[very thick,black,%yshift=0pt,
	decoration={ markings, mark=at position 0.5 with {\arrow{latex}}}, postaction={decorate}]
	(pi-0.57,0.2) -- (0.2,0.2);

	\draw[very thick,black,%xshift=0pt,
	decoration={ markings,  
		mark=at position 0.5 with {\arrow{latex}}},
	postaction={decorate}]
	(pi-0.4,3.2) -- (pi-0.4,0.4) arc (0:-90:0.2) ;

	\end{tikzpicture}
	\caption{Contours $C_0, C_0^L$ and $\gamma_{\theta}$ in the complex plane $\lambda=\lambda_1+i\lambda_2$. The stationary phase points are noted $\lambda_s$.}
	\label{C4:steepestcontour}
\end{figure}

The far-field evaluation of the integral is obtained by applying the steepest descent method, presented in appendix \ref{PhaseStationnaire}, to \eqref{C4:v1C0}. To do so, contour $C_0$ is deformed into contour $\gamma_{\theta}$, also visible in Fig.~\ref{C4:steepestcontour}. In the case $\nuti_L \in \mathbb{R}$, this leads to
\begin{equation}
v_1=v_1^{sing}+v_1^{diff}
\end{equation}
where $v_1^{sing}$ is the contribution of all the singularities of the spectral functions crossed during the deformation from $C_0$ to $\gamma_{\theta}$, corresponding to the reflected waves (for the poles of the spectral functions) and head waves (for the branch points $\lambda_c$ of the function $v_1$), and $v_1^{diff}$ is the contribution of the stationary phase point $\lambda_s=\pi-\overline{\theta}$, corresponding to the edge-diffracted wave and computed using \eqref{steepformula}. In the following, it is assumed that the saddle point $\lambda_s$ does not coalesce with a branch point. The branch points of functions $v_j, \, j=1,2$ are located at $\xi=\pm \nuti_L$ and $\xi = \pm \nuti_T$. Applying \eqref{C4:changevar2}, this means that :
\begin{subequations}
\begin{equation}
\nuti_*\cos\lambda_s=-\nuti_*\cos\theta=\pm \nuti_L
\label{C4:eqbranch1}
\end{equation}
\begin{equation}
\rm or \hspace{1em} \nuti_*\cos\lambda_s=-\nuti_*\cos\theta=\pm \nuti_T
\label{C4:eqbranch2}
\end{equation}
\label{C4:eqbranch}
\end{subequations}
For $*=L$, \eqref{C4:eqbranch1} yields $\theta=0$ or $\theta=\pi$, meaning that the direction of observation is grazing along the wedge's horizontal face $\mathcal{S}_1$. \eqref{C4:eqbranch2} does not have a real solution for $*=L$. For $*=T$, \eqref{C4:eqbranch1} yields $\theta=\theta_c=\rm acos(\nuti_L/\nuti_T)$ or $\theta=\pi- \theta_c$, $\theta_c$ is called the critical angle for reflection, and \eqref{C4:eqbranch2} yields $\theta=0$ or $\theta=\pi$. Borovikov \cite{Borovikov} gives some clues as to how to treat the case where the stationary phase point coincides with another singularity of the integrand but no high-frequency asymptotics prove satisfactory in some situations of practical interest \cite{Gridin} and are neither available when the critical transition zones overlap penumbras, that is when all three critical points (stationary, critical and pole) coalesce \cite{KamotskiCrit}. In the present work, only the contribution of the edge-diffracted waves will be computed. In order to simplify notations, we will note $\mathbf{P}_*(\lambda,1)=\mathbf{P}_*(\lambda)$, using the fact that $t=\pm 1$ and $\mathbf{P}_*(\lambda,-1)=\mathbf{P}_*(-\lambda)$). The contribution of diffracted waves is 
\begin{equation}
v_1^{diff}(r,\theta,\delta_{\beta})=\frac{e^{-i\pi/4}}{2\sqrt{2\pi}}\sum_{*=L,T}\tilde{\nu}_*^2\frac{e^{-i\tilde{\nu}_*R\cos\delta_{\beta}}}{\sqrt{\nuti_*R\cos\delta_{\beta}}}\mathbf{P_*}(\pi-\theta)\Sigma_1(-\tilde{\nu}_*\cos\theta)
\end{equation}
Analogously,
\begin{equation}
v_2^{diff}(r,\varphi-\theta,\delta_{\beta})=\frac{e^{-i\pi/4}}{2\sqrt{2\pi}}\sum_{*=L,T}\tilde{\nu}_*^2\frac{e^{-i\tilde{\nu}_*R\cos\delta_{\beta}}}{\sqrt{\nuti_*R\cos\delta_{\beta}}}\mathbf{P_*}(\pi-(\varphi-\theta))\Sigma_1(-\tilde{\nu}_*\cos(\varphi-\theta))
\end{equation}

In the case where $\nuti_L=i\eta_L \in i\mathbb{R}$, the far-field evaluation is obtained by applying the steepest descent method, presented in appendix \ref{PhaseStationnaire}, to \eqref{C4:v1C0evan}. Contour $C_0$ is deformed into contour $\gamma_{\theta}$ and contour $C_0^L$ is deformed into contour $\gamma_{\theta}^L$. This leads to
\begin{equation}
v_1=v_1^{sing}+v_1^{diff}+v_1^{evan}
\end{equation}
where $v_1^{sing}$ is the contribution of all the singularities of the spectral functions crossed during the deformation from $C_0$ to $\gamma_{\theta}$ corresponding to the reflected waves (for the poles of $v_1$) and head waves (for the branch points $\lambda_c$ of the spectral functions, which are determined in the same manner as for the case $\noncr$ detailed above), $v_1^{diff}$ is the contribution of the stationary phase point to the integral on $C_0$, corresponding to the diffracted wave, and $v_1^{evan}$ is the contribution of the integral on $C_0^L$, which decays exponentially as the far-field parameter $R$ grows, making it an evanescent longitudinal wave. Only the contribution of the transversal diffracted waves will be computed here. Contribution $v_1^{diff}$ is computed using \eqref{steepformula} :
%. The contribution of evanescent waves is 
%\begin{equation}
%v^{evan}_1(R\cos\theta,R\sin\theta)= \sqrt{\frac{2\pi}{R\eta_L\cos\delta_{\beta}}} e^{-R\eta_L\cos\delta}\mathbf{P_L}(-\theta)\Sigma_1(i\eta_L\cos\theta)
%\end{equation}
%and the contribution of diffracted waves is :
\begin{equation}
v_1^{diff}(r,\theta,\delta_{\beta})=\frac{e^{-i\pi/4}}{2\sqrt{2\pi}}\tilde{\nu}_T^2\frac{e^{-i\tilde{\nu}_TR\cos\delta_{\beta}}}{\sqrt{\nuti_TR\cos\delta_{\beta}}}\mathbf{P_T}(\pi-\theta)\Sigma_1(-\tilde{\nu}_T\cos\theta)
\end{equation}
Analogously,
%\begin{equation}
%v^{evan}_2(R\cos\theta,R\sin\theta)= \sqrt{\frac{2\pi}{R\eta_L\cos\delta_{\beta}}} e^{-R\eta_L\cos\delta}\mathbf{P_L}(-(\varphi-\theta))\Sigma_1(i\eta_L\cos(\varphi-\theta))
%\end{equation}
%and
\begin{equation}
v_2^{diff}(r,\varphi-\theta,\delta_{\beta})=\frac{e^{-i\pi/4}}{2\sqrt{2\pi}}\tilde{\nu}_T^2\frac{e^{-i\tilde{\nu}_TR\cos\delta_{\beta}}}{\sqrt{\nuti_TR\cos\delta_{\beta}}}\mathbf{P_T}(\pi-(\varphi-\theta))\Sigma_2(-\tilde{\nu}_T\cos(\varphi-\theta))
\end{equation}
In any case, the total diffracted field is
\begin{equation}
v^{diff}=v_1^{diff}+v_2^{diff}
\end{equation}

Let us now isolate L, TH and TV diffracted waves in order to compute the corresponding diffraction coefficients, defined by
\begin{equation}
v_{\beta}^{diff}(r,\theta,\delta_{\beta})=D_{\beta}^{\alpha}(\theta)\frac{e^{-i\nuti_{\beta}R\cos\delta_{\beta}}}{\sqrt{\nuti_{\beta}R\cos\delta_{\beta}}} v^{inc}(r\cos\theta,r\sin\theta) \hat{i}_{\beta}
\label{C4:coeffdiff}
\end{equation}
Using the expressions of the unit vectors given by \eqref{C4:ivec}, the $\beta$ diffracted wave is given by $v^{diff}\cdot \hat{i}_{\beta}$. This yields :
\begin{equation}
D_{\beta}^{\alpha}(\theta)=\frac{e^{-i\pi/4}}{2\sqrt{2\pi}}\sum_{j=1,2}\nuti_{\beta}^2 \,{}^t \Sigma_j(-\nuti_{\beta}\cos\theta_j)\cdot\left(\mathbf{P}_{\beta}(\pi-\theta_j).\hat{i}_{\beta}\right)
\label{C4:Dbeta}
\end{equation}
where $\theta_1=\theta$ and $\theta_2=\varphi-\theta$.

In order to determine the field diffracted by a wedge illuminated by an incident plane wave, it is sufficient to compute the diffraction coefficient. This coefficient has been expressed in terms of two unknown functions called the spectral functions. The semi-analytical computation of these functions is presented in the following section

\section{Semi-analytical evaluation of the spectral functions}
The first step in computing the spectral functions is to determine a system of functional equations of which they are a solution. We will then show that these functions can be decomposed into two parts : a singular function, computed analytically, and a regular function, approached numerically.
\subsection{Functional equations}
In the previous section, the diffracted wave has been expressed in terms of two unknown functions called the spectral functions. In this subsection, a system of functional equations satisfied by these functions is determined. 

The first step in determining a system of functional equations verified by the spectral functions, is to substitute decomposition \eqref{C4:v1+v2} into the boundary conditions :
\begin{equation}
\left\{
\begin{matrix}
B \big( v_1(x_1,0)+v_2(x_2 \cos \varphi, x_2 \sin \varphi) \big) = -B \rm v_{\alpha}^{inc}|_{\mathcal{S}_1} \\
B \big( v_2(x_2,0)+v_1(x_1 \cos \varphi, x_1 \sin \varphi) \big) = -B \rm v_{\alpha}^{inc}|_{\mathcal{S}_2}
\end{matrix}
\right.
\label{C4:Bivi}
\end{equation}
Let us note $(v_j^1,v_j^2,v_j^3)$ the coordinates of $v_j$ in the Cartesian coordinate system $(x_j,y_j,z_j)$, where $(x_1,y_1,z_1)$ is the coordinate system associated with face $\mathcal{S}_1$ and $(x_2,y_2,z_2)$ is the coordinate system associated with face $\mathcal{S}_2$. These two coordinate systems are linked by (for $j=1,2$):
\begin{equation}
    \left\{
    \begin{matrix}
    x_j=\cos\varphi .x_{3-j}+\sin\varphi. y_{3-j}\\
    y_j=\sin\varphi .x_{3-j}-\cos\varphi .y_{3-j}\\
    z_j=z_{3-j}
    \end{matrix}
    \right.
    \label{C4:changerep}
\end{equation}
Applying \eqref{C4:changerep} to each line of \eqref{C4:Bivi} yields: 
\begin{equation}
\left\{
\begin{matrix}
B_1(v_1)+B_2(v_2)=-Bv_{\alpha}^{inc}|_{\mathcal{S}_1} \\
B_1(v_2)+B_2(v_1)=-Bv_{\alpha}^{inc}|_{\mathcal{S}_2}
\end{matrix}
\right.
\label{C4:b1v1+b2v2}
\end{equation}
where
\begin{equation}
B_1(v)=
\begin{pmatrix}
\mu \left( \frac{\partial v_1}{\partial y_1}+\frac{\partial v_2}{\partial x_1} \right) \\
\frac{\partial v_2}{\partial y_1}+\lambda \left( \frac{\partial v_1}{\partial x_1}+\frac{\partial v_3}{\partial z_1} \right)\\
\mu \left( \frac{\partial v_2}{\partial z_1}+ \frac{\partial v_3}{\partial y_1}\right)
\end{pmatrix}
\label{C4:B1v1expl}
\end{equation}
and
\begin{equation}
B_2(v)=
\begin{pmatrix}
\mu \sin(2\varphi)\left( \frac{\partial v_1}{\partial x_2}-\frac{\partial v_2}{\partial y_2}\right)-\mu \cos(2\varphi)  \left( \frac{\partial v_1}{\partial y_2}+\frac{\partial v_2}{\partial x_2} \right)\\
(\lambda+2\mu \sin^2\varphi) \frac{\partial v_1}{\partial x_2}+(\lambda+2\mu \cos^2 \varphi)\frac{\partial v_2}{\partial y_2}-\mu \sin(2\varphi)  \left( \frac{\partial v_1}{\partial y_2}+\frac{\partial v_2}{\partial x_2} \right)+\lambda \frac{\partial v_3}{\partial z_2} \\
\mu\sin\varphi\left(\frac{\partial v_3}{\partial x_2}+\frac{\partial v_1}{\partial z_2} \right)-\mu\cos\varphi\left( \frac{\partial v_2}{\partial z_2} +\frac{\partial v_3}{\partial y_2} \right)
\end{pmatrix}
\label{C4:B2v2expl}
\end{equation}
Operator $B_1$ is obtained by projecting $B(v_1)$ onto $\mathcal{S}_1$. This is immediate because $v_1$ is defined on $\mathcal{S}_1$ and its components $(v_1^1,v_1^2,v_1^3)$ are expressed in the associated Cartesian coordinate system $(x_1,y_1,z_1)$. Operator $B_2$ is obtained by projecting $B(v_2)$ onto $\mathcal{S}_1$. This is done by projecting its components $(v_2^1,v_2^2,v_2^3)$ onto $\mathcal{S}_1$ and by expressing $(x_1,y_1,z_1)$ as functions of $(x_2,y_2,z_2)$, as $v_2$ is only defined on $\mathcal{S}_2$. This is done using \eqref{C4:changerep}. The second equation of system \eqref{C4:b1v1+b2v2} is obtained in a similar manner, by reversing the roles of $v_1$ and $v_2$.

The functional equations system solved by the spectral functions is obtained by substituting the integral formulation \eqref{C4:vj0} of $v_1$ and $v_2$ into \eqref{C4:b1v1+b2v2}, evaluating the first equation at $x_1\geq 0, y_1=0$ and the second at $x_2\geq 0, y_2=0$ and applying the Fourier transform, defined by \eqref{defFouriersimple} to the result. This yields :
\begin{equation}
\begin{split}
\int_0^{+\infty} e^{-ix\xi}B_1(v_1)(x)\,dx&=\frac{1}{2}\textbf{DM}(\Sigma_1)(\xi) \\
&=\frac{1}{2} \int_{\Gamma_0}\textbf{DM}(\xi,\zeta)\Sigma_1(\zeta)\,d\zeta
\end{split}
\label{C4:B1DM}
\end{equation}
where
\begin{equation}
\begin{split}
\textbf{DM}(\xi,\zeta)&=\frac{1}{2i\pi} \frac{1}{\xi-\zeta} \textbf{dm}(\zeta) \\
&=\frac{1}{2i\pi} \frac{1}{\xi-\zeta} \begin{pmatrix}
-1 & \frac{\zeta}{\zeta_T}(1-2\mu Q(\zeta)) & 0\\
-\frac{\zeta}{\zeta_L}(1-2\mu Q(\zeta))  & -1&-\frac{\tau}{\zeta_L}(1-2\mu Q(\zeta)) \\
0&\frac{\tau}{\zeta_T}(1-2\mu Q(\zeta)) &-1
\end{pmatrix},
 \end{split}
\label{C4:defDM}
\end{equation}
$\zeta_*, *=L,T$ are defined by taking $\varepsilon=0$ in \eqref{C4:defzeta} and
\begin{equation}
Q(\zeta) =\zeta_L\zeta_T+\zeta^2+\tau^2
\end{equation}
The evaluation of $B_2(v_2)$ at $x_1\geq 0, y_1=0$ is the evaluation of  $B_2(v_2)$ at $x_2=x\cos\varphi, y_2=x\sin\varphi, x\geq 0$. The Fourier transform of the second term is therefore
\begin{equation}
\begin{split}
\int_0^{+\infty} e^{-ix\xi}B_2(v_2)(x)\,dx&=\frac{1}{2}\textbf{TM}(\Sigma_2)(\xi) \\
&=\frac{1}{2} \int_{\Gamma_0}\textbf{TM}(\xi,\zeta)\Sigma_2(\zeta)\,d\zeta
\end{split}
\label{C4:B2TM}
\end{equation}
where
\begin{equation}
\textbf{TM}(\xi,\zeta)=\frac{1}{2i\pi}\sum_{*=L,TH,TV}D_*(\xi,\zeta)\textbf{tm}_*(\zeta,\mbox{sgn } \sin \varphi),
\label{C4:defTM}
\end{equation}
\begin{equation}
D_*(\xi,\zeta)=\frac{1}{\xi-(\zeta \cos \varphi + \zeta_*(\zeta) |\sin \varphi|)},
\label{C4:defDstar}
\end{equation}
$\epsilon=$sgn sin $\varphi$, and the following matrices of rank 1 are defined :
\begin{equation}
\left\{
\begin{matrix}
\textbf{tm}_L(\zeta)=\left[ \frac{\zeta}{\zeta_L} f_L\,; \, \epsilon f_L\,;  \, \frac{\tau}{\zeta_L}f_L
\right] \\
f_L = \begin{pmatrix}
\mu \lbrack \cos(2\varphi)(2\epsilon\zeta\zeta_L)-\sin(2\varphi)(\zeta^2-\zeta_L^2) \rbrack\\
-\lambda+2\mu \lbrack \sin(2\varphi)(\epsilon\zeta\zeta_L)-\zeta^2\sin^2\varphi-\zeta^2_L\cos^2\varphi\rbrack\\
-2\mu\tau\lbrack \zeta\sin\varphi -\epsilon\zeta_L\cos\varphi\rbrack
\end{pmatrix}
\end{matrix}
\right. ,
\label{C4:tmL}
\end{equation}
\begin{equation}
\left\{
\begin{matrix}
\textbf{tm}_{TH}(\zeta)=\lbrack -tf_{TH}\,;\, \frac{\zeta}{\zeta_T}f_{TH}\,;\, 0 \rbrack\\
f_{TH}=\mu\left(1+\frac{\tau^2}{\zeta^2+\zeta_T^2}\right) \begin{pmatrix}
\sin(2\varphi)(2\epsilon\zeta\zeta_T)+\cos(2\varphi)(\zeta^2-\zeta_T^2)\\
\sin(2\varphi)(\zeta^2-\zeta_T^2)-\cos(2\varphi)(2\epsilon\zeta\zeta_T)\\
\tau\lbrack \epsilon\zeta_T\sin\varphi+\zeta\cos\varphi\rbrack 
\end{pmatrix}
\end{matrix}
\right.
\label{C4:tmTH}
\end{equation}
and
\begin{equation}
\left\{
\begin{matrix}
\textbf{tm}_{TV}(\zeta)=\lbrack \frac{\zeta\tau}{\zeta_T(\zeta^2+\zeta_T^2)}f_{TV}\,;\, \frac{\epsilon\tau}{\zeta^2+\zeta_T^2}f_{TV}\,;\, -\frac{1}{\zeta_T}f_{TV} \rbrack\\
f_{TV}=\mu\begin{pmatrix}
\tau\cos(2\varphi)(2\epsilon\zeta\zeta_T)-\tau\sin(2\varphi)(\zeta^2-\zeta_T^2)\\
2\tau\lbrack \sin(2\varphi)(\epsilon\zeta\zeta_T)-\zeta^2\sin^2\varphi-\zeta_T^2\cos^2\varphi\rbrack\\
\left(\tau^2-\zeta^2+\zeta_T^2\right)\lbrack \epsilon\zeta_T\cos\varphi-\zeta\sin\varphi \rbrack
\end{pmatrix}
\end{matrix}
\right.
\label{C4:tmTV}
\end{equation}
In the following, let us note for simplification:
\begin{equation}
\mathbf{tm}_T=\mathbf{tm}_{TH}+\mathbf{tm}_{TV}
\end{equation}

It has been checked that setting $\tau=0$ in the explicit expressions of $\mathbf{DM}$ and $\mathbf{TM}$ operators leads to the same expressions as those found in the previous chapter, concerning the 2D case.

Finally, the Fourier transform of the boundary conditions on the wedge faces is obtained by summing \eqref{C4:B1DM} and \eqref{C4:B2TM}. The right-hand side of the system is obtained by taking the Fourier transform of $-Bv_{\alpha}^{inc}|_{\mathcal{S}_j}, \; j=1,2$, where B is defined by \eqref{C4:Badim} and the incident field is given by \eqref{C4:vinc}. The final system of functional equations solved by the spectral functions is 
\begin{equation}
\left\{
\begin{matrix}
\textbf{DM}(\Sigma_1)+\textbf{TM}(\Sigma_2)=\dfrac{W_1^{\alpha}}{\xi-\nu_{\alpha} \cos \theta_{inc}\cos\delta_{inc}} 
\\
~\\
\textbf{TM}(\Sigma_1)+\textbf{DM}(\Sigma_2)=\dfrac{W_2^{\alpha}}{\xi-\nu_{\alpha}\cos(\varphi-\theta_{inc})\cos\delta_{inc}}
\end{matrix}
\right.
\label{C4:equationsintegrales}
\end{equation}
where
\begin{eqnarray}
\begin{array}{lr}
W_1^L=-2\begin{pmatrix}
\mu\cos^2\delta_{inc}\sin(2\theta_{inc}) \\
1-2\mu(\cos^2\theta_{inc}\cos^2\delta_{inc}+\sin^2\delta_{inc})\\
\mu\sin(2\delta_{inc})\sin(\theta_{inc})
\end{pmatrix}\\
~\\
 W_2^L=-2\begin{pmatrix}
\mu\cos^2\delta_{inc}\sin(2\varphi-2\theta_{inc}) \\
1-2\mu(\cos^2(\varphi-\theta_{inc})\cos^2\delta_{inc}+\sin^2\delta_{inc})\\
\mu\sin(2\delta_{inc})\sin(\varphi-\theta_{inc})
\end{pmatrix}\\ 
~\\
W_1^{TV}=2\nu_T\mu\begin{pmatrix}
\frac{1}{2}\sin(2\theta_{inc})\sin(2\delta_{inc})
\\
\sin(2\delta_{inc})\sin^2\theta_{inc}\\
-\sin\theta_{inc}\cos(2\delta_{inc})
\end{pmatrix}~
W_2^{TV}=2\nu_T\mu\begin{pmatrix}
\frac{1}{2}\sin(2\varphi-2\theta_{inc})\sin(2\delta_{inc})
\\
\sin(2\delta_{inc})\sin^2(\varphi-\theta_{inc})\\
-\sin(\varphi-\theta_{inc})\cos(2\delta_{inc})
\end{pmatrix} \\
~
\\
W_1^{TH}=-2 \nu_T\mu \begin{pmatrix}
\cos\delta_{inc}\cos(2\theta_{inc})
\\
\sin(2\theta_{inc})\cos\delta_{inc}\\
\cos\theta_{inc}\sin\delta_{inc}
\end{pmatrix}
~
W_2^{TH}=2 \nu_T\mu \begin{pmatrix}
\cos\delta_{inc}\cos(2\varphi-2\theta_{inc})\\
\sin(2\varphi-2\theta_{inc})\cos\delta_{inc}\\
\cos(\varphi-\theta_{inc})\sin\delta_{inc}
\end{pmatrix}
\end{array}
\label{C4:Wj}
\end{eqnarray}

Thanks to these functional equations, the spectral functions can be decomposed into two parts : a singular function and a regular function. The evaluation of each of these parts is described in the following.

\subsection{Singular part}
\label{C4:singpart}
The first step in evaluating the spectral functions is to determine their poles and corresponding residues. As in the previous chapters, this is done using a recursive procedure, using the following translation function which appears in \eqref{C4:defDstar} (for $*=L,T$) :
\begin{equation}
T_*(\xi=\nuti_*\cos\theta)=\xi \cos \varphi+\zeta_*(\xi)\sin \phiti=\tilde{\nu}_*\cos(\theta+\tilde{\varphi})
\end{equation}
where $\phiti$ is defined in the second chapter of this manuscript, by \eqref{phitilde}. This translation operator is defined on subspace $\Omega_*^+$, represented on Fig.~\ref{C4:domega0} :
\begin{equation}
\xi \in \Omega_*^+= \{ \xi=\tilde{\nu}_* \cos \theta, \; 0 \leq \mbox{Re} \theta < \pi-\tilde{\varphi} \}
\label{C4:defOmega0}
\end{equation}

\begin{figure}[h]
\centering
\begin{subfigure}[b]{0.45\textwidth}
   \begin{tikzpicture}[scale=0.9]
% Remplissage espace Omega_0
\fill [color=gray!20]
(3,-3.62) -- plot [domain=-2:2] ({-cosh(\x)},{sinh(\x)}) -- (3,3.62);
%(3,0) -- plot [domain=0:2] ({-cosh(\x)},{sinh(\x)}) -- (3,3.62) -- cycle;
\fill[color=white] (2,3) circle (0.5);
%\fill [color=white] (2,0) circle (0.25);

\node at (2,0) {$\times$};
\node at (2,-0.35) {$\tilde{\nu}_*$};
\node at (-2,0) {$\times$};
\node at (-1,0) {$\times$};
\node at (-2.1,0.35) {$-\tilde{\nu}_*$}; 
\draw[dashed,decoration={ markings,
      mark=at position 1.0 with {\arrow{>}}},
      postaction={decorate}]
      (-5.5,0) -- (-2.25,0)  arc (-180:0:0.25)  -- (1.75,0)arc (180:0:0.25) -- (3,0); 
\draw[black, very thick,decoration={ markings,  
      mark=at position 0.1 with {\arrow{latex}},
      mark=at position 0.9 with {\arrow{latex}}},
      postaction={decorate}][domain=-2:2] plot({-cosh(\x)}, {sinh(\x)});
\node at (-1.1,1.5) {$\mathbf{\partial \Omega_*}$};
\node at (0.05,-0.35) {$-\nuti_*\cos \varphi$};

% Espace Omega_0
\draw[thick] (2,3) circle (0.5);
\node at (2,3) {$\mathbf{\Omega_*^+}$};

\draw[ thick, ->] (0.8,0.3) arc (0:45:1); 

\draw[ thick, ->] (-2.6,-0.5) arc (0:45:-1); 

\end{tikzpicture}
\caption{Contour $\partial \Omega_*$ and domain $\Omega_*^+$ in the case $\noncr$. The curved arrows show deformation of contour $\Gamma_0$ onto $\partial \Omega_*$.}
\label{C4:domegaT}     
\end{subfigure}
\hfill
\begin{subfigure}[b]{0.45\textwidth}
   \begin{tikzpicture}[scale=0.9]
\fill [color=gray!20]
(-3.6269,3.55) -- plot [domain=-2:2] ({sinh(\x)},{-cosh(\x)}) -- (3.6269,3.55) ;
\fill[color=white] (3,2) circle (0.5);
\node at (2,0) {$\times$}; 
\node at (0,-1) {$\times$};
\node at (0,1.7) {$\times$};
\node at (0,-1.7) {$\times$};
\node at (2,-0.35) {$\tilde{\nu}_T$};
\node at (-2,0) {$\times$};
\node at (-2.1,0.35) {$-\tilde{\nu}_T$}; 
\draw[dashed,decoration={ markings,
      mark=at position 1.0 with {\arrow{>}}},
      postaction={decorate}]
      (-3.6269,0) -- (-2.25,0)  arc (-180:0:0.25)  -- (1.75,0)arc (180:0:0.25) -- (3.6269,0); % ca c'est l'axe
\draw[dashed,decoration={ markings,
      mark=at position 1.0 with {\arrow{>}}},
      postaction={decorate}] (0,-4) -- (0,3.55);
% Hyperbole (contour  partial_Omega_0 )
\draw[black, very thick,decoration={ markings,
      mark=at position 0.1 with {\arrow{latex}},
      mark=at position 0.9 with {\arrow{latex}}},
      postaction={decorate}][domain=-2:2] plot({sinh(\x)},{-cosh(\x)});
\node at (1,-0.8) {$-\nuti_L\cos\varphi$};
\node at (0.5,1.7) {$\nuti_L$};
\node at (0.5,-1.7) {$-\nuti_L$};
\node at (-2.3,-3.3) {$\mathbf{\partial \Omega_L}$};

\draw[ thick, ->] (3,-0.3) arc (0:-45:1); 

\draw[ thick, ->] (-2.6,-0.5) arc (0:45:-1); 

% Espace Omega_0
\draw[thick] (3,2) circle (0.5);
\node at (3,2) {$\mathbf{\Omega_L^+}$};
\end{tikzpicture}
\caption{Contour $\partial \Omega_L$ and domain $\Omega_L^+$ in the case $\crit$. The curved arrows show deformation of contour $\Gamma_0$ onto $\partial \Omega_L$.}
\label{C4:domegaL}
\end{subfigure}
\caption{Domains $\Omega_*$ and contours $\partial \Omega_*$ in cases $\noncr$ and $\crit$.}
\label{C4:domega0}
\end{figure}

In order to determine the action of operator $\mathbf{DM}$ on a simple pole $z, \, \rm Im z\geq 0$ (as it has been done in previous chapters in equations \eqref{Int_op_decomp_DM} and \eqref{GaussDM}) contour $\Gamma_0$ in \eqref{C4:B1DM} is deformed into contour $\Gamma_1$. Contour $\Gamma_1$ is visible in Fig.~\ref{C4:gamma1noncr} for the case $\nuti_L \in \mathbb{R}$ and Cauchy's residue theorem can then be applied for Im$z\geq 0$, Im$\xi <0 $ with $z\in \mathbb{C} \backslash  (\rbrack - \infty, -\nuti_L \rbrack \cup \{\pm \nuti_L,\pm \nuti_T \})$.

\begin{figure}
\centering
\begin{tikzpicture}
\node at (0,0) {$\times$};
\node at (0,0.35) {$0$};
\node at (2,0) {$\times$}; % Pole
\node at (4,0) {$\times$}; %pole
\node at (2,-0.36) {$\tilde{\nu}_L$};
\node at (4,-0.36) {$\tilde{\nu}_T$};

\node at (-2,0) {$\times$};
\node at (-4,0) {$\times$};
\node at (-2,0.36) {$-\tilde{\nu}_L$}; %pole
\node at (-4,0.36) {$-\tilde{\nu}_T$}; %pole
\node at (-3,-0.6) {$\Gamma_1$};

\draw[dashed, decoration={markings,
 mark=at position 1.0 with {\arrow{latex}}},
      postaction={decorate}] (-1.5,0) -- (6,0);
      
\draw[ thick, ->] (3.0,0.3) arc (0:45:1); %ici c'est les fleches 
  

\draw[thick, black, yshift=0pt,
decoration={ markings,  % This schema allows for fine-tuning the positions of arrows 
      mark=at position 0.1 with {\arrow{latex}},
      mark=at position 0.9 with {\arrow{latex}}},
      postaction={decorate}]
      (-6,0) -- (-4.25,0)  arc (-180:0:0.25) -- (-2.25,0)  arc (-180:0:0.25) -- (-1.75,0) arc(-90:90:0.5)  -- (-6,1);
\end{tikzpicture}
\caption{Contour $\Gamma_1$ in the case $\noncr$. The arrow shows the deformation of contour $\Gamma_0$ into $\Gamma_1$.}
\label{C4:gamma1noncr}
\end{figure}

In the case $\crit$, contour $\Gamma_1$ is visible in Fig.~\ref{C4:gamma1cr}. Fig.~\ref{gamma1crinter} shows an intermediate step in the contour deformation from $\Gamma_0$ to $\Gamma_1$. The arrow shows the direction of deformation of the quarter-cycle which links the two parts of $\Gamma_1$ (one which circumvents $\lbrack-\infty,-\nuti_T\lbrack$ and one which circumvents $\rbrack \nuti_L, +i\infty\rbrack $) in Fig.~\ref{gamma1crinter}. The radius of this quarter-cycle tends to infinity during the deformation, resulting in the final contour, which is the reunion of contours $\Gamma_1^a$ and $\Gamma_1^b$, represented in Fig.~\ref{gamma1crfinal}. Cauchy's residue theorem can then be applied for $z\geq 0$, Im$\xi <0 $ and $ z\in \mathbb{C} \backslash ( \, \rbrack - \infty, -\nuti_T \rbrack \cup \lbrack \nuti_L,+i\infty \lbrack \, \cup \, \{\pm \nuti_L,\pm \nuti_T \} \,)$. In both cases ($\noncr$ and $\crit$), application of the residue theorem yields :
\begin{equation}
\int_{\Gamma_0} \textbf{DM}(\xi,\zeta).\frac{1}{\zeta-z}\,d\zeta = \frac{\textbf{dm}(z)}{\xi-z}+\mathbf{D_p}(z,\xi),
\label{C4:GaussDM}
\end{equation}
where
\begin{equation}
\mathbf{D_p}(z,\xi)= \int_{\Gamma_1} \frac{\textbf{DM}(\xi,\zeta)}{\zeta-z}\,d\zeta
\label{C4:defDp}
\end{equation}

\begin{figure}
\centering
\begin{subfigure}[b]{0.45\textwidth}
\begin{tikzpicture}[scale=0.8]
\node at (0,0) {$\times$};
\node at (0.35,0.35) {$0$};
\node at (0,1.5) {$\times$}; 
\node at (3,0) {$\times$}; 
\node at (-0.36,1) {$\tilde{\nu}_L$};
\node at (3,-0.36) {$\tilde{\nu}_T$};

\node at (0,-1.5) {$\times$};
\node at (-3,0) {$\times$};
\node at (-0.5,-1.5) {$-\tilde{\nu}_L$}; 
\node at (-3,0.36) {$-\tilde{\nu}_T$};
%\node at (-3,-0.6) {$\Gamma_1^a$};
%\node at (1,3) {$\Gamma_1^b$};

\draw[dashed, decoration={markings,
 mark=at position 1.0 with {\arrow{>}}},
      postaction={decorate}] (-2.5,0) -- (4,0);
      
\draw[dashed, decoration={markings,
 mark=at position 1.0 with {\arrow{>}}},
      postaction={decorate}] (0,-2.5) -- (0,4);

\draw[thick, black, yshift=0pt,
decoration={ markings,  % This schema allows for fine-tuning the positions of arrows 
      mark=at position 0.1 with {\arrow{latex}},
      mark=at position 0.9 with {\arrow{latex}}},
      postaction={decorate}]
      (-5,0) -- (-3.25,0)  arc (-180:0:0.25) -- (-2.75,0) arc(-90:90:0.5)  -- (-3.7,1);
      
\draw[thick, black, yshift=0pt,
decoration={ markings,  
      mark=at position 0.1 with {\arrow{latex}},
      mark=at position 0.9 with {\arrow{latex}}},
      postaction={decorate}]
      (-0.5,3.0) -- (-0.5,1.6) arc(0:180:-0.5)  -- (0.5,4);

\draw[ thick, ->] (-3,3)--(-3.7,3.7);
      
\draw[thick, black, yshift=0pt, decoration={ markings,  
      mark=at position 0.5 with {\arrow{latex}}},
      postaction={decorate}]
      (-3.7,1)arc(180:63.8:2.2);
\end{tikzpicture}
\caption{Intermediate contour $\Gamma_1$. The arrow shows the direction of the deformation.}
\label{gamma1crinter}
\end{subfigure}
~
\begin{subfigure}[b]{0.45\textwidth}
\begin{tikzpicture}[scale=0.8]
\node at (0,0) {$\times$};
\node at (0.35,0.35) {$0$};
\node at (0,1.5) {$\times$}; 
\node at (3,0) {$\times$}; 
\node at (-0.36,1) {$\tilde{\nu}_L$};
\node at (3,-0.36) {$\tilde{\nu}_T$};

\node at (0,-1.5) {$\times$};
\node at (-3,0) {$\times$};
\node at (-0.5,-1.5) {$-\tilde{\nu}_L$}; 
\node at (-3,0.36) {$-\tilde{\nu}_T$};
\node at (-3,-0.6) {$\Gamma_1^a$};
\node at (1,3) {$\Gamma_1^b$};

\draw[dashed, decoration={markings,
 mark=at position 1.0 with {\arrow{>}}},
      postaction={decorate}] (-2.5,0) -- (4,0);
      
\draw[dashed, decoration={markings,
 mark=at position 1.0 with {\arrow{>}}},
      postaction={decorate}] (0,-2.5) -- (0,3.5);

\draw[thick, black, yshift=0pt,
decoration={ markings,  % This schema allows for fine-tuning the positions of arrows 
      mark=at position 0.1 with {\arrow{latex}},
      mark=at position 0.9 with {\arrow{latex}}},
      postaction={decorate}]
      (-5,0) -- (-3.25,0)  arc (-180:0:0.25) -- (-2.75,0) arc(-90:90:0.5)  -- (-5,1);
      
\draw[thick, black, yshift=0pt,
decoration={ markings,  % This schema allows for fine-tuning the positions of arrows 
      mark=at position 0.1 with {\arrow{latex}},
      mark=at position 0.9 with {\arrow{latex}}},
      postaction={decorate}]
      (-0.5,3.5) -- (-0.5,1.6) arc(0:180:-0.5)  -- (0.5,3.5);
      
\end{tikzpicture}
\caption{Final contour $\Gamma_1=\Gamma_1^a \cup \Gamma_1^b$}
\label{gamma1crfinal}
\end{subfigure}
\caption{Deformation of contour $\Gamma_0$ onto contour $\Gamma_1=\Gamma_1^a \cup \Gamma_1^b$ in the case $\crit$.}
\label{C4:gamma1cr}
\end{figure}

Similarly, in order to determine the action of operator $\mathbf{TM}$ on a simple pole $z, \, \rm Im z\geq 0$, contour $\Gamma_0$ in \eqref{C4:B2TM} is deformed into contour $\partial \Omega_L$ for the L terms and $\partial \Omega_T$ for the T terms, both of which are visible in Fig.~\ref{C4:domegaT} for the case $\noncr$. Cauchy's residue theorem is applied in that case, yielding, for Im$z\geq 0$, Im$\xi <0 $ with $z\in \mathbb{C} \backslash  (\rbrack - \infty, -\nuti_L \rbrack \cup \{\pm \nuti_L,\pm \nuti_T \})$ :
\begin{equation}
\int_{\Gamma_0} \textbf{TM}(\xi,\zeta).\frac{1}{\zeta-z}\,d\zeta = \sum_{*=L,T} \frac{\textbf{tm}_*(z)}{\xi-T_*(z)}\textbf{1}_{\Omega_*}(z)+\mathbf{T_p}(z,\xi)
\label{C4:GaussTM}
\end{equation}
where $\textbf{1}_{\Omega_*}(z)=1$ if $z\in \Omega_*$ and $\textbf{1}_{\Omega_*}(z)=0$ elsewhere and
\begin{equation}
\mathbf{T_p}(z,\xi)= \frac{1}{2i\pi} \sum_{*=L,T} \int_{\partial \Omega_*} D_*(\xi,\zeta) .\dfrac{\textbf{tm}_*(\zeta)}{\zeta-z}\, d\zeta
\label{C4:defTp}
\end{equation}

In the case where $\crit$, $\partial \Omega_L$ is visible in Fig.~\ref{C4:domegaL}. During the deformation of contour $\Gamma_0$ into contour $\partial\Omega_L$, the pole $z$, Im$z\geq 0$, is not crossed by the contour deformation which only spans a part of the lower half of the complex plane. Therefore, the pole $z$ does not contribute to the integral. However, a pole that may be crossed during this contour deformation (if such a pole exists) is $\zeta_0 \in \Omega_L^+$ such that $T_L(\zeta_0)=\xi$ and Im$\zeta_0<0$. We will now determine in which cases such a pole $\zeta_0$ may appear. This will be done by supposing that $\zeta_0$ exists and deriving some necessary conditions for the existence of $\zeta_0$ to be possible. 

Let us therefore suppose that there exists $\theta_0=\theta_0'+i\theta''_0 \in \mathbb{C}$ such that $\zeta_0=\nuti_L\cos\theta_0$ and $\xi=T_L(\zeta_0)=\nuti_L\cos(\theta_0+\phiti)$ with $\nuti_L=i\eta_L \in i\mathbb{R}$, we then have :
\begin{equation}
\rm Im \zeta_0=\eta_L\cos\theta_0'\cosh\theta_0''  
\end{equation}
Pole $\zeta_0$ is crossed by the deformation of contour $\Gamma_0$ into $\partial \Omega_L^+$ (visible on Fig.~\ref{C4:domegaL}) if and only if Im$\zeta_0<0$ and $\zeta_0 \in \Omega_L^+$, with $\Omega_L^+$ defined by \eqref{C4:defOmega0}. These conditions lead to the following condition on $\theta'_0$ :
\begin{equation}
\dfrac{\pi}{2}<\theta_0'<\pi-\phiti
\label{C4:condcr}
\end{equation}
Condition \eqref{C4:condcr} makes sense if and only if
\begin{equation}
\dfrac{\pi}{2}<\pi-\phiti \Leftrightarrow \phiti<\dfrac{\pi}{2}
\end{equation}
From hereon after, we will suppose that this is not the case (this restriction is only made when $\crit$) and that there is therefore no pole $\zeta_0$ such that $T_L(\zeta_0)=\xi$. In practice, this is not too restrictive (it corresponds to wedge angles $\varphi<\frac{\pi}{2}$ or $\varphi>\frac{3\pi}{2}$) since, as explained in \ref{Chapter5:sing_part}, the spectral functions method is less accurate for small values of $\phiti$. Therefore, when $\crit$, Cauchy's residue theorem yields, for Im$z\geq 0$, Im$\xi <0 $ with $z\in \mathbb{C} \backslash ( \rbrack - \infty, -\nuti_L \rbrack\cup \lbrack \nuti_L,+i\infty\lbrack \d, \cup \, \{\pm\tilde{\nu}_L,\pm\tilde{\nu}_T \})$ :
\begin{equation}
\int_{\Gamma_0} \textbf{TM}(\xi,\zeta).\frac{1}{\zeta-z}\,d\zeta = \frac{\textbf{tm}_T(z)}{\xi-T_T(z)}\textbf{1}_{\Omega_T}(z)+\mathbf{T_p}(z,\xi)
\label{C4:GaussTMcr}
\end{equation}
with $\mathbf{T_p}$ defined by \eqref{C4:defTp}. The hypotheses Im$z\geq 0$, Im$\xi <0 $ is made in all three chapters of this manuscript. Im$z<\geq0$ corresponds to the location of the poles of the spectral functions in the complex plane, as will be shown in the following, and Im$\xi<0$ simplifies the reasoning as it assures us that $z$ and $\xi$ are in two different half-planes and it allows us later to apply the "propagation of the solution" technique, detailed in \ref{C4:propag}.

It is important to note that in all the aforementioned contour deformations, no branch points $\pm \nuti_L$ or $\pm \nuti_T$ of the integrands are crossed (in the deformation represented in Fig.~\ref{C4:domegaL}, points $\pm \nuti_T$ are crossed but they are different from the branch points of the integrand which are $\pm \nuti_L$). Therefore, it is assumed that Croisille and Lebeau's \cite{CroisilleLebeau} proof that $\mathbf{D_p}(z,\cdot)$ and $\mathbf{T_p}(z,\cdot)$ belong to a special class of functions $\mathcal{H}^3$ can be adapted to the 3D case. It will therefore be assumed that in the case $\noncr$, $\mathbf{D_p}(z,\cdot) \in \mathcal{H}^3$ and $\mathbf{T_p}(z,\cdot) \in \mathcal{H}^3$ and in the case $\crit$, $\mathbf{D_p}(z,\cdot) \in \tilde{\mathcal{H}}^3$ and $\mathbf{T_p}(z,\cdot) \in \mathcal{H}^3$. $\mathcal{H}$ and $\tilde{\mathcal{H}}$ are defined hereafter
\begin{definition}
\label{C4:defH}
$\mathcal{H}$ is the space of the functions f analytical in $\mathbb{C}\backslash \rbrack -\infty,-\nuti_L\rbrack$ such that $\forall \epsilon \in \rbrack0,\pi \lbrack, f(e^{i\epsilon} \cdot)\in H^+$, where $H^+$ is defined in Def.~\ref{defHpl}.
\end{definition}
\begin{definition}
\label{C4:defHcrit}
$\tilde{\mathcal{H}}$ is the space of the functions f analytical in  $\mathbb{C} \backslash ( \, \rbrack - \infty, -\nuti_T \rbrack \cup \lbrack \nuti_L,+i\infty \lbrack \,)$ such that $\forall \epsilon \in \rbrack0,\pi \lbrack, f(e^{i\epsilon} \cdot)\in H^+$, where $H^+$ is defined in Def.~\ref{defHpl}.
\end{definition}

The recursive procedure used to extract all the poles and corresponding residues of the spectral functions is analogous to the one described in \ref{C3:singpart} and will not be repeated here. 
%Let us now extract all the poles and corresponding residues of the spectral functions, using system \eqref{C4:equationsintegrales} and results \eqref{C4:GaussDM} and \eqref{C4:GaussTM}. We begin with the following decomposition, for Im $\xi \le 0$ : 
%\begin{equation}
%\Sigma_j(\xi)= \frac{V_j^{(0)}}{\xi-Z^{(0)}_j}+X'_j(\xi)
%\label{C4:extraction1}
%\end{equation}
%where $V_j^{(0)}$ are to be determined, $X'_j$ is an unknown function and
%\begin{subequations}
%\begin{equation}
%Z_1^{(0)}=\nu_{\alpha} \cos \theta_{inc}\cos\delta_{inc}=\nuti_{\alpha}\cos\theta_{inc},
%\end{equation}
%\begin{equation}
%Z_2^{(0)}=\nu_{\alpha} \cos(\varphi-\theta_{inc})\cos\delta_{inc}=\nuti_{\alpha}\cos(\varphi-\theta_{inc}).
%\end{equation}
%\end{subequations}
%Substituting \eqref{C4:extraction1} into \eqref{C4:equationsintegrales} and applying \eqref{C4:GaussDM} yields
%\begin{eqnarray}
%\left\{
%\begin{array}{l}
%\textbf{DM}(X'_1)+\textbf{TM}(X'_2)+\textbf{TM}(\frac{V_2^{(0)}}{\xi-Z_2^{(0)}})=\frac{W^1_{\alpha}}{\xi-Z_1^{(0)}}-\frac{\textbf{dm}(Z_1^{(0)}).V_1^{(0)}}{\xi-Z_1^{(0)}}-\mathbf{D_p}(Z_1^{(0)},\xi).V_1^{(0)} \\
%~
%\\
%\textbf{TM}(X'_1)+\textbf{DM}(X'_2)+\textbf{TM}(\frac{V_1^{(0)}}{\xi-Z_1^{(0)}})=\frac{W^2_{\alpha}}{\xi-Z_2^{(0)}}-\frac{\textbf{dm}(Z_2^{(0)}).V_2^{(0)}}{\xi-Z_2^{(0)}}-\mathbf{D_p}(Z_2^{(0)},\xi).V_2^{(0)} 
%\end{array}
%\right.
%\label{C4:mille}
%\end{eqnarray}
%The singular terms in the right-hand side of this system are eliminated by setting
%\begin{equation}
%V_j^{(0)}=\textbf{dm}^{-1}(Z_j^{(0)}).W_j^{\alpha}
%\end{equation}
%In all the following, we will suppose that we have det$(\mathbf{dm}) \neq 0$. Applying \eqref{C4:GaussTM} reveals two new poles, leading to a second decomposition 
%\begin{equation}
%X'_j=X''_j+\frac{V_{j,L}^{(1)}}{\xi-Z_{j,L}^{(1)}}+\frac{V_{j,T}^{(1)}}{\xi-Z_{j,T}^{(1)}}
%\end{equation}
%where, for $*=L,T$ 
%\begin{equation}
%Z_{j,*}^{(1)}=T_*(Z_{3-j}^{(0)})
%\end{equation}
%and $V_{j,*}^{(1)}$ remain to be determined. Once again, they are chosen so as to eliminate the singular terms in the right-hand side of the system :
%\begin{equation}
%V_{j,*}^{(1)}=-\mathbf{dm}^{-1}(T_*(Z_{j}^{(0)})).\mathbf{tm_*}(Z_{3-j}^{(0)}).V_{3-j}^{(0)}
%\end{equation}
%These steps are repeated recursively as long as $\textbf{1}_{\Omega_L}(Z_{j,*}^{(k)}=1$ and $\textbf{1}_{\Omega_T}(Z_{j,*}^{(k)}=1$. 
In the end, we have, for  $\mbox{Im} \xi <0$
\begin{equation}
\Sigma_j(\xi)=Y_j(\xi)+X_j(\xi) \label{C4:decomp}
\end{equation}
where, when $\noncr$ :
\begin{equation}
Y_j(\xi)=\sum_k \sum_{*=L,T} \frac{V_{j,*}^{(k)}}{\xi-Z_{j,*}^{(k)}}
\label{C4:yj},
\end{equation}
with
\begin{equation}
\begin{matrix}
Z_{1}^{(0)}=\nu_{\alpha} \cos \theta_{inc}\cos\delta_{inc},  & Z_{2}^{(0)}=\nu_{\alpha} \cos(\varphi-\theta_{inc})\cos\delta_{inc} \\
Z_{j,L}^{(k+1)}= T_L(Z_{3-j,*}^{(k)}) &Z_{j,T}^{(k+1)}= T_T(Z_{3-j,*}^{(k)}) 
\end{matrix}
\label{C4:poles}
\end{equation}
and
\begin{equation}
\begin{matrix}
V_{j}^{(0)}=\textbf{dm}^{-1}(Z_{j}^{(0)}).W_j^{\alpha}\\
V_{j,L}^{(k+1)}=-\textbf{dm}^{-1}(Z_{j,*}^{(k+1)}).\textbf{tm}_L(Z_{3-j,*}^{(k)}).V_{3-j,*}^{(k)}.\textbf{1}_{\Omega_L}(Z_{3-j,*}^{(k)}) \\ 
V_{j,T}^{(k+1)}=-\textbf{dm}^{-1}(Z_{j,*}^{(k+1)}).\textbf{tm}_T(Z_{3-j,*}^{(k)}).V_{3-j,*}^{(k)}.\textbf{1}_{\Omega_T}(Z_{3-j,*}^{(k)}) 
\end{matrix}
\label{C4:residus}
\end{equation}
where $W_j^{\alpha}$ is given by \eqref{C4:Wj}. When $\crit$,
\begin{equation}
Y_j(\xi)=\sum_k \frac{V_{j,T}^{(k)}}{\xi-Z_{j,T}^{(k)}}
\label{C4:yjcr},
\end{equation}
with
\begin{equation}
\begin{matrix}
Z_{1}^{(0)}=\nu_T \cos \theta_{inc}\cos\delta_{inc},\\ 
Z_{2}^{(0)}=\nu_T \cos(\varphi-\theta_{inc})\cos\delta_{inc}\\
Z_{j,T}^{(k+1)}= T_T(Z_{3-j,T}^{(k)}) 
\end{matrix}
\label{C4:polescr}
\end{equation}
and
\begin{equation}
\begin{matrix}
V_{j}^{(0)}=\textbf{dm}^{-1}(Z_{j}^{(0)}).W_j^{\alpha}\\
V_{j,T}^{(k+1)}=-\textbf{dm}^{-1}(Z_{j,*}^{(k+1)}).\textbf{tm}_T(Z_{3-j,*}^{(k)}).V_{3-j,*}^{(k)}.\textbf{1}_{\Omega_T}(Z_{3-j,*}^{(k)}) 
\end{matrix}
\label{C4:residuscr}
\end{equation}
The recursive procedure stops when no more poles can be found by deforming contour $\Gamma_0$ into $\partial \Omega_L$ or $\partial \Omega_T$. In the 2D case, Croisille and Lebeau \cite{CroisilleLebeau} have shown that this defines a finite number of poles. The sequence of poles generated in the 3D case being similar to the ones generated in the 2D case (parameters $\nu_L$ and $\nu_T$ in the 2D case are replaced by parameters $\nuti_L$ and $\nuti_T$), their demonstration is still valid here. We have thus extracted all the poles from the spectral functions and have computed them analytically, along with their corresponding residues.

\subsection{Regular Part}
\label{C4:regpart}
The singular parts $Y_j$ of the spectral functions having been determined, two new functions $X_1$ and $X_2$ are defined by \eqref{C4:decomp}. In the following, a numerical approximation method for $X_j$ is proposed. In order to do so, a system of functional equations solved by $X_1, X_2$ is derived by subtracting vector 
\begin{equation}
\begin{pmatrix}
\textbf{DM}(Y_1)+\textbf{TM}(Y_2) \\
\textbf{TM}(Y_1)+\textbf{DM}(Y_2)
\end{pmatrix},
\end{equation}
from both sides of \eqref{C4:equationsintegrales}, where $Y_1$ and $Y_2$ are given by equations \eqref{C4:yj} to \eqref{C4:residuscr} :
\begin{equation}
\left\{ 
\begin{matrix}
\mathbf{DM}(X_1)(\xi)+\textbf{TM}(X_2)(\xi)=u_1(\xi)\\
\textbf{TM}(X_1)(\xi)+\textbf{DM}(X_2)(\xi)=u_2(\xi)
\end{matrix}
\right.,
\label{C4:regparteqn}
\end{equation}
with, for $j=1,2$
\begin{equation}
u_j(\xi)=-\sum_k \sum_{*=L,T} \left[ \mathbf{D_p}(Z_{j,*}^{(k)},\xi).V_{j,*}^{(k)}+\mathbf{T_p}(Z_{3-j,*}^{(k)},\xi).V_{3-j,*}^{(k)}\right]
\label{C4:scndmembre}
\end{equation}
It is assumed that Kamotski and Lebeau's \cite{KamotskiLebeau} proof that this system \eqref{regparteqn} has a unique solution in $\mathcal{H}^2$ can be adapted to the 3D case, meaning that system \eqref{C4:regparteqn} has a unique solution $(X_1,X_2)$ in $\mathcal{H}^3$ if $\noncr$ and in $\tilde{\mathcal{H}}^3$ if $\crit$ where $\mathcal{H}$ is defined by Def.~\ref{C4:defH} and $\tilde{\mathcal{H}}$ is defined by Def.~\ref{C4:defHcrit}. Once again, a numerical approximation of the regular parts $X_j$ will be computed using the Galerkin collocation method. 

The functional space $\mathcal{H}$ is approached by the finite-dimension subspace generated by basis functions $(\varphi_k)_{1 \leq k \leq 2N}$ defined by \eqref{Gal_basis}, with $(a_k)_{1 \leq k \leq 2N} \in \lbrack \nuti_L, + \infty \lbrack^N$. For a point $a_k \in \lbrack \nuti_L, + \infty \lbrack$, the corresponding Galerkin function $\varphi_k$ will have a pole at $-a_k \in \rbrack-\infty,-\nuti_L\rbrack$. The basis $(\varphi_k)_{1 \leq k \leq 2N}$ therefore generates a subspace of functions analytical in $\mathbb{C} \backslash  \rbrack-\infty,-\nuti_L\rbrack$.% The domain in which functions $X_1,X_2$ are regular is visible on Fig.~\ref{C4:gamma1noncr} where the branch on which the functions are not analytical is the branch excluded from the deformation of contour $\Gamma_0$ onto $\Gamma_1$.

The functional space $\tilde{\mathcal{H}}$ is approached by the finite-dimension subspace generated by basis functions $(\varphi_k)_{1 \leq k \leq 2N}$ defined by \eqref{Gal_basis}, with $(a_k)_{1 \leq k \leq N} \in \lbrack \nuti_T, + \infty \lbrack^N ,\; \; (a_k)_{N+1\leq k \leq 2N} \in \rbrack -i\infty, -\nuti_L\rbrack^N$. For a point $a_k \in \lbrack \nuti_T, + \infty \lbrack \cup \rbrack -i\infty, -\nuti_L\rbrack$, the corresponding Galerkin function $\varphi_k$ will have a pole at $-a_k \in \rbrack-\infty,-\nuti_T\rbrack \cup \lbrack \nuti_L,+i\infty\lbrack$. The basis $(\varphi_k)_{1 \leq k \leq 2N}$ therefore generates a subspace of functions analytical in $\mathbb{C} \backslash ( \rbrack-\infty,-\nuti_T\rbrack \cup \lbrack \nuti_L,+i\infty\lbrack)$. %The domain in which functions $X_1,X_2$ are regular is visible on Fig.~\ref{C4:gamma1cr} where the branch on which the functions are not analytical is the branch excluded from the deformation of contour $\Gamma_0$ onto $\Gamma_1$.

In both cases, functions $X_j$ are approximated in the adapted finite dimension subspace by :
\begin{equation}
 X_j(\xi) \approx \sum_{k=1}^N \tilde{X}_j^k \varphi_k(\xi) ,\hspace{1em} \tilde{X}_j^k \in \mathbb{C}^3
 \label{C4:Xj}
\end{equation}
Approximation \eqref{C4:Xj} is substituted into \eqref{C4:regparteqn} and the variable change $\zeta=iy$ is applied in the resulting system. This system is then evaluated at collocation points $\xi=b_1,...,b_{2N}$, leading to a linear system of equations which can be written in matrix form :
\begin{eqnarray}
\begin{pmatrix}
\mathbb{D}&\mathbb{T}\\
\mathbb{T}&\mathbb{D}
\end{pmatrix}
\begin{pmatrix}
\mathbb{X}_1 \\
\mathbb{X}_2
\end{pmatrix}
=
\begin{pmatrix}
\mathbb{U}_1 \\
\mathbb{U}_2
\end{pmatrix}
\Leftrightarrow
\left\{
\begin{array}{l}
(\mathbb{D}+\mathbb{T})(\mathbb{X}_1+\mathbb{X}_2)= \mathbb{U}_1+\mathbb{U}_2 \\
(\mathbb{D}-\mathbb{T})(\mathbb{X}_1-\mathbb{X}_2)= \mathbb{U}_1-\mathbb{U}_2
\end{array}
	\right.,
\label{C4:systmat}
\end{eqnarray}
where matrices $(6N\times 6N)$ are defined by $3\times3$ blocks:
\begin{equation}
\begin{split}
\mathbb{D}_{lk}&=\int_{-\infty}^{+\infty} \textbf{DM}(b_l,iy)e_{a_k}(y) \, dy =\frac{1}{2i\pi} \int_{-\infty}^{+\infty} \frac{\mathbf{dm}}{b_l-iy} 
%\begin{pmatrix}
%-1&A(iy) \\
%B(iy)&-1
%\end{pmatrix}
\sqrt{\frac{a_k}{\pi}}\frac{1}{y-ia_k} \, dy \\
&=- \frac{\sqrt{a_k}}{2\pi \sqrt{\pi}}
\begin{pmatrix}
\mathcal{D}_1(a,b) &\mathcal{D}_2^T(a,b) &0\\
-\mathcal{D}_2^L(a,b) &\mathcal{D}_1(a,b)&-\mathcal{D}_3^L(a,b)\\
0&\mathcal{D}_3^T(a,b)&\mathcal{D}_1(a,b)
\end{pmatrix}=\frac{\sqrt{a_k}}{2\pi \sqrt{\pi}}\mathbb{D}(a_k,b_l)
\end{split}
\label{C4:Dab}
\end{equation}
where functions $e_{a_k}$ are defined by \eqref{Galerkin_basis} and the explicit expressions of coefficients of matrix $\mathbb{D}(a,b)$ and their values are computed in appendix \ref{finalD3D}. The other matrices involved are, for $1\leq l,k \leq 2N$
\begin{equation}
\begin{split}
\mathbb{T}_{lk}&=\int_{-\infty}^{+\infty} \textbf{TM}(b_l,iy)e_{a_k}(y) \, dy 
=\frac{1}{2i\pi} \int_{-\infty}^{+\infty} \sum_{*=L,T} \frac{\textbf{tm}_* (iy, \mbox{sgn} \sin \varphi)}{b_l-T_*(iy)} \sqrt{\frac{a_k}{\pi}}\frac{1}{y-ia_k}\,dy \\
&=\frac{1}{2i\pi}\sqrt{\frac{a_k}{\pi}}\sum_{*=L,T} \int_{-\infty}^{+\infty} \frac{\textbf{tm}_*(iy,\epsilon)}{\lbrack b_l-(iy \cos \varphi +  \zeta_*(iy)| \sin \varphi|)\rbrack(y-ia_k)} \, dy,
\end{split}
\end{equation}
where $\epsilon= \mbox{sgn}( \sin \varphi)$. Let us define
\begin{equation}
\mathbb{T}_{lk}=\frac{1}{2i\pi}\sqrt{\frac{a_k}{\pi}}
\sum_{*=L,TH,TV}
\begin{pmatrix}
\mathcal{T}_1^*(a,b) &  \mathcal{T}_2^*(a,b) &\mathcal{T}_3^*(a,b) \\
\mathcal{T}_4^*(a,b) &\mathcal{T}_5^*(a,b)&\mathcal{T}_6^*(a,b)\\
\mathcal{T}_7^*(a,b)&\mathcal{T}_8^*(a,b)&\mathcal{T}_9^*(a,b)
\end{pmatrix}
=\frac{1}{2i\pi}\sqrt{\frac{a_k}{\pi}}\mathbb{T}(a_k,b_l)
\label{Tab}
\end{equation}
The explicit expressions of operators $\mathcal{T}_i^*, 1\leq i\leq9, *=L,TH,TV$ and their values are computed in \ref{finalT3D}. Finally:
\begin{equation}
\mathbb{X}_j=
\begin{pmatrix}
\tilde{X}_j^1\\
\vdots \\
\tilde{X}_j^{2N}
\end{pmatrix}
\hspace{3em}
\mathbb{U}_j=
\begin{pmatrix}
u_j(b_1)\\
\vdots \\
u_j(b_{2N})
\end{pmatrix}
\label{C4:vecXUj}
\end{equation}
where $u_j(\xi)$ is given by \eqref{C4:scndmembre}. The same considerations as those made in \ref{C3:regpart}, equations \eqref{Dpscm} to \eqref{uDT}, lead to an expression of $u_j$ with respect to operators $\mathbb{D}(\cdot,\cdot)$ and $\mathbb{T}(\cdot,\cdot)$ :
\begin{equation}
u_j(\xi)=-\frac{1}{2i\pi}\sum_{k}\sum_{*=L,T}\left(i\mathbb{D}(-Z_{j,*}^{(k)},\xi).V_{j,*}^{(k)}+\mathbb{T}(-Z_{3-j,*}^{(k)},\xi).V_{3-j,*}^{(k)} \right) +\frac{W_j^{\alpha}}{\xi-Z_j^{(0)}}
\label{C4:uDT}
\end{equation}

Using these results, the linear system \eqref{C4:systmat} is implemented and solved numerically using the C++ library Eigen, and an evaluation of the regular part of the spectral functions is obtained. However, for values of $\xi$ lying in certain parts of the complex plane, this evaluation is not sufficiently accurate. The technique used to solve this problem is called the propagation of the solution.
 
\subsection{Propagation of the solution}
\label{C4:propag}
The method called propagation of the solution is used to propagate the accuracy of the numerical approximation of the regular functions $X_1$ and $X_2$ from parts of the complex plane where they are evaluated accurately to parts of the complex plane where they are not. The validity domains in the complex plane will be detailed hereafter.

The first step of this procedure is to deform contour $\Gamma_0$ (visible in Fig.~\ref{C4:Gamma0}) in operator $\mathbf{DM}$ into contour $\Gamma_2$ in functional system \eqref{C4:regparteqn}. Contour $\Gamma_2$ is visible on Fig.~\ref{C4:contour2} for the case $\noncr$ and in Fig.~\ref{C4:contour2crit} for the case $\crit$. Fig.9~\ref{gamma2crinter} shows an intermediate step in the contour deformation from $\Gamma_0$ to $\Gamma_2$. The straight arrow shows the direction of deformation of the quarter-cycle linking the two parts of $\Gamma_2$ (the one which circumvents $\lbrack -\nuti_L,-i\infty \lbrack $ and the one which circumvents $\lbrack \nuti_T, +\infty \lbrack$) in fig.~\ref{gamma2crinter}. The radius of this quarter-cycle tends to infinity during the deformation, resulting in the final contour, which is the reunion of contours $\Gamma_2^a$ and $\Gamma_2^b$, represented in Fig.~\ref{gamma2crfinal}. During this deformation, the half-plane Im$\xi<0$ is crossed (with the exception of branch $\lbrack -\nuti_L,-\infty \lbrack$ in the case $\crit$). The contribution of pole $\zeta$ whose value is $\zeta=\xi$, Im$\xi<0$ crossed during this contour deformation is given by Cauchy's residue formula :
\begin{equation}
\int_{\Gamma_0} \textbf{DM}(\xi,\zeta)X_j(\zeta)\, d\zeta = \int_{\Gamma_2}  \textbf{DM}(\xi,\zeta)X_j(\zeta)\, d\zeta + \textbf{dm}(\xi).X_j(\xi)
\label{C4:DM2}
\end{equation}

\begin{figure}[h]
	\centering
	\begin{tikzpicture}
	\node at (0,0) {$\times$};
	\node at (0,0.35) {$0$};
	\node at (2,0) {$\times$}; % Pole
	\node at (2,-0.35) {$\nuti_L$};
	\node at (4,0) {$\times$};
	\node at (4,-0.35) {$\nuti_T$};
	\node at (-2,0) {$\times$};
	\node at (-2,0.35) {$-\nuti_L$};
	\node at (-4,0) {$\times$};
	\node at (-4,0.35) {$-\nuti_T$};
	\node at (5.8,0.35) {$\Gamma_2$};
	\node at (-5.8,0.35) {$\Gamma_0$};
	\draw[very thick, black,yshift=0pt,
	decoration={ markings,
		mark=at position 0.2 with {\arrow{latex}},
		mark=at position 0.9 with {\arrow{latex}}},
	postaction={decorate}]
	(6,-1) -- (1.5,-1)  arc (90:-90:-0.5) -- (1.8,0) arc (180:0:0.2) -- (3.8,0) arc (180:0:0.2) -- (6,0);
	
	\draw[dashed, line width = 1pt,yshift=0pt,
	decoration={ markings,
		mark=at position 0.1 with {\arrow{latex}}},
	postaction={decorate}]
	(-6,0) -- (-4.2,0) arc (-180:0:0.2) -- (-2.2,0)  arc (-180:0:0.2)  -- (1.5,0);
	
	\draw[ thick, ->] (-1,-0.25) arc (180:320:1);
	\end{tikzpicture}
	\caption{Integration contour $\Gamma_2$ in the case $\noncr$. The curved arrow indicates the contour deformation from $\Gamma_0$ to $\Gamma_2$.}
	\label{C4:contour2}
\end{figure}

\begin{figure}
\centering
\begin{subfigure}[b]{0.45\textwidth}
	\begin{tikzpicture}[scale=0.9]
	\node at (0,0) {$\times$};
	\node at (0.35,0.35) {$0$};
	\node at (0,1.5) {$\times$}; % Pole
	\node at (0.4,1.5) {$\nuti_L$};
	\node at (2.5,0) {$\times$};
	\node at (2.5,-0.35) {$\nuti_T$};
	\node at (0,-1.5) {$\times$};
	\node at (-0.5,-1) {$-\nuti_L$};
	\node at (-2.5,0) {$\times$};
	\node at (-2.5,0.35) {$-\nuti_T$};
%	\node at (5.8,0.35) {$\Gamma_2^a$};
%	\node at (1,-4) {$\Gamma_2^b$};
%	\node at (-5.8,0.35) {$(\Gamma_0)$};
	
	\draw[very thick, black,yshift=0pt,
	decoration={ markings,  % This schema allows for fine-tuning the positions of arrows 
		mark=at position 0.2 with {\arrow{latex}},
		mark=at position 0.9 with {\arrow{latex}}},
	postaction={decorate}]
	(2.85,-1) -- (2.3,-1)  arc (90:-90:-0.5)  -- (2.3,0) arc (180:0:0.2) -- (4,0);
	
	\draw[very thick, black,yshift=0pt,
	decoration={ markings,  % This schema allows for fine-tuning the positions of arrows 
		mark=at position 0.2 with {\arrow{latex}},
		mark=at position 0.9 with {\arrow{latex}}},
	postaction={decorate}]
	(-0.5,-4) -- (-0.5,-1.7) arc (0:-180:-0.5) -- (0.5,-3);
	
	\draw[very thick, black,yshift=0pt,
	decoration={ markings,mark=at position 0.5 with {\arrow{latex}}},
	postaction={decorate}]
	(0.5,-3) arc (-90:-9.5:2.4);
	
	\draw[dashed, line width =1pt, yshift=0pt,
	decoration={ markings,mark=at position 1 with {\arrow{>}}},
	postaction={decorate}]
	(0,-4)--(0,2.5);
	
	\draw[ thick, ->] (-2,-0.5) arc(0:90:-1);
	
%	\draw[dashed, line width = 1pt,yshift=0pt,
%	decoration={ markings,mark=at position 0.1 with {\arrow{latex}}},
%	postaction={decorate}]
%	(-6,0) -- (-3.2,0) arc (-180:0:0.2) -- (2.8,0)arc (180:0:0.2);

\draw[dashed, line width = 1pt,yshift=0pt]
	(-4,0) -- (-2.7,0) arc (-180:0:0.2) -- (2.3,0);
	
	\draw[ thick, ->] (2.5,-2.5)--(3.2,-3.2);
	
	\end{tikzpicture}
	\caption{Intermediate contour $\Gamma_2$. The arrows show the direction of the deformation.}
	\label{gamma2crinter}
\end{subfigure}
~
\begin{subfigure}[b]{0.45\textwidth}
	\begin{tikzpicture}[scale=0.9]
	\node at (0,0) {$\times$};
	\node at (0.35,0.35) {$0$};
	\node at (0,1.5) {$\times$}; % Pole
	\node at (0.4,1.5) {$\nuti_L$};
	\node at (2.5,0) {$\times$};
	\node at (2.5,-0.35) {$\nuti_T$};
	\node at (0,-1.5) {$\times$};
	\node at (-0.5,-1) {$-\nuti_L$};
	\node at (-2.5,0) {$\times$};
	\node at (-2.5,0.35) {$-\nuti_T$};
	\node at (3.8,0.4) {$\Gamma_2^a$};
	\node at (1,-3.8) {$\Gamma_2^b$};
%	\node at (-4.8,0.35) {$(\Gamma_0)$};
	
	\draw[very thick, black,yshift=0pt,
	decoration={ markings,  
		mark=at position 0.2 with {\arrow{latex}},
		mark=at position 0.9 with {\arrow{latex}}},
	postaction={decorate}]
	(4,-1) -- (2.3,-1)  arc (90:-90:-0.5)  -- (2.3,0) arc (180:0:0.2) -- (4,0);
	
	\draw[very thick, black,yshift=0pt,
	decoration={ markings, 
		mark=at position 0.2 with {\arrow{latex}},
		mark=at position 0.9 with {\arrow{latex}}},
	postaction={decorate}]
	(-0.5,-4) -- (-0.5,-1.7) arc (0:-180:-0.5) -- (0.5,-4);
	
	\draw[dashed, line width =1pt, yshift=0pt,
	decoration={ markings,mark=at position 1 with {\arrow{>}}},
	postaction={decorate}]
	(0,-4)--(0,2.5);
	
%	\draw[dashed, line width = 1pt,yshift=0pt,
%	decoration={ markings,mark=at position 0.1 with {\arrow{latex}}},
%	postaction={decorate}]
%	(-6,0) -- (-3.2,0) arc (-180:0:0.2) -- (2.8,0)arc (180:0:0.2);

	\draw[dashed, line width = 1pt,yshift=0pt]
	(-4,0) -- (-2.7,0) arc (-180:0:0.2) -- (2.3,0);
	
	\end{tikzpicture}
	\caption{Final contour $\Gamma_2=\Gamma_2^a\cup\Gamma_2^b$}
	\label{gamma2crfinal}
\end{subfigure}
\caption{Deformation of contour $\Gamma_0$ onto contour $\Gamma_2=\Gamma_2^a\cup\Gamma_2^b$ in the case $\crit$.}
\label{C4:contour2crit}
\end{figure}

The next step is to define the inverse translation operator $T_*^{-1}:\Omega_*^-\rightarrow\mathbb{C}, *=L,T$ :
\begin{equation}
T_*^{-1}(\xi=\nu_*\cos\theta)=\xi \cos \phiti-\zeta_*(\xi)\sin\phiti=\nu_*\cos(\theta-\tilde{\varphi}).
\end{equation}
$\cos\theta$ is well defined for $0\leq\rm Re \theta\leq \pi$, therefore this operator is defined on subspace $\Omega_*^-$, visible on Fig.~\ref{C4:dOmegamoins} and defined as 
\begin{equation}
\Omega_*^-=\{ \xi \in \mathbb{C}, \; \xi=\tilde{\nu}_* \cos \theta,  \tilde{\varphi}\leq\mbox{Re}(\theta)\leq\pi \}
\end{equation}
Using these definitions, contour $\Gamma_0$ in operator $\mathbf{TM}$ is deformed into contour $\partial \Omega_*^-$, visible on Fig.~\ref{C4:dOmegamoins}. In the case where $\noncr$, the contours are represented in Fig.~\ref{C4:Omegamoinsnoncr}, and the deformation from $\Gamma_0$ to $\partial \Omega_*^-$ (represented by the arrows on the figure) only spans the  bottom half of domain $\Omega_*^-$. In the case where $\crit$, the contour $\partial \Omega_L^-$ is represented in Fig.~\ref{C4:OmegamoinsL}, and the deformation from $\Gamma_0$ to $\partial \Omega_L^-$ (represented by the arrows on the figure) only spans the part of domain $\Omega_L^-$ which is contained in the upper half of the complex plane. In both cases, the poles $\zeta$ of the integrand are $\zeta=T_*^{-1}(\xi), \rm Im \xi <0$. These poles are crossed if and only if $\xi \in \Omega_*^-$ and Im$\xi<0$, where domain $\Omega_*^-$ is represented in grey on Figs.~\ref{C4:Omegamoinsnoncr} and \ref{C4:OmegamoinsL}. Their contribution is determined thanks to Cauchy's residue theorem :
\begin{equation}
\int_{\Gamma_0} \textbf{TM}(\xi,\zeta)X_j(\zeta)\, d\zeta = \sum_{*=L,T}\int_{\partial \Omega_*^-}  \dfrac{\textbf{tm}_*(\zeta)}{\xi-T_*(\zeta)}.X_j(\zeta)\, d\zeta+ \mathbf{M}_*(\xi).X_j(T^{-1}_*(\xi))\textbf{1}_{\Omega_*^-}(\xi),
\label{C4:TM2}
\end{equation}
where $\textbf{1}_{\Omega_*^-}(\xi)=1$ when $\xi \in \Omega_*^-$ and $\rm Im \xi<0$ and $\textbf{1}_{\Omega_*^-}(\xi)=0$ elsewhere and
\begin{equation}
\mathbf{M}_*(\xi=\nuti_*\cos\theta)=-\frac{\sin(\theta-\tilde{\varphi})}{\sin\theta} \textbf{tm}_*(T_*^{-1}(\xi))
\end{equation}
 
%Let us now determine in which cases these poles are crossed by the contour deformation.
%
%When $\noncr$, poles $\zeta=T_*^{-1}(\xi)$, Im$\xi<0$ are crossed if $\xi \in \Omega_*^-$. Their contribution is determined thanks to Cauchy's residue theorem :
%\begin{equation}
%\int_{\Gamma_0} \textbf{TM}(\xi,\zeta)X_j(\zeta)\, d\zeta = \sum_{*=L,T}\int_{\partial \Omega_*^-}  \dfrac{\textbf{tm}_*(\zeta)}{\xi-T_*(\zeta)}.X_j(\zeta)\, d\zeta+ \mathbf{M}_*(\xi).X_j(T^{-1}_*(\xi))\textbf{1}_{\Omega_*^-}(\xi),
%\label{C4:TM2}
%\end{equation}
%where $\textbf{1}_{\Omega_*^-}(\xi)=1$ when $\xi \in \Omega_*^-$ and $\rm Im \xi<0$ and $\textbf{1}_{\Omega_*^-}(\xi)=0$ elsewhere and
%\begin{equation}
%\mathbf{M}_*(\xi=\nuti_*\cos\theta)=-\frac{\sin(\theta-\tilde{\varphi})}{\sin\theta} \textbf{tm}_*(T_*^{-1}(\xi))
%\end{equation}
%When $\crit$, the contour deformation only spans a part of the upper half of the complex plane, as shown in Fig.~\ref{C4:OmegamoinsL}. Since the poles $\zeta=T_L^{-1}(\xi)$ have a negative imaginary part, no poles are crossed during the deformation of contour $\Gamma_0$ into contour $\partial \Omega_L^-$, visible Fig.~\ref{C4:OmegamoinsL}. Cauchy's residue theorem then yields :
%\begin{equation}
%\int_{\Gamma_0} \textbf{TM}(\xi,\zeta)X_j(\zeta)\, d\zeta =\mathbf{M}_T(\xi).X_j(T^{-1}_T(\xi))\textbf{1}_{\Omega_T^-}(\xi),+ \sum_{*=L,T}\int_{\partial \Omega_*^-}  \textbf{TM}(\xi,\zeta)X_j(\zeta)\, d\zeta
%\label{C4:TM2cr}
%\end{equation}

\begin{figure}[ht]%
\centering
\begin{subfigure}[b]{0.45\textwidth}
	\begin{tikzpicture}
	% Filling Omega_*^-
	\fill [color=gray!20]
	(-3,{-sinh(1.7)})
	-- plot [domain= -1.7:1.7] ({cosh(\x)},{sinh(\x)})
	-- (-3,{sinh(1.7)});
	
%	\fill [color=gray!20]
%	({cosh(1.7)},{-sinh(1.7)})
%	-- plot [domain= 0:-3] ({\x},{-sinh(1.7)})
%	-- (-3,0)
%	-- cycle;
	
%	\fill[color=white] (-2,0) circle (0.25);
	
	\draw[dashed, ->,>=stealth] (-3,0)  -- (-2.25,0) arc(-180:0:0.25)--(1.75,0) arc (180:0:0.25)--(4,0) ;%node[above]{$\xi_1$};
	\draw[dashed, ->,>=stealth](0,{-sinh(1.7)})--(0,{sinh(1.7)}) ;%node[left]{$\xi_2$};
	\node at (2,0) { $\times$}; 
%	\node at (0,1.5) { $\times$}; 
	\node at (2,-0.5) {$\nuti_T$};
%	\node at (0.5,1.5) {$\nuti_L$};
	\node at (-2,0) { $\times$};
%	\node at (0,-1.5) { $\times$}; 
	\node at (-2.1,0.35) {$-\nuti_T$};
%	\node at (-0.5,-1.5) {$-\nuti_L$};
	
	% Hyperbola (contour  partial_Omega_0 )
	\draw[black, thick,decoration={ markings,  % This schema allows for fine-tuning the positions of arrows 
		mark=at position 0.2 with {\arrow{latex}},
		mark=at position 0.9 with {\arrow{latex}}},
	postaction={decorate}][domain=-1.7:1.7] plot({cosh(\x)}, {sinh(\x)});
	
	\node at (3.5,-2.5) {$\partial \Omega_*^-$};
	\draw[thin, ->,>=stealth](1.7,0.45) -- (1.1, 0.06);
	\node at (2.2, 0.75) { $\nuti_*\cos \tilde{\varphi}$};
	
	\draw[black,->] (3.5,0.2) arc (0:50:1);
	
	\draw[black,->] (-0.5,-0.2) arc (0:70:-1);
	
	\node[align=left] at (0.1,-1.8) {recursive \\ evaluation};
	
	\node[align=left] at (3,-1.3) {direct \\ evaluation};

	% Espace Omega_0
\fill [color=white] (-2,-1.5) circle (0.5);
\draw[thick] (-2,-1.5) circle (0.5);
\node at (-2,-1.5) {$\Omega_*^-$};
	\end{tikzpicture}
	\caption{Domain $\Omega_*^-$ and contour $\partial\Omega_*^-$ in the case $\noncr$. The arrows show the deformation from contour $\Gamma_0$ to contour $\partial \Omega_*^-$.}
	\label{C4:Omegamoinsnoncr}
	\end{subfigure}
	\hfill
	\begin{subfigure}[b]{0.45\textwidth}
	    \begin{tikzpicture}
	    	% Filling Omega_*^-
	\fill [color=gray!20]
	({-sinh(2)},-2)
	-- plot [domain= -2:2] ({sinh(\x)},{cosh(\x)})
	--({sinh(2)},-2);

	\draw[dashed, ->,>=stealth] (-3.6269,0)  -- (-2.25,0) arc(-180:0:0.25)--(1.75,0) arc (180:0:0.25)--(3.6269,0);% node[above]{$\xi_1$};
	
	\draw[dashed, ->,>=stealth](0,-2)--(0,{sinh(2)});% node[left]{$\xi_2$};
	\node at (2,0) { $\times$}; 
	\node at (0,1.5) { $\times$}; 
	\node at (2,-0.5) {$\nuti_T$};
	\node at (0.5,1.5) {$\nuti_L$};
	\node at (-2,0) { $\times$};
	\node at (0,-1.5) { $\times$}; 
	\node at (-2.1,0.35) {$-\nuti_T$};
	\node at (-0.5,-1.5) {$-\nuti_L$};
	
	% Hyperbola (contour  partial_Omega_0 )
	\draw[black, thick,decoration={ markings,
		mark=at position 0.2 with {\arrow{latex}},
		mark=at position 0.9 with {\arrow{latex}}},
	postaction={decorate}][domain=-2:2] plot({sinh(\x)},{cosh(\x)});
	
	\node at (2,3) {$\partial \Omega_L^-$};
	\draw[thin, ->,>=stealth](1,0.7) -- (0.06,0.9);
	\node at (1.8, 0.65) { $\nuti_L\cos \tilde{\varphi}$};

	% Espace Omega_0
\fill [color=white] (-2,-1.3) circle (0.5);
\draw[thick] (-2,-1.3) circle (0.5);
\node at (-2,-1.3) {$\Omega_L^-$};

	\draw[black,->] (3.2,0.2) arc (0:40:1);
	
	\draw[black,->] (-3.2,0.2) arc (0:-40:-1);
	
	\node[align=left] at (2,-1.2) {recursive \\ evaluation};
	
	\node[align=left] at (-1.1,3) {direct \\ evaluation};

	    \end{tikzpicture}
	    \caption{Domain $\Omega_L^-$ and contour $\partial\Omega_L^-$ in the case $\crit$. The arrows show the deformation from contour $\Gamma_0$ to contour $\partial \Omega_L^-$.}
    \label{C4:OmegamoinsL}
	\end{subfigure}
\caption{Domains $\Omega_*^-$ and contours $\partial \Omega_*^-$ in cases $\noncr$ and $\crit$.}
\label{C4:dOmegamoins}
\end{figure}

The recursive system of functional equations solved by the regular part is obtained by substituting \eqref{C4:DM2} and \eqref{C4:TM2} into \eqref{C4:regparteqn}:
\begin{equation}
\left\{
\begin{matrix}
X_1(\xi) =g_1(\xi)-\textbf{dm}^{-1}(\xi).\underset{*=L,T}{\sum} \mathbf{M}_*(\xi).X_2(T_*^{-1}(\xi))\textbf{1}_{\Omega_*^-}(\xi) \\
X_2(\xi) =g_2(\xi)-\textbf{dm}^{-1}(\xi).\underset{*=L,T}{\sum} \mathbf{M}_*(\xi).X_1(T_*^{-1}(\xi))\textbf{1}_{\Omega_*^-}(\xi)
\end{matrix}
\right.,
\label{C4:recur}
\end{equation}
where , for $j=1,2$
\begin{equation}
g_j(\xi)=\textbf{dm}^{-1}(\xi)\left( u_j(\xi)- \int_{\Gamma_2}  \textbf{DM}(\xi,\zeta)X_j(\zeta)\, d\zeta- \int_{\partial \Omega_*^-}  \textbf{TM}(\xi,\zeta)X_{3-j}(\zeta)\, d\zeta \right) 
\label{C4:g1g2}
\end{equation}
The same consideration as those made in \ref{C3:propag} lead to an expression of functions $g_j$ using operators $\mathbb{D}(\cdot,\cdot)$ and $\mathbb{T}(\cdot,\cdot)$ :
\begin{equation}
\textbf{dm}(\xi).g_j(\xi)=u_j(\xi)-\sum_{k=1}^{2N} \sqrt{\frac{a_k}{\pi}}\left( \mathbb{ND}(a_k,\xi).\tilde{X}_j^k+\mathbb{NT}(a_k,\xi).\tilde{X}_{3-j}^k \right) ,
\label{C4:gjfinal}
\end{equation}
where
\begin{equation}
\mathbb{ND}(a,b)=\frac{1}{2\pi}\mathbb{D}(a,b)-\frac{\textbf{dm}(b)}{a+b}
\label{C4:defND}
\end{equation}
and
\begin{equation}
\mathbb{NT}(a,b)=\frac{1}{2i\pi}\mathbb{T}(a,b)-\sum_{*=L,T}\frac{\mathbf{M}_*(b)}{T^{-1}_*(b)+a} .
\end{equation}

In system \eqref{C4:recur}, the value of the regular part of the spectral function in domain $\Omega_*^-$, visible Fig.~\ref{C4:dOmegamoins}, is expressed using its value in the domain $\xi \notin \Omega_*^-$, where the numerical approximation \eqref{C4:Xj} is valid. To do so, functions $g_j,\, j=1,2$ are evaluated numerically using \eqref{C4:gjfinal}. The accuracy of the numerical evaluation in domain $\xi \notin \Omega_*^-$ is therefore propagated to domain $\Omega_*^-$. 

This concludes the semi-analytical computation of the spectral functions. The L, TH and TV diffraction coefficients can now be computed using \eqref{C4:Dbeta}. Numerical testing is presented in the following. 

\section{Numerical Tests}
The spectral functions are evaluated numerically using the semi-analytical scheme described in the previous sections. This is achieved by, first, computing the poles and residues of the spectral functions analytically using the recursive algorithm described in subsection \ref{C4:singpart}. Then, the regular parts of the spectral functions are approached numerically by solving \eqref{C4:regparteqn} thanks to the Galerkin collocation method described in subsection \ref{C4:regpart}. In the case where $\noncr$, the Galerkin parameters are set to:
\begin{equation}
a_k=1.001+0.05e^{k\frac{\log 10}{4}}-1, \hspace{3em} b_k=a_k-0.1i, \hspace{3em} 1\leq k\leq20
\end{equation}
And in the case where $\crit$, meaning for the case of an incident transversal wave with $|\delta_{inc}|>\delta_c$ (with $\delta_c \approx 33^o$ in steel), the Galerkin parameters are set to
\begin{equation}
\begin{matrix}
a_k=1.001+0.05e^{k\frac{\log 10}{4}}-1, & b_k=a_k-0.1i, & 1\leq k\leq10\\
a_k=-i\lbrack 1.001+0.05e^{(k-10)\frac{\log 10}{4}}-1\rbrack, & b_k=a_k+0.1, & 11\leq k\leq20
\end{matrix}
\end{equation}

Finally, the solution is rendered accurate in the entire complex domain by applying the recursive procedure called the propagation of the solution described in subsection \ref{C4:propag}.

Following these steps, the diffraction coefficients have been computed using \eqref{C4:Dbeta} and tested numerically.

\subsection{Comparison to the 2D code}
The first test on the 3D code is to check that when $\delta_{\alpha}=0$, the results obtained using the 3D code are the same as those obtained using the 2D code presented and validated numerically (see section \ref{C3:numval}) and experimentally (see section \ref{C3:expval}) in the previous chapter. This has been checked for the theoretical computations and must also be verified numerically.

The spectral functions are evaluated at $\xi=\nuti_L\cos\theta -i10^{-6}$ (a small negative imaginary part is added to ensure that the recursive equations \eqref{C4:recur} are valid) every $0,5^o$ for $0\leq\theta\leq \varphi$ and at $\delta_{\alpha} =0^o$,  using the 3D code. The L and TH diffraction coefficients are computed using \eqref{C4:Dbeta}, for an elastic wave propagating in a steel wedge ($c_L=5700m.s^{-1}, \, \, c_T=3200m.s^{-1}$). For the 3D problem, TH waves defined by \eqref{C4:ivec} correspond to the T waves of the 2D problem. The results are compared to the diffraction coefficients, given by \eqref{DL} and \eqref{DT}, obtained using the 2D elastic code presented in the previous chapter.

Figs.~\ref{C4:14070} and \ref{C4:25065} show the absolute value of the diffraction coefficients obtained using the 2D and 3D \acrshort{sf} codes for a wedge of angle $\varphi=140^o$ illuminated by a wave incident with an angle $\theta_{inc}=70^o$ and for a wedge of angle $\varphi=250^o$ illuminated by a wave incident with an angle $\theta_{inc}=65^o$.

\begin{figure}[h]
\centering
    \begin{subfigure}[b]{0.49\textwidth}
        \includegraphics[width=\textwidth]{images/chapter4/XpropL_140_70_0_L.png}
        \caption{Diffracted and incident L waves.}
        \label{C4:DLL_14070}
    \end{subfigure}  
    \begin{subfigure}[b]{0.49\textwidth}
        \includegraphics[width=\textwidth]{images/chapter4/XpropTH_140_70_0_L.png}
        \caption{Diffracted T wave and incident L wave.}
        \label{C4:DLT_14070}
     \end{subfigure}   
     \begin{subfigure}[b]{0.49\textwidth}
        \includegraphics[width=\textwidth]{images/chapter4/XpropL_140_70_0_TH.png}
        \caption{Diffracted L wave and incident T wave.}
        \label{C4:DTL_14070}
    \end{subfigure}
    \begin{subfigure}[b]{0.49\textwidth}
        \includegraphics[width=\textwidth]{images/chapter4/XpropTH_140_70_0_TH.png}
        \caption{Diffracted and incident T waves.}
        \label{C4:DTT_14070}
     \end{subfigure}
     \caption{Diffraction coefficients for $\varphi=140^o, \theta_{inc}=70^o$}
     \label{C4:14070}
\end{figure} 

\begin{figure}[h]
\centering
    \begin{subfigure}[b]{0.49\textwidth}
        \includegraphics[width=\textwidth]{images/chapter4/XpropL_250_65_0_L.png}
        \caption{Diffracted and incident L waves.}
        \label{C4:DLL_25065}
    \end{subfigure}  
    \begin{subfigure}[b]{0.49\textwidth}
        \includegraphics[width=\textwidth]{images/chapter4/XpropTH_250_65_0_L.png}
        \caption{Diffracted T wave and incident L wave.}
        \label{C4:DLT_25065}
     \end{subfigure}   
     \begin{subfigure}[b]{0.49\textwidth}
        \includegraphics[width=\textwidth]{images/chapter4/XpropL_250_65_0_TH.png}
        \caption{Diffracted L wave and incident T wave.}
        \label{C4:DTL_25065}
    \end{subfigure}
    \begin{subfigure}[b]{0.49\textwidth}
        \includegraphics[width=\textwidth]{images/chapter4/XpropTH_250_65_0_TH.png}
        \caption{Diffracted and incident T waves.}
        \label{C4:DTT_25065}
     \end{subfigure}
     \caption{Diffraction coefficients for $\varphi=250^o, \theta_{inc}=65^o$}
     \label{C4:25065}
\end{figure}

In Figs.~\ref{C4:DLL_14070}-\ref{C4:DTL_14070}-\ref{C4:DLL_25065}-\ref{C4:DTL_25065} and Figs.~\ref{C4:DLT_14070}-\ref{C4:DTT_14070}-\ref{C4:DLT_25065}-\ref{C4:DTT_25065}, representing the L and T diffraction coefficients respectively, the thick blue line represents the results obtained using the 2D code and the dashed lines (red and green respectively) represent the results obtained using the 3D code.

In all of these figures, and in all other tested configurations, the 2D and 3D plots are perfectly overlapping. When $\delta_{\alpha}=0^o$, the 3D code yields exactly the same results as the 2D code, which is in accord with the theoretical computations. This validates the computation of the "2D terms" (meaning the terms that are not canceled by setting $\delta_{\alpha}=0^o$) of the 3D code. The following numerical test, comparison of the 3D elastic code to Sommerfeld's analytical expression for an acoustic wave, validates a different set of terms (the ones that are purely 3D and longitudinal) computed by the spectral functions method.

\subsection{Acoustic limit}
In the second chapter of this manuscript, we have seen that Sommerfeld \cite{Sommerfeld} provides an analytical expression for the \acrshort{gtd} diffraction coefficient in the case of an acoustic wave incident on a wedge with Dirichlet or Neumann boundaries. This expression is still valid for 3D incidences, and the expression is provided by Keller \cite{GTD}. In the case of a wedge with Dirichlet boundaries, we have :
\begin{equation}
\begin{split}
v^{ac,diff}(r,\theta)=D^{Dir}(\theta)\dfrac{e^{-ik_0r}}{\sqrt{k_0r\cos\delta_{\alpha}}}
\end{split}
\end{equation}
where $v^{ac,diff}$ is the acoustic diffracted field, $k_0$ is the acoustic wave number and $D^{Dir}$, is given by \eqref{GTDCoeff_Dir}. For an acoustic wave, the diffraction coefficient does not depend on the incident skew angle $\delta_{\alpha}$. The dependency of the diffracted field with respect to this parameter is fully contained in the term $(k_0r\cos\delta_{\alpha})^{-1/2}$.

The case of an acoustic wave incident on a wedge with Dirichlet boundary conditions can be mimicked using the elastic code. By setting $c_L=1$ and $c_T \rightarrow 0$ and considering incident L waves, the L diffraction coefficient should behave like the diffraction coefficient of an acoustic wave. 

In the 3D elastic code, the wave velocities are set to $c_L=1 m.s^{-1}$ and $c_T=10^{-7} m.s^{-1}$ and the incident wave is longitudinal. The spectral functions are evaluated at $\xi=\nuti_L\cos\theta -i10^{-6}$ every $0,5^o$ for $0\leq\theta\leq \varphi$ and for $-90^o\leq \delta_{\alpha} \leq 90^o$ and the L diffraction coefficient is deduced using \eqref{C4:Dbeta}. The results are compared to the analytical expression of the Sommerfeld diffraction coefficients for a wedge with Dirichlet boundary conditions.

Figs.~\ref{C4:ac16040} and \ref{C4:ac280240} show the absolute value of the diffraction coefficient obtained using the \acrshort{sf} code (Figs.~\ref{C4:acSF16040} and \ref{C4:acSF280240}) and with Sommerfeld's analytical expression (Figs.~\ref{C4:Som16040} and \ref{C4:Som280240}) for a wave incident with an angle $\theta_{inc}=40^o$ on a wedge of angle $\varphi=160^o$ and for $\theta_{inc}=240^o$ and $\varphi=280^o$, respectively. 

\begin{figure}[h]
\centering
\begin{subfigure}[b]{0.49\textwidth}
        \includegraphics[width=\textwidth]{images/chapter4/Xprop_160_40.png}
        \caption{\acrshort{sf} diffraction coefficient}
        \label{C4:acSF16040}
    \end{subfigure}
\begin{subfigure}[b]{0.49\textwidth}
        \includegraphics[width=\textwidth]{images/chapter4/Sommerfeld_160_40.png}
        \caption{Sommerfeld diffraction coefficient}
        \label{C4:Som16040}
    \end{subfigure}
\caption{Absolute value of the diffraction coefficient computed with the spectral functions and with the Sommerfeld method for a wave incident with an angle $\theta_{inc}=40^o$ on a wedge of angle $\varphi=160^o$}
\label{C4:ac16040}
\end{figure}

In both cases, the diffraction coefficients are computed for various incident skew angles $\delta_{\alpha}$ to check that the \acrshort{sf} diffraction coefficient is independent of this parameter, as it should be in the acoustic case. The horizontal axis corresponds to the observation angle $\theta$, the vertical axis corresponds to the incident skew angle $\delta_{\alpha}$ and the magnitude of the diffraction coefficients is represented in color in the $(\theta,\delta_{\alpha})$ plane. For both wedges, the figures representing the \acrshort{sf} diffraction coefficients and those representing the Sommerfeld diffraction coefficients appear to be identical and are invariant by vertical translation (meaning that the coefficients do not depend on the angle $\delta_{\alpha}$). The diffraction coefficients can therefore be plotted for a single skew angle, without loss of generality. This is also the case for the angular phases of the diffraction coefficients, and their plots in the $(\theta,\delta_{\alpha})$ plane are not reproduced here.

\begin{figure}
\centering
\begin{subfigure}[b]{0.49\textwidth}
        \includegraphics[width=\textwidth]{images/chapter4/Xprop_280_239.png}
        \caption{\acrshort{sf} diffraction coefficient}
        \label{C4:acSF280240}
    \end{subfigure}
\begin{subfigure}[b]{0.49\textwidth}
        \includegraphics[width=\textwidth]{images/chapter4/Sommerfeld_280_239.png}
        \caption{Sommerfeld diffraction coefficient}
        \label{C4:Som280240}
    \end{subfigure}
\caption{Absolute value of the diffraction coefficient computed with the spectral functions and with the Sommerfeld method for a wave incident with an angle $\theta_{inc}=240^o$ on a wedge of angle $\varphi=280^o$}
\label{C4:ac280240}
\end{figure}

\begin{figure}
\centering
\begin{subfigure}[b]{0.49\textwidth}
        \includegraphics[width=\textwidth]{images/chapter4/D_160_40_0.png}
        \caption{$\varphi=160^o,  \, \, \theta_{inc}=40^o$}
        \label{C4:compac16040}
    \end{subfigure}
\begin{subfigure}[b]{0.49\textwidth}
        \includegraphics[width=\textwidth]{images/chapter4/D_280_239_0.png}
        \caption{$\varphi=280^o, \, \, \theta_{inc}=240^o$}
        \label{C4:compac280240}
    \end{subfigure}
\caption{Absolute value of the diffraction coefficient computed with the spectral functions and with the Sommerfeld method for $\delta_{inc}=0^o$.}
\label{C4:compac}
\end{figure}

\begin{figure}
\centering
\begin{subfigure}[b]{0.49\textwidth}
        \includegraphics[width=\textwidth]{images/chapter4/ArgD_160_40_0.png}
        \caption{$\varphi=160^o,  \, \, \theta_{inc}=40^o$}
        \label{C4:argac16040}
    \end{subfigure}
\begin{subfigure}[b]{0.49\textwidth}
        \includegraphics[width=\textwidth]{images/chapter4/ArgD_280_239_0.png}
        \caption{$\varphi=280^o, \, \, \theta_{inc}=240^o$}
        \label{C4:argac280240}
    \end{subfigure}
\caption{Angular phase of the diffraction coefficient computed with the spectral functions and with the Sommerfeld method for $\delta_{inc}=0^o$.}
\label{C4:Argcompac}
\end{figure}

Figs.~\ref{C4:compac} and \ref{C4:Argcompac} respectively show the absolute value and the angular phase of the diffraction coefficient, plotted for $\delta_{\alpha}=0^o$ for a wave incident with an angle $\theta_{\alpha}=40^o$ on a wedge of angle $\varphi=160^o$ (see Fig.~\ref{C4:compac16040}) and for a wave incident with an angle $\theta_{\alpha}=240^o$ on a wedge of angle $\varphi=280^o$ (see Fig.~\ref{C4:compac280240}). In all four figures, the thick blue line is the solution computed using Sommerfeld's analytical expression and the dashed red line is the result obtained using acoustic limit of the 3D \acrshort{sf} code. Both lines are perfectly overlapping, except for some discrepancies in the angular phase, for observation directions near the wedge faces.

The "acoustic limit" of the 3D elastic code is thus validated for wedge angles lower and higher than $\pi$. This shows that the terms appearing in the evaluation of the spectral functions that depend on $\nuti_T$ tend to $0$ when transversal wave velocity tends to $0$ and that the terms that depend on $\nuti_L$ are computed correctly. 

\subsection{Verification of the regular part for an infinite plane}
In the case where $\varphi=\pi$, the wedge degenerates into an infinite plane and there is no edge diffracted wave. The regular part of the spectral functions, which is determined by system \eqref{C4:systmat} and is the part of the solution corresponding to the diffraction phenomena, vanishes and we should have, for $j=1,2$ :
\begin{equation}
||\mathbb{U}_j||=0 \hspace{1em} %\rm and \hspace{1em} ||\mathbb{X}_j||=0,
\end{equation}
where $\mathbb{U}_j$ is the right-hand side of system \eqref{C4:systmat} and is given by \eqref{C4:vecXUj}. Verifying that this is the case provides a check on the lengthy computations of the explicit expressions of operators $\mathbb{D}(\cdot,\cdot)$ and $\mathbb{T}(\cdot,\cdot)$. According to \eqref{C4:vecXUj}, the value of $||\mathbb{U}_j||$ does not depend on the observation angle $\theta$, therefore in our tests, only the skew incidence angle $\delta_{\alpha}$ varies.

Fig.~\ref{Resultat_3D:U1} shows $||\mathbb{U}_j||, \, \, j=1,2$ for incident L (see Fig.~\ref{Resultat_3D:U1L}), TH (see Fig.~\ref{Resultat_3D:U1TH}) and TV (see Fig.~\ref{Resultat_3D:U1TV}) waves with an angle $\theta_{inc}=50^o$ on an infinite plane. The thick blue line represents $||\mathbb{U}_1||$ and the dashed red line represents $||\mathbb{U}_2||$. In the case of an incident L wave, as expected, $||\mathbb{U}_1||$ and $||\mathbb{U}_2||$ are very small (of the order of the numerical computation error). For incident T waves, however, when the incident skew angle is higher than the critical angle, $||\mathbb{U}_1||$ and $||\mathbb{U}_2||$ are suddenly very large, rather than quasi null. Because this is only the case when $\crit$, we believe that this is not due to a miscalculation of operators $\mathbb{D}(\cdot,\cdot)$ or $\mathbb{T}(\cdot,\cdot)$ (the corresponding computations are detailed in appendices \ref{matD} and \ref{matT}), which would have produced errors visible in cases where $\noncr$. 

\begin{figure}[h]
\centering
\begin{subfigure}[b]{0.32\textwidth}
        \includegraphics[width=\textwidth]{images/chapter4/Resultats_3D/U1L_180_50.png}
        \caption{Incident L wave}
        \label{Resultat_3D:U1L}
    \end{subfigure}
\begin{subfigure}[b]{0.32\textwidth}
        \includegraphics[width=\textwidth]{images/chapter4/Resultats_3D/U1TH_180_50.png}
        \caption{Incident TH wave}
        \label{Resultat_3D:U1TH}
    \end{subfigure}
   \begin{subfigure}[b]{0.32\textwidth}
        \includegraphics[width=\textwidth]{images/chapter4/Resultats_3D/U1TV_180_50.png}
        \caption{Incident TV wave}
        \label{Resultat_3D:U1TV}
    \end{subfigure} 
\caption{$||\mathbb{U}_j||, \, \, j=1,2$ for $\varphi=180^o$ and $\theta_{inc}=50^o$}
\label{Resultat_3D:U1}
\end{figure}

Nonetheless, when $\crit$, the code developed according to the theory described in this chapter produces diverging results (see for example Fig.~\ref{Resultat_3D:D} showing diffraction coefficients computed with the standard theory of the current chapter). We are not sure what the cause of this error is, and additional work is necessary in order to solve this problem.

\begin{figure}[h]
\centering
\begin{subfigure}[b]{0.32\textwidth}
        \includegraphics[width=\textwidth]{images/chapter4/Resultats_3D/XpropL_140_70_TH.png}
        \caption{Diffracted L wave}
        \label{Resultat_3D:DL}
    \end{subfigure}
\begin{subfigure}[b]{0.32\textwidth}
        \includegraphics[width=\textwidth]{images/chapter4/Resultats_3D/XpropTH_140_70_TH.png}
        \caption{Diffracted TH wave}
        \label{Resultat_3D:DTH}
    \end{subfigure}
   \begin{subfigure}[b]{0.32\textwidth}
        \includegraphics[width=\textwidth]{images/chapter4/Resultats_3D/XpropTV_140_70_TH.png}
        \caption{Diffracted TV wave}
        \label{Resultat_3D:DTV}
    \end{subfigure} 
\caption{Absolute value of the diffraction coefficient computed in a standard manner (i.e. without approximation \eqref{const_reg_approx}) for an incident TH wave on a wedge of angle $\varphi=140^o$ with $\theta_{inc}=70^o$}
\label{Resultat_3D:D}
\end{figure}

Fig.~\ref{Resultat_3D:D} shows the absolute value of the diffraction coefficients obtained using the SF code without approximation \eqref{const_reg_approx}. The L, TH and TV diffraction coefficients are computed using \eqref{C4:Dbeta} for a steel wedge ($c_L=5700m.s^{-1}$ and $c_T=3200m.s^{-1}$) of angle $\varphi=140^o$ illuminated by a TH wave with and angle $\theta_{inc}=70^o$. The horizontal axis corresponds to the observation angle $\theta$, the vertical axis corresponds to the incident skew angle $\delta_{\alpha}$ and the magnitude of the diffraction coefficient is represented in color in the $(\theta,\delta_{\alpha})$ plane. Fig.~\ref{Resultat_3D:DL} shows the L diffraction coefficient, Fig.~\ref{Resultat_3D:DTH} shows the TH diffraction coefficient and Fig.~\ref{Resultat_3D:DTV} shows the TV diffraction coefficient. It is clear from these last two figures that the diffraction coefficient abruptly diverges when $\crit$.

The regular part of the spectral functions diverges in cases where $\crit$ and additionnal work must be done to correct this problem. In the meantime, a new approximation has been proposed in order to obtain a non-diverging diffraction coefficient in these cases. This approximation and its effects are detailed in the following.

\subsection{Numerical approximation in the case $\crit$}

In the previous subsection, it has been made apparent that the regular part of the spectral functions is miscalculated in the case of an incident T wave with a skew angle higher than the critical angle. The cause of this has not yet been found. In the meantime, in order to obtain physically coherent results, the following approximation is applied (only for incident T waves) :
\begin{subequations}
\begin{equation}
\mathbb{D}(\cdot,\cdot)|_{\delta_{\beta}>\delta_C} \approx \mathbb{D}(\cdot,\cdot)|_{\delta_{\beta}=\delta_c-0.25^o}
\end{equation}
\begin{equation}
\mathbb{T}(\cdot,\cdot)|_{\delta_{\beta}>\delta_C} \approx \mathbb{T}(\cdot,\cdot)|_{\delta_{\beta}=\delta_c-0.25^o}
\end{equation}
\label{const_reg_approx}
\end{subequations}

Fig.~\ref{const_reg:U1} shows $||\mathbb{U}_j||, \, \, j=1,2$ obtained with approximations \eqref{const_reg_approx} for incident TH (see Fig.~\ref{const_reg:U1TH}) and TV (see Fig.~\ref{const_reg:U1TV}) waves. with an angle $\theta_{inc}=50^o$ on an infinite plane. The thick blue line represents $||\mathbb{U}_1||$ and the dashed red line represents $||\mathbb{U}_2||$. Using  approximation \eqref{const_reg_approx}, $||\mathbb{U}_1||$ and $||\mathbb{U}_2||$ now behave as expected, even when $\crit$ and are  of the order of the numerical computation error.

\begin{figure}[h]
\centering
\begin{subfigure}[b]{0.45\textwidth}
        \includegraphics[width=\textwidth]{images/chapter4/const_reg/U1TH_180_50.png}
        \caption{Incident TH wave}
        \label{const_reg:U1TH}
    \end{subfigure}
   \begin{subfigure}[b]{0.45\textwidth}
        \includegraphics[width=\textwidth]{images/chapter4/const_reg/U1TV_180_50.png}
        \caption{Incident TV wave}
        \label{const_reg:U1TV}
    \end{subfigure} 
\caption{$||\mathbb{U}_j||, \, \, j=1,2$ for $\varphi=180^o$ and $\theta_{inc}=50^o$}
\label{const_reg:U1}
\end{figure}

In order to illustrate the effect of approximation \eqref{const_reg_approx} in the case where $\crit$, we provide an example of the effect of this approximation on the resulting diffraction coefficient. To do so, the L, TH and TV diffraction coefficients are computed using \eqref{C4:Dbeta} for a steel wedge ($c_L=5700m.s^{-1}$ and $c_T=3200m.s^{-1}$) of angle $\varphi=140^o$ illuminated by a TH wave with and angle $\theta_{inc}=70^o$. The spectral functions are evaluated at $\xi=\nuti_{\beta}\cos\theta -i10^{-6}$ every $0,5^o$ for $0\leq\theta\leq \varphi$ and for $-90^o\leq \delta_{\alpha} \leq 90^o$.

Fig.~\ref{const_reg:D} shows the absolute value of the diffraction coefficients obtained using the SF code with approximation \eqref{const_reg_approx}. The horizontal axis corresponds to the observation angle $\theta$, the vertical axis corresponds to the incident skew angle $\delta_{\alpha}$ and the magnitude of the diffraction coefficient is represented in color in the $(\theta,\delta_{\alpha})$ plane. Fig.~\ref{const_reg:DTH} shows the TH diffraction coefficient and Fig.~\ref{const_reg:DTV} shows the TV diffraction coefficient. The diffraction coefficients visible in these two figures are no longer divergent when $\crit$ and their behaviour seems physically coherent. It can be noted that these coefficients seem to diverge when $\delta_{\alpha}$ approaches $\pm90^o$ (but not for $\delta_{\alpha}=\pm90^o$ exactly), meaning when the incidence grazes the wedge edge. In this case, the \acrshort{gtd} field diverges because of its proportionality to the factor $(\cos\delta_{\beta})^{-1/2}$, see \eqref{C4:coeffdiff} and another computation method should be considered.

\begin{figure}[h]
\centering
\begin{subfigure}[b]{0.45\textwidth}
        \includegraphics[width=\textwidth]{images/chapter4/const_reg/XpropTH_140_70_TH.png}
        \caption{Diffracted TH wave}
        \label{const_reg:DTH}
    \end{subfigure}
   \begin{subfigure}[b]{0.45\textwidth}
        \includegraphics[width=\textwidth]{images/chapter4/const_reg/XpropTV_140_70_TH.png}
        \caption{Diffracted TV wave}
        \label{const_reg:DTV}
    \end{subfigure} 
\caption{Absolute value of the diffraction coefficient computed with approximation \eqref{const_reg_approx} for an incident TH wave on a wedge of angle $\varphi=140^o$ with $\theta_{inc}=70^o$}
\label{const_reg:D}
\end{figure}

The diffraction coefficients have been computed using the spectral functions method for a steel wedge of angle $\varphi=140^o$ illuminated by an incident TH wave with angle $\theta_{inc}=70^o$ for various incident skew angles. The regular parts of the spectral functions, computed according to the method described in \ref{C4:regpart}, diverge when $\crit$. Additional work must be done to find the reason for this instability. In the meantime, a numerical approximation is proposed in order to obtain a diffraction coefficient that only diverges in the directions of specular reflection (as is expected for a \acrshort{gtd} diffraction coefficient). These coefficients have yet to be validated numerically or experimentally.

\section*{Conclusion}
Using the spectral functions method, the elastic wave diffracted by a skew incident plane wave on as stress-free wedge has been studied. In cases where Snell's law of diffraction yields a propagative wave for both longitudinal and transversal diffracted waves, a semi-analytical computation method is developed theoretically and numerically. The corresponding code has been tested in three different manners (by comparison to the 2D elastic code for 2D configurations, by testing the acoustic limit of the code and by computing the regular part in the case of reflection on an infinite plane), yielding promising results, but has yet to be validated (numerically or experimentally) for 3D elastic cases. 

In the case of an incident transversal wave, with a skew angle higher than the critical angle in diffraction, Snell's law of diffraction leads only to transversal diffracted waves. This case is also treated theoretically but the corresponding numerical code produces diverging results. Further investigations are necessary in order to solve this problem. In the meantime, an approximate solution is proposed, in order to obtain a less exact, yet physically meaningful result. This approximate solution still remains to be tested.
