\chapter[][3D Elastic Case]{The spectral functions method for 3D elastic wave diffraction by a stress-free wedge}
\label{chap-3D}

\section*{Introduction}

\section{Problem statement}

\begin{figure}[h]
\centering
	\includegraphics[width=\textwidth]{images/chapter4/wedge_3D.png}
\caption{Geometry of the problem}
\label{diedre_coords}
\end{figure}

Let us consider the problem of an elastic wave diffracted by a stress-free wedge delimited by faces $\mathcal{S}_1$ and $\mathcal{S}_2$. The geometry of the problem is shown on Fig.~\ref{diedre_coords}. The domain $\Omega$ is the inside of the wedge, defined by :
\begin{equation}
\Omega=\{ (r\cos \theta \cos \delta, r \sin \theta \cos \delta, r \sin \delta)\, \backslash \, \theta \in \rbrack 0, \varphi \lbrack, \, \delta \in \rbrack -\frac{\pi}{2}, \frac{\pi}{2} \lbrack \}
\end{equation}
%et sa surface
%$$ \mathcal{S}=\mathcal{S}_1 \cup \mathcal{S}_2 $$

The incident wave is a plane wave of the form
\begin{equation}
\mathbf{u}^{inc}(\mathbf{x},t)=\mathbf{A}_{\alpha}e^{i(\mathbf{k}_{\alpha}^{inc}\cdot \mathbf{x}-\omega t)}
\end{equation}
where $\mathbf{A}_{\alpha}$ is the amplitude vector of the incident wave and $\mathbf{k}_{\alpha}^{inc}$ is the incident wave vector. The type of the incident wave is denoted $\mathbf{k}_{\alpha}^{inc}$ (L for a longitudinal wave, TH for transverse horizontal and TV for transverse vertical). $(\mathbf{x}_1,\mathbf{y}_1, \mathbf{z}_1)$ is a Cartesian coordinate system associated to face $\mathcal{S}_1$. In this system, the incident wave vector is given by :
\begin{equation}
\mathbf{k}_{\alpha}^{inc}=\frac{\omega}{c_{\alpha}} \begin{pmatrix}
\cos\theta_{inc} \cos \delta_{inc} \\ \sin\theta_{inc} \cos \delta_{inc} \\
\sin \delta_{inc}
\end{pmatrix}
\end{equation}
As always, $c_L$ is the velocity of longitudinal waves and $c_T$ is the velocity of transverse waves.

The amplitude vector can be directed by three different and two-by-two orthogonal vectors, depending on the incident wave's polarization. These unit polarization vectors are noted $\hat{i}_*$, where $*=L, TH, TV$ and are given by Achenbach \cite{Achenbach} :
\begin{equation}
\hat{i}_L = \begin{pmatrix}
\cos\theta_{inc} \cos \delta_{inc} \\ \sin\theta_{inc} \cos \delta_{inc} \\
\pm \sin \delta_{inc}
\end{pmatrix}
\hfill
\hfill
\hat{i}_{TV} = \begin{pmatrix}
\mp\cos\theta_{inc} \sin \delta_{inc} \\ \mp\sin\theta_{inc} \sin \delta_{inc} \\
\cos \delta_{inc}
\end{pmatrix}
\hfill
\hfill
\hat{i}_{TH} = \begin{pmatrix}
-\sin\theta_{inc} \\ \cos\theta_{inc} \\
0
\end{pmatrix}
\label{ivec}
\end{equation}
where the top sign gives the polarization of an incident wave and the bottom sign gives the polarization of a diffracted wave.

In all the following, vectors are expressed in the coordinate system $(\mathbf{x}_1,\mathbf{y}_1, \mathbf{z}_1)$, except when explicitly mentioned otherwise. For a homogeneous, isotropic material, the linear elasticity equation solved by the displacement field $\mathbf{u}$ is 
\begin{equation}
\underline{\mu} \Delta \mathbf{u} + (\underline{\lambda}+\underline{\mu})\nabla \nabla \mathbf{u} = \rho \frac{\partial^2 \mathbf{u}}{\partial t^2}
\label{C4:Elasticitelin}
\end{equation}
On each of the wedge faces, the displacement field verifies the zero-stress boundary conditions, expressed as :
\begin{equation}
(\underline{\lambda} \nabla \mathbf{u} .\mathbf{\mathbb{I}_3}+2\underline{\mu} \mathbf{\varepsilon} (\mathbf{u})).\mathbf{n}=0
\label{C4:stressfree}
\end{equation}
where $\mathbf{\mathbb{I}_3}$ is the identity matrix of the third order, $n$ is the inward facing normal to the wedge face ($n=\mathbf{y}_1$ on $\mathcal{S}_1$  and $n=\mathbf{y}_2$ on $\mathcal{S}_2$) and $\underline{\lambda}, \underline{\mu}$ are the Lamé coefficients of the considered elastic medium. The expression of the deformations tensor is :
\begin{equation}
\mathbf{\varepsilon}(\mathbf{u})=\frac{1}{2} \begin{pmatrix}
2\dfrac{\partial u_1}{\partial x_1} & \dfrac{\partial u_1}{\partial y_1}+\dfrac{\partial u_2}{\partial x_1}&\dfrac{\partial u_1}{\partial z}+\dfrac{\partial u_3}{\partial x_1} \\
\dfrac{\partial u_1}{\partial y_1}+\dfrac{\partial u_2}{\partial x_1}&2\dfrac{\partial u_2}{\partial y_1}&\dfrac{\partial u_2}{\partial z}+\dfrac{\partial u_3}{\partial y_1} \\
\dfrac{\partial u_1}{\partial z}+\dfrac{\partial u_3}{\partial x_1} &\dfrac{\partial u_2}{\partial z}+\dfrac{\partial u_3}{\partial y_1} & 2\dfrac{\partial u_3}{\partial z}
\end{pmatrix}
\end{equation}
Kamotski and Lebeau \cite{KamotskiLebeau} have proven existence and uniqueness of the solution to this problem in the 2D the case. We will suppose that their demonstration is still valid in the 3D case.

From hereon after, bold characters will be reserved to matrices in order to simplify notations. The solutions being time harmonic, the factor $e^{-i\omega t}$ will be implied but omitted everywhere. Furthermore, since there is no obstacle to propagation in the $z$ direction, $e^{i\frac{\omega}{c_{\alpha}}\sin\delta_{inc}z}$ is also a common factor to all the terms which appear in the solution.

The total field is written as the sum of an incident field $u^{inc}$ and a scattered field $u_0$
\begin{equation}
u=u_0+u^{inc}
\label{C4:scat}
\end{equation}

The dimensionless problem is obtained by applying the following variable change :
\begin{equation}
u_0(x,y,z)=v\left( \frac{\omega}{c_L} x, \frac{\omega}{c_L} y \right)e^{i\nu_{\beta}\sin\delta_{\beta}z}
\label{C4:adiming}
\end{equation}
where $\delta_{\beta}$ is the angle of Snell's cone of diffraction, the dimensionless Lamé parameters $\lambda,\mu$ are given by \eqref{LameAdim} and parameters $\nu_L$ and $\nu_T$ are defined by \eqref{nuLnuT}. Since $e^{i\nu_{\alpha}\sin\delta_{inc}z}$ is a common factor to all the terms of the solution, we can deduce Snell's law of diffraction :
\begin{equation}
\nu_{\alpha}\sin\delta_{inc}=\nu_{\beta}\sin\delta_{\beta}
\label{C4:Snelldiff}
\end{equation}
To simplify notations, we define the following parameter $\tau$ is defined by :
\begin{equation}
\tau=\nu_{\alpha}\sin\delta_{\alpha}
\label{deftau}
\end{equation}
Note that we therefore always have $\tau \in \lbrack -\nu_{\alpha}, \nu_{\alpha} \rbrack$. $u_0$'s $z$-dependency is entirely contained in the factor $e^{i\tau z}$ which will be implied but omitted in all the following.

Substituting \eqref{C4:scat} and \eqref{C4:adiming} into \eqref{C4:Elasticitelin} and \eqref{C4:stressfree} yields the dimensionless problem
\begin{eqnarray}
(\mathcal{P}^*) \hspace{2em} \left\{
\begin{array}{lr}
(E+1)v=0 & (\Omega) \\
Bv=-Bv_{\alpha}^{inc} & (\mathcal{S})
\end{array}
\right.
\label{C4:Padim}
\end{eqnarray}
where $(v_1,v_2,v_3)$ are the components of vector $v$ :
\begin{equation}
Ev=\mu (\Delta v -\tau^2 v)+(\lambda+\mu)
\begin{pmatrix}
\frac{\partial^2 v_1}{\partial x^2}+\frac{\partial^2 v_2}{\partial x \partial y} + i\tau\frac{\partial v_3}{\partial x} \\
\frac{\partial^2 v_1}{\partial x \partial y}+\frac{\partial^2 v_2}{\partial y^2}+ i\tau\frac{\partial v_3}{\partial y}\\
i\tau\left( \frac{\partial v_1}{\partial x}+\frac{\partial v_2}{\partial y}\right)-\tau^2 v_3
\end{pmatrix}
\label{C4:Eadim}
\end{equation}
and
\begin{equation}
Bv=\begin{pmatrix}
\mu \left(\frac{\partial v_x}{\partial y}+\frac{\partial v_y}{\partial x}\right) \\
\frac{\partial v_y}{\partial y}+\lambda\left( \frac{\partial v_x}{\partial x}+i\tau v_z\right)\\
\mu \left(\frac{\partial v_z}{\partial y}+i\tau v_y\right)
\end{pmatrix}
\label{C4:Badim}
\end{equation}
where $E$ and $B$ are respectively the dimensionless linear elasticity operator and normal stress operator. The first equation of system \eqref{C4:Padim} is the dimensionless version of the linear elasticity equation and the second equation is the dimensionless version of the stress-free boundary conditions.

\section{Integral formulation of the solution}
As for the previous cases, the first step in solving problem $(\mathcal{P}^{\alpha})$ is to formulate the solution as an integral. 
\subsection{Limiting absorption principle}
The limiting absorption principle is applied to $(\mathcal{P}^{\alpha})$. This means that it is considered as a special case $(\varepsilon=0)$ of the problem
\begin{eqnarray}
(\mathcal{P}^*_{\epsilon}) \hspace{2em} \left\{
\begin{array}{lr}
(E+e^{-2i\epsilon})v^{\epsilon}=0 & (\Omega) \\
Bv^{\epsilon}=-Bv_*^{inc} & (\mathcal{S})
\end{array}
\right.
\label{C4:Pabs}
\end{eqnarray}
Following Kamotski and Lebeau \cite{KamotskiLebeau}, we will once again suppose that the solution can be expressed as the sum of two contributions, corresponding to each of the wedge faces :
\begin{equation}
v^{\epsilon}=v_1^{\epsilon}+v_2^{\epsilon}
\label{C4:v1+v2}
\end{equation}
where functions $v_j^{\epsilon}$ are now defined on all of  $\mathbb{R}^3$ by
\begin{equation}
v_j^{\epsilon}=-(E+e^{-2i\epsilon})^{-1} \begin{bmatrix}
\begin{pmatrix}
\alpha_j \\
\beta_j \\
\gamma_j
\end{pmatrix}
\otimes \delta_{\mathcal{S}_j}
\end{bmatrix}
\label{C4:vjdef}
\end{equation}
Distributions $\alpha_j,\beta_j, \gamma_j $ are unknown and are supposed to belong to the special call $\mathcal{A}$ defined in \ref{defClassA}. We can now define the outgoing solution of $(\mathcal{P}^{\alpha})$ analogously to the 2D case :
\begin{definition}
	 v is called an outgoing solution of equation \eqref{C4:Padim} if v is a solution of the form
	\begin{equation}
	\label{C4:decomposition}
	v=v_1|_{\Omega}+v_2|_{\Omega}
	\end{equation}
	where, for $j=1,2$ :
	\begin{equation}
	\label{C4:inv_potentiels}
	v_j=-\lim_{\epsilon \to 0} (E+e^{-2i\epsilon})^{-1} \begin{bmatrix}
\begin{pmatrix}
\alpha_j \\
\beta_j \\
\gamma_j
\end{pmatrix}
\otimes \delta_{\mathcal{S}_j}
\end{bmatrix}
	\end{equation}
	where $\alpha_j,\beta_j,\gamma_j \in \mathcal{A}$. %and where $\delta_{\mathcal{S}_1}$ and $\delta_{\mathcal{S}_2}$ sont les distributions de Dirac associés aux faces $\mathcal{S}_1$ et $\mathcal{S}_2$ du dièdre respectivement.
\end{definition}
The following theorem was proved by Kamotski and Lebeau \cite{KamotskiLebeau} in the 2D case. We will suppose that their proof can be adapted to the 3D case and that the theorem is still true.
\begin{theorem}
Equation \eqref{C4:Padim} admits a unique outgoing solution.
\end{theorem}
Nous that the outgoing solution has been defined, we will derive an integral formulation of this solution.

\subsection{Integral formulation}
The two-sided Fourier transform and its inverse transform are defined by \eqref{fullfourierdef}. The first step in determining an integral formulation of the solution is to apply the two-sided Fourier transform to \eqref{C4:vjdef}. This possible because all the distributions that appear in this equation are tempered distributions and they therefore admit a Fourier transform. We then have :
\begin{equation}
\hat{v}^{\epsilon}_j(\xi,\eta)=(\mathbf{M}-e^{-2i\epsilon}\mathbf{\mathbb{I}_3})^{-1}\Sigma_j(\xi)
\label{C4:matMvjeps}
\end{equation}
where $\Sigma_j, j=1,2$  are the unknown spectral functions, defined by :
\begin{equation}
\Sigma_j(\xi,)=\begin{pmatrix}
\hat{\alpha_j}(\xi)\\ \hat{\beta_j}(\xi) \\ \hat{\gamma_j}(\xi)
\end{pmatrix}
\end{equation}
and where \textbf{M} is the two-sided Fourier transform of operator $E$. Its expression is :
\begin{equation}
\mathbf{M}(\xi,\eta)=
\begin{pmatrix}
\xi^2+\mu(\eta^2+\tau^2) & (\lambda+\mu)\xi \eta & (\lambda+\mu)\xi\tau \\
 (\lambda+\mu)\xi \eta & \eta^2+\mu(\xi^2+\tau^2) & (\lambda+\mu)\eta\tau \\
(\lambda+\mu)\xi\tau & (\lambda+\mu)\eta\tau & \tau^2+\mu(\xi^2+\eta^2) 
\end{pmatrix}
\label{C4:matM}
\end{equation}
Substituting $\lambda$ by $1-2\mu$ and $\mu$ by $1/\nu_T^2$, \eqref{C4:matM} yields
\begin{multline}
(\mathbf{M}-e^{-2i\epsilon}\mathbb{I}_3)^{-1}=\\
\frac{\begin{pmatrix}
\xi^2+\nu_T^2(\eta^2+\tau^2-e^{-2i\epsilon}) & (1-\nu_T^2)\xi \eta & (1-\nu_T^2)\xi \tau \\
(1-\nu_T^2)\xi \eta & \eta^2+\nu_T^2(\xi^2+\tau^2-e^{-2i\epsilon}) & (1-\nu_T^2)\tau \eta \\
(1-\nu_T^2)\xi \tau & (1-\nu_T^2)\tau \eta & \tau^2+\nu_T^2(\eta^2+\xi^2-e^{-2i\epsilon})
\end{pmatrix}}{(\xi^2+\eta^2+\tau^2-e^{-2i\epsilon})(\xi^2+\eta^2+\tau^2-\nu_T^2e^{-2i\epsilon})} 
\end{multline}

Finally, the integral formulation of $v_j$ is obtained by inverting the two-sided Fourier transform applied in \eqref{C4:matMvjeps} :
\begin{equation}
v_j^{\epsilon}(x_j,y_j)=\frac{1}{4\pi^2}\int_{\mathbb{R}^2} e^{i x_j\xi}\left( \int_{-\infty}^{+\infty}e^{iy_j\eta} (\mathbf{M}-e^{-2i\epsilon}\mathbf{\mathbb{I}_3})^{-1} \,d\eta \right) \Sigma_j(\xi) \,d\xi
 \label{C4:invdouble}
\end{equation}

The poles of $(\mathbf{M}-e^{-2i\epsilon}\mathbf{\mathbb{I}_3})^{-1}$ are located in $\eta=\pm \zeta_*^{\epsilon}(\xi)$, where $*=L,T$ and
\begin{equation}
\zeta_*^{\epsilon}(\xi)=\sqrt{e^{-2i\epsilon}\nu^2_*-(\xi^2+\tau^2)}
\end{equation}
Let us define $\nuti_*, *=L,T$ by
\begin{equation}
\nuti_*^{\epsilon}=\sqrt{e^{-2i\epsilon}\nu_*^2-\tau^2}
\label{defnutieps}
\end{equation}
If the incident wave is longitudinal, then $\nuti \in \mathbb{R}$. However, if the incident wave is transverse, then two cases may occur :
\begin{itemize}
	\item if $|\sin\delta_{inc}| \leq \frac{\nu_L}{\nu_T}$, then $\nuti_L \in \mathbb{R}$
	\item if $|\sin\delta_{inc}| >\frac{\nu_L}{\nu_T}$, then $\nuti_L=i\eta_L \in i\mathbb{R}$ 
\end{itemize} 
In any case,
\begin{equation}
\zeta_*^{\epsilon}(\xi)=\sqrt{\tilde{\nu}_*^{\epsilon^2}-\xi^2}
\end{equation}
The chosen branch cut is the one for which the square root's imaginary part is positive. It is defined by
 \begin{eqnarray}
 \zeta_*(\xi)=
 \left\{
 \begin{array}{lr}
 i\sqrt{\xi^2-\tilde{\nu}_*^2}& \mbox{si } |\xi| \geq |\nuti_*| \\
 -\sqrt{\tilde{\nu}_*^2-\xi^2}& \mbox{si } |\xi| \leq |\nuti_*|
 \end{array}
 \right.
 \label{defzeta}
 \end{eqnarray}
%La racine carrée complexe étant définie au signe près, $\sqrt{z}$ peut prendre deux valeurs distinctes pour $z$ fixé. C'est ce qu'on appelle les branches de coupure de la fonction. Les points de branchement sont les valeurs pour lesquelles ces branches se croisent (ici il s'agit de $\xi=\pm \tilde{\nu}_*^{\epsilon}$). Afin de bien définir $\zeta_*^{\epsilon}(\xi)$, on choisit la branche dont la partie imaginaire est positive. 

The inner integral of \eqref{C4:invdouble} is computed using Cauchy's residue theorem :
\begin{equation}
v_j^{\epsilon}(x_j,y_j)=\frac{i}{4\pi}e^{2i\epsilon}\int_{\mathbb{R}} e^{ix_j\xi}\sum_{*=L,T}e^{i|y_j|\zeta_*^{\epsilon}(\xi)}\mathbf{M_*}(\xi,\mbox{sgn }y_j)\Sigma_j(\xi)\,d\xi
\label{C4:vjeps}
\end{equation}
where $\mathbf{M_*}(\xi,t),$ and $t=\mbox{sgn}( y_j)$ are defined by
\begin{subequations}
\begin{equation}
\mathbf{M_L}(\xi,t)=\begin{pmatrix}
\frac{\xi^2}{\zeta_L^{\epsilon}} &t\xi & \frac{\xi\tau}{\zeta_L^{\epsilon}} \\
t\xi & \zeta_L^{\epsilon} & t\tau \\
\frac{\xi\tau}{\zeta_L^{\epsilon}}& t\tau & \frac{\tau^2}{\zeta_L^{\epsilon}}
\end{pmatrix}
\label{C4:MLeps}
\end{equation}
\begin{equation}
\mathbf{M_T^{\epsilon}}(\xi,t)=\begin{pmatrix}
\zeta_T^{\epsilon} +\frac{\tau^2}{\zeta_T^{\epsilon}}& -t\xi &-\frac{\xi\tau}{\zeta_T^{\epsilon}}\\
-t\xi & \frac{\xi^2+\tau^2}{\zeta_T^{\epsilon}}&-t\tau \\
-\frac{\xi\tau}{\zeta_T^{\epsilon}}&-t\tau&\zeta_T^{\epsilon} +\frac{\xi^2}{\zeta_T^{\epsilon}}
\end{pmatrix}
\label{C4:MTeps}
\end{equation}
\label{C4:M*eps}
\end{subequations}

These matrices can be computed in two different manners, providing a way to test this result. the first way to compute these matrices is by direct computation of the residues of matrix $(\mathbf{M}-e^{-2i\epsilon}\mathbf{\mathbb{I}_3})^{-1}$. The second option for computing these matrices is by computing the eigen vectors and eigen values of $\mathbf{M}$. %On trouve bien les mêmes résultats avec les deux méthodes.

The three eigen vectors of $\mathbf{M}$ and the corresponding eigen values are :
\begin{subequations}
\begin{equation}
    \mathbf{M}\begin{pmatrix}
    \xi \\ \eta \\ \tau 
    \end{pmatrix}
    =(\xi^2+\eta^2+\tau^2)\begin{pmatrix}
    \xi \\ \eta \\ \tau 
    \end{pmatrix}
\end{equation}
\begin{equation}
   \mathbf{M}\begin{pmatrix}
    -\eta \\ \xi \\ 0
    \end{pmatrix}
    =\frac{\xi^2+\eta^2+\tau^2}{\nu_T^2}\begin{pmatrix}
    \xi \\ \eta \\ \tau 
    \end{pmatrix} 
\end{equation}
\begin{equation}
   \mathbf{M}\begin{pmatrix}
    -\xi\tau \\ \eta\tau \\ \xi^2+\eta^2
    \end{pmatrix}
    =\frac{\xi^2+\eta^2+\tau^2}{\nu_T^2}\begin{pmatrix}
    -\xi\tau \\ \eta\tau \\ \xi^2+\eta^2
    \end{pmatrix} 
\end{equation}
\end{subequations}
These three vectors are linearly independent and constitute a vector basis of $\mathbb{C}^3$. This means that any vector of $\mathbb{C}^3$ can be expressed as a linear combination of these three vectors. Notably :
\begin{equation}
    \begin{pmatrix}
    \hat{\alpha}_j\\ \hat{\beta}_j\\ \hat{\gamma}_j
    \end{pmatrix}
    = \frac{\xi\hat{\alpha}_j+\eta\hat{\beta}_j+\tau\hat{\gamma}_j}{\xi^2+\eta^2+\tau^2}\begin{pmatrix}
    \xi \\ \eta \\ \tau 
    \end{pmatrix} + \frac{\xi\hat{\beta}_j-\eta\hat{\alpha}_j}{\xi^2+\eta^2} \begin{pmatrix}
    -\eta \\ \xi \\ 0
    \end{pmatrix} + \frac{(\xi^2+\eta^2)\hat{\gamma}_j-\tau(\xi\hat{\alpha}_j+\eta\hat{\beta}_j)}{(\xi^2+\eta^2)(\xi^2+\eta^2+\tau^2)} \begin{pmatrix}
    -\xi\tau \\ \eta\tau \\ \xi^2+\eta^2
    \end{pmatrix}
\end{equation}
This second computation method thus yields
\begin{equation}
v_j^{\epsilon}(x_j,y_j)=\frac{i}{4\pi}e^{2i\epsilon}\int_{\mathbb{R}} e^{ix_j\xi}\sum_{*=L,TH,TV}e^{i|y_j|\zeta_*^{\epsilon}(\xi)}\mathbf{M_*}(\xi,\mbox{sgn }y_j)\Sigma_j(\xi)\,d\xi
\label{C4:vjeps2}
\end{equation}
where $\mathbf{M_L}(\xi,t),$ is given by \eqref{C4:MLeps} and
\begin{subequations}
\begin{equation}
\mathbf{M_{TV}}(\xi,t)=\begin{pmatrix}
\frac{\xi^2\tau^2}{\zeta_T(\xi^2+\zeta^2)}&\frac{t\xi\tau^2}{\xi^2+\zeta_T^2}&\frac{-\xi\tau}{\zeta_T}\\
\frac{t\xi\tau^2}{\xi^2+\zeta_T^2}&\frac{\zeta_T\tau^2}{\xi^2+\zeta_T^2}&-t\tau\\
\frac{-\xi\tau}{\zeta_T}&-t\tau&\frac{\xi^2+\zeta_T^2}{\zeta_T}
\end{pmatrix}
\label{C4:MTVeps}
\end{equation}
\begin{equation}
\mathbf{M_{TH}}(\xi,t)=\left(1+\frac{\tau^2}{\xi^2+\zeta_T^2}\right)\begin{pmatrix}
\zeta_T&-t\xi&0\\
-t\xi&\frac{\xi^2}{\zeta_T}&0\\
0&0&0
\end{pmatrix}
\label{C4:MTHeps}
\end{equation}
\label{C4:M*eps2}
\end{subequations}
Note that $ \mathbf{M_T}=\mathbf{M_{TH}}+\mathbf{M_{TV}}$. Expressions \eqref{C4:vjeps} and \eqref{C4:vjeps2} are equivalent.

Integral \eqref{C4:vjeps} is well defined for all values of $\epsilon \in \rbrack 0, \pi \lbrack $, since for these values of $\epsilon$, the integration contour never crosses the branch points of the integrand, which are located at $\xi=\pm \tilde{\nu}_*^{\epsilon}$, outside of the real axis.

According to Croisille et Lebeau \cite{CroisilleLebeau}, convergence in the 2D case is verified for $\epsilon \rightarrow 0$. We will suppose that this is still the case in 3D. The deformation contour $\mathbb{R}$ is deformed into contour $\Gamma_{0}$, visible on Fig.~\ref{C4:Gamma0_noncr} in the case where $\nuti_L\in\mathbb{R}$ and on Fig.~\ref{C4:Gamma0_cr} in the other case. This way the branch points of the integrand are avoided.

In all the following, superscript $\epsilon=0$ will be omitted in order to alleviate notations. Finally:
\begin{equation}
v_j(x_j,y_j)=\frac{i}{4\pi}\int_{\Gamma_0} e^{ix_j\xi}\sum_{*=L,T}e^{i|y_j|\zeta_*(\xi)}\mathbf{M_*}(\xi,\mbox{sgn }y_j)\Sigma_j(\xi)\,d\xi
\label{vj0}
\end{equation}

\begin{figure}
\centering
\begin{subfigure}[b]{0.45\textwidth}
\begin{tikzpicture}
\node at (0,0) {$\times$};
\node at (0.35,0.35) {$0$};
\node at (2.25,0) {$\times$}; % Pole
\node at (1.5,0) {$\times$}; %pole
\node at (1.5,-0.45) {$\tilde{\nu}_L$};
\node at (2.25,-0.45) {$\tilde{\nu}_T$};
\node at (-1.5,0) {$\times$};
\node at (-2.25,0) {$\times$};
\node at (-1.5,0.45) {$-\tilde{\nu}_L$}; %pole
\node at (-2.25,0.45) {$-\tilde{\nu}_T$}; %pole
\node at (3.25,0.38) {$(\Gamma_0)$};
\node at (3.7,-0.38) {$\xi_1$};
\node at (0.38,2.25) {$\xi_2$};
%\draw[ thick, ->] (-2.5,-0.5) arc (180:235:1);
%\node at (-2.7,-0.9) {$F_1$};
%\draw[ thick, ->] (2.5,0.5) arc (0:45:1); %ici c'est les fleches 
%\node at (2.7,0.9) {$F_2$};
\draw[thick,black,yshift=0pt,decoration={markings,
mark=at position 1 with {\arrow{stealth}}},
postaction={decorate}](0,-2.5) -- (0,2.5);
\draw[thick,black,yshift=0pt,
decoration={ markings,  % This schema allows for fine-tuning the positions of arrows 
      mark=at position 0.1 with {\arrow{latex}},
      mark=at position 0.6 with {\arrow{latex}},
      mark=at position 0.9 with {\arrow{latex}}},
      mark=at position 1 with {\arrow{stealth}},
      postaction={decorate}]
      (-4,0) -- (-2.5,0)  arc (-180:0:0.25) -- (-1.75,0)  arc (-180:0:0.25)  -- (1.25,0)arc (180:0:0.25)  -- (2,0)arc (180:0:0.25) -- (3.75,0); % ca c'est l'axe
\end{tikzpicture}
\caption{Contour $\Gamma_0$ in the case where $\nuti_L \in \mathbb{R}$}
\label{C4:Gamma0_noncr}
\end{subfigure}
\hfill
\centering
\begin{subfigure}[b]{0.45\textwidth}
\begin{tikzpicture}
\node at (0,0) {$\times$};
\node at (0.35,0.35) {$0$};
\node at (2.25,0) {$\times$}; %pole
\node at (2.25,-0.45) {$\tilde{\nu}_T$};
\node at (-2.25,0) {$\times$};
\node at (-2.25,0.45) {$-\tilde{\nu}_T$}; %pole
\node at (3.5,0.38) {$(\Gamma_0)$};
\node at (0,1.5) {$\times$};
\node at (0.45, 1.5) {$\nuti_L$};
\node at (0,-1.5) {$\times$};
\node at (0.45,-1.5) {$-\nuti_L$};
\node at (3.9,-0.38) {$\xi_1$};
\node at (0.38,2.25) {$\xi_2$};
\draw[thick,black,yshift=0pt,decoration={markings,
mark=at position 0.99 with {\arrow{stealth}}},
postaction={decorate}](0,-2.5) -- (0,2.5);
\draw[thick,black,yshift=0pt,
decoration={ markings,  % This schema allows for fine-tuning the positions of arrows 
      mark=at position 0.1 with {\arrow{latex}},
      mark=at position 0.6 with {\arrow{latex}},
      mark=at position 0.9 with {\arrow{latex}},
      mark=at position 0.99 with {\arrow{stealth}}},
     postaction={decorate}]
      (-3.75,0) -- (-2.5,0)  arc (-180:0:0.25) -- (2,0)arc (180:0:0.25) -- (4,0); % ca c'est l'axe
\end{tikzpicture}
\caption{Contour $\Gamma_0$ in the case where $\nuti_L \in i\mathbb{R}$}
\label{C4:Gamma0_cr}
\end{subfigure}
\caption{Contour $\Gamma_0$ in the complex plane $\xi=\xi_1+i\xi_2$}
\label{C4:Gamma0}
\end{figure}


\subsection{Far field approximation}
$x=(x_1,y_1,z_1)=(r\cos\theta\cos\delta,r\sin\theta\cos\delta,r\sin\delta)$ is an observation point, indexed by its spherical coordinates, visible on Fig.~\ref{diedre_coords}. According to \eqref{C4:adiming},the scattered field at point $P$ is :
\begin{equation}
u_0(x_1,y_1,z)=v(\frac{\omega}{c_L}r\cos\theta\cos\delta,\frac{\omega}{c_L}r\sin\theta\cos\delta)e^{ik_{\beta}\sin\delta_{\beta}z}
\end{equation}
The far field parameter is $R=\frac{\omega r}{c_L}$. The aim is to determine the asymptotic behavior of $v(R\cos\theta\cos\delta,R\sin\theta\cos\delta)$ when $R\rightarrow +\infty$.The first step is to apply the change following of variables in integral $v_1$ :
\begin{equation}
\begin{split}
\xi&=\tilde{\nu}_*\cos\lambda \\
d\xi&=-\tilde{\nu}_*\sin\lambda\, d\lambda
\end{split}
\label{C4:changevar2}
\end{equation}
yielding
\begin{equation}
v_1(r,\theta,\delta)=\frac{i}{4\pi} \int_{C_0}\sum_{*=L,T}\tilde{\nu}_*^2 e^{i\tilde{\nu}_*R\cos\delta\cos(\lambda+\bar{\theta})}\mathbf{ P_*}(\lambda,t)\Sigma_1(\tilde{\nu}_*\cos\lambda) \, d \lambda
\label{C4:v1C0}
\end{equation}
where $\bar{\theta}$ has been defined by \eqref{obs} and
\begin{equation}
\mathbf{P_L}(\lambda,t)=
\begin{pmatrix}
\cos^2\lambda & -t\cos\lambda\sin\lambda &\frac{\tau}{\tilde{\nu}_L} \cos\lambda \\
-t\cos\lambda\sin\lambda & \sin^2\lambda&-t\frac{\tau}{\tilde{\nu}_L}\sin\lambda \\
\frac{\tau}{\tilde{\nu}_L} \cos\lambda&-t\frac{\tau}{\tilde{\nu}_L}\sin\lambda&\frac{\tau^2}{\tilde{\nu}_L^2}
\end{pmatrix}
\end{equation}
and
\begin{equation}
\mathbf{P_T}(\lambda,t)=
\begin{pmatrix}
\sin^2\lambda+\frac{\tau^2}{\tilde{\nu}_T^2} & t\cos\lambda\sin\lambda &-\frac{\tau}{\tilde{\nu}_T}\cos\lambda \\
t\cos\lambda\sin\lambda & \cos^2\lambda+\frac{\tau^2}{\tilde{\nu}_T^2}&t\frac{\tau}{\tilde{\nu}_T}\sin\lambda \\
-\frac{\tau}{\tilde{\nu}_T}\cos\lambda&t\frac{\tau}{\tilde{\nu}_T}\sin\lambda&1
\end{pmatrix}
\end{equation}
$t=\mbox{sgn} \sin\theta$ and contour $C_0$ is visible on Fig.~\ref{C4:steepestcontour}.

Note that in the case $\nuti_L =i\eta_L \in i\mathbb{R}$, variable change \eqref{C4:changevar2} produces an evanescent term :
\begin{equation}
\begin{split}
v_1(r,\theta,\delta)&=-\frac{i}{4\pi} \int_{C_0^L}\eta_L^2 e^{-\eta_L R\cos\delta\cos(\lambda+\bar{\theta})}\mathbf{ P_L}(\lambda,t)\Sigma_1(i\eta_L\cos\lambda) \, d\lambda\\
&+\frac{i}{4\pi} \int_{C_0}\tilde{\nu}_T^2 e^{i\nuti_T R\cos\delta\cos(\lambda+\bar{\theta)})}\mathbf{ P_T}(\lambda,t)\Sigma_1(\tilde{\nu}_T\cos\lambda) \, d\lambda
\end{split}
\label{C4:v1C0evan}
\end{equation}
where contour $C_0^L$ is visible on Fig.~\ref{C4:steepestcontour}. The exponential term in the first integral of \eqref{C4:v1C0evan} is $e^{-\eta_L R\cos\delta\cos(\lambda+\bar{\theta})}$, where $\eta_L>0$, $R>0$ and $\cos\delta>0$. Furthermore, for $\lambda :\in C_0^L$, we have :
\begin{equation}
\lambda=-\left(\dfrac{\pi}{2}^+\right)+i\lambda_2
\end{equation}
where the notation $\dfrac{\pi}{2}^+$ refers to a real number which tends to $\frac{\pi}{2}$ with superior values and $\lambda_2=\rm Im(\lambda)$. This yields
\begin{equation}
\cos(\lambda+\bar{\theta})=\cos(\bar{\theta}-\dfrac{\pi}{2})\cosh\lambda_2-i\sin(\bar{\theta}-\dfrac{\pi}{2})\sinh\lambda_2
\end{equation}
Having $\bar{\theta} \in \lbrack 0,\pi \rbrack$, we have $\cos(\bar{\theta}-\dfrac{\pi}{2})=\sin\bar{\theta}\leq 0$ and 
\begin{equation}
|e^{-\eta_L R\cos\delta\cos(\lambda+\bar{\theta})}|=e^{-\eta_L R\cos\delta\sin\bar{\theta}\cosh\lambda_2}
\end{equation}
The amplitude of the integrand in the first integral of \eqref{C4:v1C0evan} decreases exponentially as the distance from the edge grows. %This means that it is an evanescent term.

\begin{figure}
\centering
%\begin{subfigure}[b]{0.45\textwidth}
%    \begin{tikzpicture}
%	\draw[step=1.5cm,gray,very thin,dashed](-5,-2.7)grid(-1,3.7);
%
%	\draw[thin] (-5,0)  -- (-1,0) node[above]{$\lambda_1$};
%	\draw[thin](-1.5,-2.5)--(-1.5,3.5) node[left]{$\lambda_2$};
%
%	\node at (-4.5,-0.2) {$-\pi$};
%	\node at (-1.3,0.2) {$0$};
%	\node at (-3.8,0.3) {$\lambda_s$};
%	\node at (-3.5,3.6) {$\mathcal{C}_0^L$};
%	\node at (-5.5,3) {$\gamma_{\theta}^L$};
%	
%	%% steepest descent
%	\draw[black,very thick,decoration={ markings, mark=at position 0.5 with {\arrow{latex}}},
%	postaction={decorate}][domain=-1:-4] plot({0.5+-pi*0.7-2*asin(1/sinh(-\x))*pi/180},\x+1);
%	\draw[black,very thick,decoration={ markings, mark=at position 0.5 with {\arrow{latex}}},
%	postaction={decorate}][domain=4:1] plot({0.5+-2*pi-2*asin(1/sinh(-\x))*pi/180},\x-1);
%	
%	%% Courbe C0L
%	\draw[black,very thick,decoration={ markings, mark=at position 0.5 with {\arrow{latex}}},
%	postaction={decorate}][domain=3.5:0] plot({-3 -acos(1/cosh(\x))/360},\x); 
%	\draw[black,very thick,decoration={ markings, mark=at position 0.5 with {\arrow{latex}}},
%	postaction={decorate}][domain=0:-2.8] plot({-3+acos(1/cosh(\x))/360},\x);
%
%	\end{tikzpicture}
%	\caption{Contours $C_0^L$ and $\gamma_{\theta}^L$ in the case $\nuti_L \in i\mathbb{R}$ }
%	\label{C0L}
%\end{subfigure}
%\hfill
%\centering
%\begin{subfigure}[b]{0.45\textwidth}
\begin{tikzpicture}
	\draw[step=1.5cm,gray,very thin,dashed](-2.5,-2.7)grid(3.7,3.7);
	\draw[thin, decoration={ markings,  
		mark=at position 1 with {\arrow{stealth}}},
	postaction={decorate}] (-2.5,0)  -- (3.7,0) node[above]{$\lambda_1$};
	\draw[thin, decoration={ markings,  
		mark=at position 1 with {\arrow{stealth}}},
	postaction={decorate}](0,-2.5)--(0,3.5) node[left]{$\lambda_2$};

	\node at (2.8,-0.2) {$\pi$};
	\node at (-0.2,0.2) {$0$};
	\node at (0.92,0.5) {$\lambda_s$};
	\node at (3,3.1) {$C_0$};
	\node at (1.9,3.1) {$\gamma_{\theta}$};
	\node at (-1.9,3.3) {$C_0^L$};
	
	\draw[black,very thick, decoration={ markings,  
		mark=at position 0.2 with {\arrow{latex}},
		mark=at position 0.8 with {\arrow{latex}}},
	postaction={decorate}] (-1.7,3.5)--(-1.7,-2.5);
	
	\draw[black,very thick][domain=0:3.5] plot({2*pi/7 + acos(1/cosh(\x))*pi/180},\x);
	\draw[black,very thick, decoration={ markings,  
		mark=at position 0.2 with {\arrow{latex}}},
	postaction={decorate}][domain=0:-2.8] plot({2*pi/7 - acos(1/cosh(\x))*pi/180},\x);
	
	\draw[very thick,black,%xshift=0pt,
	decoration={ markings,  
		mark=at position 0.8 with {\arrow{latex}}},
	postaction={decorate}]
	(0.2,0.2) arc (0:-90:0.4) -- (-0.2,-3); 
	
	\draw[very thick,black,%yshift=0pt,
	decoration={ markings, mark=at position 0.5 with {\arrow{latex}}}, postaction={decorate}]
	(pi-0.57,0.2) -- (0.2,0.2);

	\draw[very thick,black,%xshift=0pt,
	decoration={ markings,  
		mark=at position 0.5 with {\arrow{latex}}},
	postaction={decorate}]
	(pi-0.4,3.2) -- (pi-0.4,0.4) arc (0:-90:0.2) ;

	\end{tikzpicture}
%	\caption{Contours $C_0$ and $\gamma_{\theta}$ in the case $\nuti_L \in \mathbb{R}$}
%	\label{C0T}
%\end{subfigure}
	\caption{Contours $C_0$ and $\gamma_{\theta}$ in the complex plane $\lambda=\lambda_1+i\lambda_2$. The stationary phase points are noted $\lambda_s$.}
	\label{C4:steepestcontour}
\end{figure}

The far-field evaluation of the integral is obtained by applying the steepest descent method, presented in appendix \ref{PhaseStationnaire}, to \eqref{C4:v1C0}. To do so, contour $C_0$ is deformed into contour $\gamma_{\theta}$, also visible in Fig.~\ref{C4:steepestcontour}. In the case $\nuti_L \in \mathbb{R}$, this leads to
\begin{equation}
v_1=v_1^{sing}+v_1^{diff}
\end{equation}
where $v_1^{sing}$ is the contribution of all the singularities of the spectral functions crossed during the deformation from $C_0$ to $\gamma_{\theta}$, corresponding to the reflected and head waves, and $v_1^{diff}$ is the contribution of the stationary phase point, corresponding to the diffracted wave and computed using \eqref{steepformula}. Only the contribution of the diffracted waves will be computed here. In order to simplify notations, we will note $\mathbf{P}_*(\lambda,1)=\mathbf{P}_*(\lambda)$ and $\mathbf{P}_*(\lambda,-1)=\mathbf{P}_*(-\lambda)$). The contribution of diffracted waves is 
\begin{equation}
v_1^{diff}(R\cos\theta,R\sin\theta)=\frac{e^{-i\pi/4}}{2\sqrt{2\pi}}\sum_{*=L,T}\tilde{\nu}_*^2\frac{e^{-i\tilde{\nu}_*R\cos\delta}}{\sqrt{\nuti_*R\cos\delta}}\mathbf{P_*}(\pi-\theta)\Sigma_1(-\tilde{\nu}_*\cos\theta)
\end{equation}
Analogously,
\begin{equation}
v_2^{diff}(R\cos(\varphi-\theta),R\sin(\varphi-\theta))=\frac{e^{-i\pi/4}}{2\sqrt{2\pi}}\sum_{*=L,T}\tilde{\nu}_*^2\frac{e^{-i\tilde{\nu}_*R\cos\delta}}{\sqrt{\nuti_*R\cos\delta}}\mathbf{P_*}(\pi-(\varphi-\theta))\Sigma_1(-\tilde{\nu}_*\cos(\varphi-\theta))
\end{equation}

In the case where $\nuti_L=i\eta_L \in i\mathbb{R}$, the far-field evaluation is obtained by applying the steepest descent method, presented in appendix \ref{PhaseStationnaire}, to \eqref{C4:v1C0evan}. Contour $C_0$ is deformed into contour $\gamma_{\theta}$ and contour $C_0^L$ is deformed into contour $\gamma_{\theta}^L$. This leads to
\begin{equation}
v_1=v_1^{sing}+v_1^{diff}+v_1^{evan}
\end{equation}
where $v_1^{sing}$ is the contribution of all the singularities of the spectral functions crossed during the deformation from $C_0$ to $\gamma_{\theta}$, $v_1^{diff}$ is the contribution of the stationary phase point to the integral on $C_0$, corresponding to the diffracted wave, and $v_1^{evan}$ is the contribution of the integral on $C_0^L$, which decays exponentially as the far-field parameter $R$ grows, making it an evanescent wave. Once again, only the contribution of the diffracted waves will be computed here. Contribution $v_1^{diff}$ is computed using \eqref{steepformula} :
%. The contribution of evanescent waves is 
%\begin{equation}
%v^{evan}_1(R\cos\theta,R\sin\theta)= \sqrt{\frac{2\pi}{R\eta_L\cos\delta_{\beta}}} e^{-R\eta_L\cos\delta}\mathbf{P_L}(-\theta)\Sigma_1(i\eta_L\cos\theta)
%\end{equation}
%and the contribution of diffracted waves is :
\begin{equation}
v_1^{diff}(R\cos\theta,R\sin\theta)=\frac{e^{-i\pi/4}}{2\sqrt{2\pi}}\tilde{\nu}_T^2\frac{e^{-i\tilde{\nu}_TR\cos\delta}}{\sqrt{\nuti_TR\cos\delta}}\mathbf{P_T}(\pi-\theta)\Sigma_1(-\tilde{\nu}_T\cos\theta)
\end{equation}
Analogously,
%\begin{equation}
%v^{evan}_2(R\cos\theta,R\sin\theta)= \sqrt{\frac{2\pi}{R\eta_L\cos\delta_{\beta}}} e^{-R\eta_L\cos\delta}\mathbf{P_L}(-(\varphi-\theta))\Sigma_1(i\eta_L\cos(\varphi-\theta))
%\end{equation}
%and
\begin{equation}
v_2^{diff}(R\cos\theta,R\sin\theta)=\frac{e^{-i\pi/4}}{2\sqrt{2\pi}}\tilde{\nu}_T^2\frac{e^{-i\tilde{\nu}_TR\cos\delta}}{\sqrt{\nuti_TR\cos\delta}}\mathbf{P_T}(\pi-(\varphi-\theta))\Sigma_2(-\tilde{\nu}_T\cos(\varphi-\theta))
\end{equation}

In both cases, the diffraction coefficient is defined by
\begin{equation}
v_{\beta}^{diff}(R\cos\theta,R\sin\theta)=D_{\beta}^{\alpha}(\theta,\delta)\frac{e^{-i\nuti_{\beta}R\cos\delta_{\beta}}}{\sqrt{\nuti_{\beta}R\cos\delta_{\beta}}} v^{inc}(R\cos\theta,R\sin\theta) \hat{i}_{\beta}
\label{C4:coeffdiff}
\end{equation}
The total diffracted field is
\begin{equation}
v^{diff}=v_1^{diff}+v_2^{diff}
\end{equation}

Let us now isolate L, TH and TV diffracted waves in order to compute the corresponding diffraction coefficients. Using the expressions of the unit vectors given by \eqref{ivec}, the $\beta$ diffracted wave is given by $v^{diff}\cdot \hat{i}_{\beta}$. This yields :
\begin{equation}
D_{\beta}^{\alpha}(\theta,\delta)=\frac{e^{-i\pi/4}}{2\sqrt{2\pi}}\sum_{j=1,2}\nuti_{\beta}^2 \,{}^t \Sigma_j(-\nuti_{\beta}\cos\theta_j)\cdot\left(\mathbf{P}_{\beta}(\pi-\theta_j).\hat{i}_{\beta}\right)
\end{equation}
where $\theta_1=\theta$ and $\theta_2=\varphi-\theta$.

In order to determine the field diffracted by a wedge illuminated by an incident plane wave, it is sufficient to compute the diffraction coefficient. This coefficient has been expressed in terms of two unknown functions called the spectral functions. The semi-analytical computation of these functions is presented in the following section

\section{Semi-analytical evaluation of the spectral functions}
The first step in computing the spectral functions is to determine a system of functional equations of which they are a solution. We will then show that these functions can be decomposed into two parts : a singular function, computed analytically, and a regular function, approached numerically.
\subsection{Functional equations}
In the previous section, the diffracted wave has been expressed in terms of two unknown functions called the spectral functions. In this subsection, a system of functional equations satisfied by these functions is determined. 

The first step in determining a system of functional equations verified by the spectral functions, is to substitute decomposition \eqref{C4:v1+v2} into the boundary conditions :
\begin{equation}
\left\{
\begin{matrix}
B \big( v_1(x_1,0)+v_2(x_2 \cos \varphi, x_2 \sin \varphi) \big) = -B \rm v_{\alpha}^{inc}|_{\mathcal{S}_1} \\
B \big( v_2(x_2,0)+v_1(x_1 \cos \varphi, x_1 \sin \varphi) \big) = -B \rm v_{\alpha}^{inc}|_{\mathcal{S}_2}
\end{matrix}
\right.
\label{C4:Bivi}
\end{equation}
Let us note $(v_j^1,v_j^2,v_j^3)$ the coordinates of $v_j$ in the Cartesian coordinate system $(x_j,y_j,z_j)$, where $(x_1,y_1,z_1)$ is the coordinate system associated with face $\mathcal{S}_1$ and $(x_2,y_2,z_2)$ is the coordinate system associated with face $\mathcal{S}_2$. These two coordinate systems are linked by (for $j=1,2$):
\begin{equation}
    \left\{
    \begin{matrix}
    x_j=\cos\varphi .x_{3-j}+\sin\varphi. y_{3-j}\\
    y_j=\sin\varphi .x_{3-j}-\cos\varphi .y_{3-j}\\
    z_j=z_{3-j}
    \end{matrix}
    \right.
    \label{C4:changerep}
\end{equation}
Applying \eqref{C4:changerep} to each line of \eqref{C4:Bivi} yields: 
\begin{equation}
\left\{
\begin{matrix}
B_1(v_1)+B_2(v_2)=-Bv_{\alpha}^{inc}|_{\mathcal{S}_1} \\
B_1(v_2)+B_2(v_1)=-Bv_{\alpha}^{inc}|_{\mathcal{S}_2}
\end{matrix}
\right.
\label{C4:b1v1+b2v2}
\end{equation}
where
\begin{equation}
B_1(v)=
\begin{pmatrix}
\mu \left( \frac{\partial v_1}{\partial y_1}+\frac{\partial v_2}{\partial x_1} \right) \\
\frac{\partial v_2}{\partial y_1}+\lambda \left( \frac{\partial v_1}{\partial x_1}+i\tau v_3 \right)\\
\mu \left( \frac{\partial v_2}{\partial z_1}+ \frac{\partial v_3}{\partial y_1}\right)
\end{pmatrix}
\label{C4:B1v1expl}
\end{equation}
and
\begin{equation}
B_2(v)=
\begin{pmatrix}
\mu \sin(2\varphi)\left( \frac{\partial v_1}{\partial x_2}-\frac{\partial v_2}{\partial y_2}\right)-\mu \cos(2\varphi)  \left( \frac{\partial v_1}{\partial y_2}+\frac{\partial v_2}{\partial x_2} \right)\\
(\lambda+2\mu \sin^2\varphi) \frac{\partial v_1}{\partial x_2}+(\lambda+2\mu \cos^2 \varphi)\frac{\partial v_2}{\partial y_2}-\mu \sin(2\varphi)  \left( \frac{\partial v_1}{\partial y_2}+\frac{\partial v_2}{\partial x_2} \right)+\lambda \frac{\partial v_3}{\partial z_2} \\
\mu\sin\varphi\left(\frac{\partial v_3}{\partial x_2}+\frac{\partial v_1}{\partial z_2} \right)-\mu\cos\varphi\left( \frac{\partial v_2}{\partial z_2} +\frac{\partial v_3}{\partial y_2} \right)
\end{pmatrix}
\label{C4:B2v2expl}
\end{equation}
Operator $B_1$ is obtained by projecting $B(v_1)$ onto $\mathcal{S}_1$. This is immediate because $v_1$ is defined on $\mathcal{S}_1$ and its components $(v_1^1,v_1^2,v_1^3)$ are expressed in the associated Cartesian coordinate system $(x_1,y_1,z_1)$. Operator $B_2$ is obtained by projecting $B(v_2)$ onto $\mathcal{S}_1$. This is done by projecting its components $(v_2^1,v_2^2,v_2^3)$ onto $\mathcal{S}_1$ and by expressing $(x_1,y_1,z_1)$ as functions of $(x_2,y_2,z_2)$, as $v_2$ is only defined on $\mathcal{S}_2$. This is done using \eqref{C4:changerep}. The second equation of system \eqref{C4:b1v1+b2v2} is obtained analogously to the first (operators are projected onto $\mathcal{S}_2$ instead of $\mathcal{S}_1$).

