\chapter[][Résumé de la thèse en français]{Résumé de la thèse en français}
\section*{Introduction}

\section{Chapitre 1 : Revue des approximations hautes fréquences pour la diffraction}

Dans le premier chapitre de ce manuscrit, une revue des modèles de diffraction par coin à haute fréquence est faite. Premièrement, les deux principales méthodes asymptotiques non uniformes sont décrites : \acrfull{ge}, qui modélise uniquement les rayons réfléchis et réfractés, et \acrfull{gtd}, qui prend en compte la diffraction mais diverge dans des directions d'observation proches des réflexions spéculaires. Ensuite, quelques corrections uniformes de ces modèles sont présentées. Le \acrfull{ka}, qui produit un champ dispersé uniforme mais modélise la diffraction de façon inexacte, le \acrfull{ptd} qui fournit une bonne description du champ dispersé dans toutes les directions mais est coûteux en calcul, le \acrfull{uat} qui fournit également une bonne description du champ dispersé mais est difficile à mettre en œuvre et enfin le \acrfull{utd} qui est précis, simple à mettre en œuvre et bon marché en calcul. Pour ces raisons, \acrshort{utd} est le modèle asymptotique uniforme préféré pour la diffraction en coin. Sa précision repose sur l'existence d'un modèle de diffraction en coin fiable. Dans cette optique, les deux principaux modèles de diffraction par coin existants, la méthode \acrfull{lt} et la méthode \acrfull{si}, sont brièvement présentés. Aucun d'eux n'a été développé pour un incident de vague élastique sur un coin d'angle supérieur à $\pi$.

\section{Chapitre 2 : La méthode des fonctions spectrales pour la diffraction d'une onde acoustique par un dièdre}

Dans le deuxième chapitre de ce manuscrit, la méthode des fonctions spectrales est développée comme première étape dans le cas plus simple d'une onde acoustique diffusée par un coin d'angle arbitraire mou (conditions limites de Dirichlet) ou dur (conditions limites de Neumann). Pour ce faire, une formulation intégrale de la solution au problème de la diffusion est dérivée. Cette formulation est donnée par rapport à deux fonctions inconnues appelées fonctions spectrales. Une évaluation asymptotique en champ lointain de cette formulation intégrale conduit à une expression du coefficient de diffraction \acrshort{gtd} en fonction des fonctions spectrales. La formulation intégrale est ensuite injectée dans les conditions limites du problème, donnant un système intégral d'équations fonctionnelles dont les fonctions spectrales sont la solution. Ce système est ensuite résolu de manière semi-analytique. Cela signifie que les fonctions spectrales sont décomposées comme la somme de deux termes : une fonction singulière, qui est déterminée analytiquement grâce à un algorithme récursif, et une fonction régulière, qui est approchée numériquement grâce à une méthode de collocation Galerkin. Enfin, la précision de l'approximation numérique de la pièce régulière est améliorée par une technique appelée "propagation de la solution". La méthode est validée en comparant les coefficients de diffraction \acrshort{gtd} obtenus en utilisant la méthode des fonctions spectrales semi-analytiques aux coefficients de diffraction \acrshort{gtd} dérivés de la solution exacte donnée par Sommerfeld.

\section{Chapitre 3 : La méthode des fonctions spectrales pour la diffraction 2D d'une onde élastique par un dièdre à faces libres}

Dans le troisième chapitre du manuscrit, la méthode des fonctions spectrales est appliquée au problème plus complexe de la diffraction des ondes élastiques par un coin sans contrainte d'angle arbitraire. Les principales étapes de la méthode sont les mêmes que dans le chapitre précédent, mais les calculs correspondants sont plus complexes, puisque les fonctions spectrales sont maintenant des vecteurs bidimensionnels et que les ondes incidentes, réfléchies et diffractées par les bords peuvent être polarisées longitudinalement et transversalement. Ces deux modes de propagation sont couplés par les conditions aux limites du coin, ce qui signifie que la conversion de mode a lieu. Pour chaque configuration donnée, deux coefficients de diffraction sont donc calculés : un pour les ondes diffractées longitudinales et un pour les ondes diffractées transversales. Le code développé à l'aide de la méthode \acrfull{sf} est validé pour les angles de coin inférieurs à $\pi$ par rapport au code \acrfull{lt}. Cependant, le code \acrshort{lt} existant n'est valide que pour les angles de coin inférieurs à $\pi$. Pour les angles de coin supérieurs à $\pi$, le code \acrshort{sf} est validé par comparaison à un code par éléments finis. Enfin, les résultats sont également validés expérimentalement en utilisant les mêmes mesures que celles qui ont été prises pour valider le code \acrshort{lt}.

\section{Chapitre 4 : La méthode des fonctions spectrales pour la diffraction 3D d'une onde élastique par un dièdre à faces libres}

Dans le quatrième et dernier chapitre du manuscrit, la méthode des fonctions spectrales est appliquée au cas 3D de diffraction d'onde élastique par un coin sans contrainte, où le vecteur d'onde incidente n'est pas nécessairement dans le plan normal au bord du coin. Dans ce cas, le rayon incident sur le bord du coin produit un cône de rayons diffractés appelé cône de diffraction de Keller. L'angle de ce cône est déterminé par la loi de diffraction de Snell. Selon la loi de diffraction de Snell, lorsque l'onde incidente est transversale et que l'angle de biais incident (c'est-à-dire l'angle entre le vecteur d'onde incidente et le plan perpendiculaire au bord du coin) est supérieur à un certain angle appelé angle critique, aucune onde longitudinale diffractée ne se produit. Les fonctions spectrales ont alors des points de branchement imaginaires et il faut faire très attention à les traiter. La méthode des fonctions spectrales est développée en détail pour le cas 3D, pour tous les types d'incidences et pour les angles de coin supérieurs et inférieurs à $\pi$. Une approximation numérique supplémentaire est proposée pour calculer la partie régulière des fonctions spectrales dans le cas d'une onde incidente transversale dont l'angle de biais est supérieur à l'angle critique. Le code correspondant est testé avec succès dans les cas particuliers d'incidences 2D (l'angle d'obliquité est fixé à $0$), de la "limite acoustique" (les vitesses d'onde longitudinale et transversale sont fixées pour imiter la propagation des ondes acoustiques) et d'un plan infini (l'angle du coin est égal à $\pi$ et aucune onde diffractée).

\section*{Conclusion}

Les résultats obtenus au cours de cette thèse ont mené à la publication de deux articles dans des journaux à comité de revue \cite{article, articleelasto} ainsi qu'à deux communications dans des conférences internationales \cite{DD2018,AFPAC}.