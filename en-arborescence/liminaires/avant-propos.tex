\chapter[][Résumé de la thèse en français]{Résumé de la thèse en français}

\section[Introduction]{Introduction}

Le terme de Contrôle Non Destructif (CND) désigne l'ensemble des méthodes d'inspection de l'état d'une pièce qui préservent l'intégrité physique de celle-ci. Il existe une grande variété de techniques de CND et la présente thèse se focalise sur le contrôle par ultrasons. Afin de prédire la faisabilité des inspections, mais également d'aider à l'analyse du signal reçu, le CEA-LIST (Commissariat à l’Énergie Atomique et aux Énergies nouvelles - Laboratoire d’Intégration des Systèmes et Technologies) développe la plateforme logicielle de simulation d'inspections CIVA. Les inspections par ultrasons mettant en jeu des ondes haute fréquence ($f\approx2-5$ MHz), les simulations par éléments ou différences finies peuvent s'avérer très couteuses numériquement et des méthodes semi-analytiques sont alors préférées pour traiter les problèmes haute fréquences.

Le contrôle par ultrasons d'une pièce génère des échos d'entrée et de fond sur les faces de la pièce. Si ces faces contiennent des dièdres, il est alors nécessaire de modéliser correctement les interactions entre le faisceau ultrasonore et ces dièdres. Ces interactions sont liées à deux phénomènes : la réflection par les faces du dièdre et la diffraction par l'arrête. L'objectif de cette thèse est de développer et de valider un modèle générique et fiable de diffraction des ondes élastiques par les dièdres, valide pour tout angle de dièdre ainsi que pour les configurations 3D, en étendant une méthode appelée méthode des Fonctions Spectrales (FS).


\section[Résumé du chapitre 1]{Chapitre 1 : Revue des approximations hautes fréquences pour la diffraction par un dièdre}

Dans le premier chapitre de ce manuscrit, une revue des modèles haute fréquence pour la diffraction d'ondes élastiques par un dièdre est effectuée. Cela commence par une description des deux principales méthodes asymptotiques non uniformes : l'Élastique Géométrique (EG), également appelé modèle spéculaire, qui tient uniquement compte des rayons réfléchis et réfractés, et la théorie géométrique de la diffraction (TGD), qui prend en compte la diffraction mais diverge dans des directions d'observation proches des réflexions spéculaires. 

Dans un second temps, les principales corrections uniformes de ces modèles sont présentées. L'Approximation de Kirchhoff (AK) produit un champ dispersé uniforme mais modélise la diffraction de façon inexacte. La Théorie Physique de la Diffraction (TPD) fournit une bonne description du champ dispersé dans toutes les directions mais est coûteuse numériquement. La Théorie Uniforme Asymptotique (TUA) fournit également une bonne description du champ dispersé mais nécessite le tracé de rayons fictifs et est donc difficile à mettre en œuvre numériquement. Enfin, la Théorie Uniforme de la Diffraction (TUD) produit un résultat précis, est simple à mettre en œuvre et peu couteuse numériquement. Pour ces raisons, la TUD est le modèle asymptotique uniforme le plus usité pour la diffraction par des arrêtes de dièdre. Sa précision repose sur l'existence d'un modèle TGD fiable de diffraction par un dièdre. 

Dans cette optique, les deux principaux modèles TGD existants de diffraction par l'arrête d'un dièdre sont brièvement présentés. Il s'agit de la méthode dite de la Transformée de Laplace (TL) et de la méthode dite de l'Intégrale de Sommerfeld (IS). A notre connaissance, aucune de ces deux méthodes n'a pas été développée pour une onde élastique incidente sur un dièdre d'angle supérieur à $\pi$, ni pour les configurations tridimensionnelles.

\section[Résumé du chapitre 2]{Chapitre 2 : La méthode des fonctions spectrales pour la diffraction d'une onde acoustique par un dièdre}

Le deuxième chapitre de ce manuscrit présente la première étape des développements. La méthode des fonctions spectrales est développée dans le cas simple d'une onde acoustique diffusée par un dièdre mou (conditions aux limites de type Dirichlet) ou dur (conditions aux limites de type Neumann) d'angle arbitraire. 

Pour ce faire, une formulation intégrale de la solution au problème de diffusion est déterminée. Cette intégrale est exprimée en fonction de deux fonctions inconnues appelées les fonctions spectrales. Une évaluation asymptotique en champ lointain de cette formulation intégrale conduit à une expression du coefficient de diffraction TGD dépendant des fonctions spectrales.

La formulation intégrale de la solution est ensuite injectée dans les conditions aux limites du problème, menant à un système intégral d'équations fonctionnelles dont les fonctions spectrales sont la solution. Ce système est ensuite résolu de manière semi-analytique. Cela signifie que les fonctions spectrales sont décomposées en une somme de deux termes : une fonction singulière, qui est déterminée analytiquement grâce à un algorithme récursif, et une fonction régulière, qui est approchée numériquement grâce à une méthode de collocation de Galerkin. Enfin, la précision de l'approximation numérique de la partie régulière de la solution est améliorée grâce à une technique appelée "propagation de la solution".

La méthode est validée en comparant les coefficients de diffraction TGD obtenus par la méthode des fonctions spectrales aux formules analytiques des coefficients de diffraction TGD dérivés de la solution exacte exprimée par Sommerfeld.

\section[Résumé du chapitre 3]{Chapitre 3 : La méthode des fonctions spectrales pour la diffraction 2D d'une onde élastique par un dièdre à faces libres}

Dans le troisième chapitre du manuscrit, la méthode des fonctions spectrales est étendue au problème plus complexe de la diffraction des ondes élastiques par un dièdre d'angle arbitraire et à faces libres de contraintes.

Les principales étapes de la méthode sont les mêmes que dans le chapitre précédent, mais les calculs correspondants sont plus complexes, puisque les fonctions spectrales sont maintenant des vecteurs bidimensionnels et que les ondes incidentes, réfléchies et diffractées par les faces et l'arrête peuvent être polarisées longitudinalement ou transversalement. Ces deux modes de propagation sont couplés par les conditions aux limites, ce qui signifie que des conversions de modes peuvent avoir lieu. 

Pour une configuration donnée, deux coefficients de diffraction sont donc calculés : un pour les ondes diffractées longitudinales et un pour les ondes diffractées transversales. Le code développé à l'aide de la méthode FS est validé pour les angles de dièdre inférieurs à $\pi$ par comparaison des résultats à ceux du code TL. Cependant, le code TL existant n'étant valide que pour les angles de dièdre inférieurs à $\pi$, le code FS est validé par comparaison à un code éléments finis pour les angles de dièdre supérieurs à $\pi$. Enfin, les résultats sont également validés expérimentalement en utilisant des mesures préexistantes qui avaient été effectuées pour valider le code TL.

\section[Résumé du chapitre 4]{Chapitre 4 : La méthode des fonctions spectrales pour la diffraction 3D d'une onde élastique par un dièdre à faces libres}

Dans le quatrième et dernier chapitre du manuscrit, la méthode des fonctions spectrales est étendue au cas 3D de la diffraction d'une onde élastique par un dièdre à faces, où le vecteur d'onde incident n'est pas nécessairement dans le plan normal à l'arête du dièdre. 

Dans ce cas, le rayon incident sur l'arête du dièdre produit une multitude de rayons diffractés formant un cône appelé cône de Keller. L'angle à la base de ce cône est déterminé par la loi Snell pour la diffraction. Selon cette loi, lorsque l'onde incidente est transversale et que l'angle d'inclinaison du rayon incident (c'est-à-dire l'angle entre le vecteur d'onde incident et le plan perpendiculaire à l'arête du dièdre) est supérieur à un certain angle appelé angle critique, aucune onde longitudinale diffractée n'apparait. Les fonctions spectrales ont alors des points de branchement imaginaires purs dont le traitement demande une attention particulière. 

La méthode des fonctions spectrales est détaillée pour les cas 3D, pour tous les types d'incidences et pour les angles de dièdre supérieurs et inférieurs à $\pi$. Une approximation numérique supplémentaire est proposée pour calculer la partie régulière des fonctions spectrales dans le cas d'une onde incidente transversale dont l'angle d'inclinaison est supérieur à l'angle critique. Le code correspondant est testé avec succès dans les cas particuliers d'incidences 2D (l'angle d'inclinaison vaut $0$), dans le cas de la "limite acoustique" (les vitesses d'onde longitudinale et transversale sont fixées pour simuler la propagation des ondes acoustiques) et dans le cas d'un plan infini (l'angle du dièdre est égal à $\pi$ et aucune onde n'est diffractée).

\section[Conclusion]{Conclusion}

L'objectif de cette thèse est de développer et de valider un modèle générique et fiable de diffraction des ondes élastiques par les dièdres, valide pour tout angle de dièdre ainsi que pour les configurations 3D, en étendant une méthode appelée méthode des Fonctions Spectrales (FS). Dans la conclusion, les principaux résultats sont résumés et des perspectives pour de futurs travaux sont proposées. Ces perspectives sont :
\begin{itemize}
\item Déterminer la cause de la divergence de la partie régulière des fonctions spectrales dans le cas d'une onde transversale incidente avec une inclinaison supérieure à l'angle critique
\item Évaluer la contribution asymptotique de l'onde longitudinale évanescente générée dans ce cas
\item Mener une validation numérique et/ou expérimentale complète du code élastique 3D
\item Proposer une modélisation rigoureuse de la contribution des ondes de tête
\item Étendre la méthode des fonctions spectrales au traitement des interfaces diédrales entre deux solides et au traitement des jonctions de dièdres
\end{itemize}


Les résultats obtenus au cours de cette thèse ont mené à la publication de deux articles dans des journaux à comité de revue \cite{article, articleelasto} ainsi qu'à deux communications dans des conférences internationales \cite{DD2018,AFPAC}.