\chapter[][Résumé de la thèse en français]{Résumé de la thèse en français}

\section[Introduction]{Introduction}

\hspace{2em}Le terme de \acrfull{cnd} désigne l'ensemble des méthodes d'inspection de l'état d'une pièce qui préservent l'intégrité physique de celle-ci. Afin de prédire la faisabilité des inspections, mais également d'aider à l'analyse du signal reçu, le CEA LIST (Commissariat à l'Énergie Atomique et aux Énergies Alternatives - Laboratoire d'Intégration des Systèmes et Technologies) développe la plateforme logicielle de simulation d'inspections CIVA. Il existe de nombreuses techniques de CND et la présente thèse se focalise sur le contrôle par ultrasons. Lors d'un contrôle par ultrasons d'une pièce, cette dernière génère des échos issus de ses surfaces d'entrée, internes et de fond. Si ces faces contiennent des dièdres, il est alors nécessaire de modéliser correctement les interactions entre le faisceau ultrasonore et ces dièdres. Ces interactions sont liées à deux phénomènes : la réflexion par les faces du dièdre et la diffraction par l'arête.

Les inspections par ultrasons mettant en jeu des ondes haute fréquence ($f \approx 2-5$ MHz); les simulations par éléments ou différences finies peuvent s'avérer très couteuses numériquement et des méthodes semi-analytiques sont alors préférées pour traiter les problèmes haute fréquences. L'objectif de cette thèse est de développer et de valider un modèle générique et fiable de diffraction haute fréquence des ondes élastiques par les dièdres, valide pour tout angle de dièdre ainsi que pour les configurations 3D, en étendant une méthode semi-analytique appelée méthode des Fonctions Spectrales (\acrshort{sf} en anglais).


\section[Résumé du chapitre 1]{Chapitre 1 : Revue des approximations hautes fréquences pour la diffraction par un dièdre}

\hspace{2em}Dans le premier chapitre de ce manuscrit, une revue des modèles haute fréquence pour la diffraction d'ondes élastiques par un dièdre est effectuée. Cela commence par une description des deux principales méthodes asymptotiques non uniformes : l'élastodynamique géométrique, qui a donné lieu au développement d'un modèle appelé spéculaire dans CIVA et qui tient uniquement compte des rayons réfléchis et réfractés, et la Théorie Géométrique de la Diffraction (\acrshort{gtd} en anglais), qui prend en compte la diffraction mais diverge dans des directions d'observation proches des réflexions spéculaires.

Dans un second temps, les principales solutions uniformes basées sur ces modèles sont présentées. L'approximation de Kirchhoff produit un champ diffusé uniforme mais modélise la diffraction de façon inexacte. La Théorie Physique de la Diffraction (\acrshort{ptd} en anglais) fournit une bonne description du champ dispersé dans toutes les directions mais est coûteuse numériquement dans le cas de grands obstacles. La Théorie Asymptotique Uniforme (\acrshort{uat} en anglais) fournit également une bonne description du champ dispersé mais nécessite le tracé de rayons fictifs et est donc difficile à mettre en œuvre numériquement. Enfin, la Théorie Uniforme de la Diffraction (\acrshort{utd} en anglais), développée en élastodynamique par Audrey Kamta-Djakou \cite{AKDthese} au cours de sa thèse, produit un résultat précis, est simple à mettre en oeuvre et peu couteuse numériquement. Pour ces raisons, l'\acrshort{utd} est le modèle asymptotique uniforme le plus adapté pour la diffraction par les arêtes de dièdre des grandes surfaces des pièces inspectées par ultrasons. Sa précision repose sur l'existence d'un modèle \acrshort{gtd} fiable de diffraction par un dièdre. 

Dans cette optique, les deux principaux modèles \acrshort{gtd} existants de diffraction par l'arête d'un dièdre sont brièvement présentés. Il s'agit de la méthode dite de la Transformée de Laplace (\acrshort{lt} en anglais) et de la méthode dite de l'Intégrale de Sommerfeld (\acrshort{si} en anglais). La méthode \acrshort{lt} se fonde sur une formulation intégrale des composantes du champ de déplacement utilisant le tenseur de Green et valable dans tout l'espace pour obtenir un système d'équations fonctionnelles dont la transformée de Laplace du champ de déplacement est solution. La méthode \acrshort{si} se base sur l'expression exacte des potentiels élastodynamiques donnée par Sommerfeld sous forme d'intégrales dépendant de fonctions inconnues, appelées amplitudes de Sommerfeld, pour obtenir un autre système d'équations fonctionnelles, dont les solutions sont les amplitudes de Sommerfeld. Dans les deux méthodes, les systèmes d'équations fonctionnelles sont résolus en décomposant les solutions en une somme de deux termes : une fonction singulière qui est déterminée analytiquement et une fonction régulière qui est approchée numériquement. A notre connaissance, aucune de ces deux méthodes n'a été développée pour une onde élastique incidente sur un dièdre d'angle supérieur à $\pi$, ou pour les configurations tridimensionnelles (c'est-à-dire lorsque le vecteur d'onde incident n'est pas contenu dans le plan normal à l'arête du dièdre).

\section[Résumé du chapitre 2]{Chapitre 2 : La méthode des fonctions spectrales pour la diffraction d'une onde acoustique par un dièdre}

\hspace{2em}Le deuxième chapitre de ce manuscrit présente la première étape des développements menés au cours de la thèse. La méthode des Fonctions Spectrales (\acrshort{sf} en anglais) y est développée dans le cas plus simple d'une onde acoustique diffusée par un dièdre mou (conditions aux limites de type Dirichlet) ou dur (conditions aux limites de type Neumann) et d'angle arbitraire.

Tout comme les méthodes \acrshort{lt} et \acrshort{si}, la méthode \acrshort{sf} se base sur une formulation intégrale de la solution pour en déduire un système d'équations fonctionnelles qui est ensuite résolu de manière semi-analytique en décomposant les solutions en la somme d'une fonction singulière et d'une fonction régulière. Toutefois, contrairement aux méthodes \acrshort{lt} et \acrshort{si}, la méthode \acrshort{sf} est valide pour tous les angles de dièdre, y compris les angles supérieurs à $\pi$. Cette généricité est obtenue grâce à la définition d'une nouvelle variable angulaire $\tilde{\varphi}$ qui est une fonction de l'angle du dièdre $\varphi$ mais dont l'expression est différente lorsque $\varphi \leq \pi$ et lorsque $\varphi>\pi$. Dans la version acoustique de la méthode \acrshort{sf}, la formulation intégrale susmentionnée est obtenue grâce à une transformée de Fourier de l'équation de Helmoltz. Cette intégrale est exprimée en fonction de deux fonctions inconnues appelées les fonctions spectrales. Une évaluation asymptotique en champ lointain de cette formulation intégrale à l'aide de la méthode de la plus grande pente conduit à une expression du coefficient de diffraction \acrshort{gtd} dépendant des fonctions spectrales. La formulation intégrale de la solution est ensuite injectée dans les conditions aux limites du problème, menant à un système intégral d'équations fonctionnelles dont les fonctions spectrales sont la solution. Ce système est ensuite résolu de manière semi-analytique. Cela signifie que les fonctions spectrales sont décomposées en une somme de deux termes : une fonction singulière, qui est déterminée analytiquement grâce à un algorithme récursif, et une fonction régulière, qui est approchée numériquement grâce à une méthode de collocation de Galerkin. Enfin, la précision dans tout le plan complexe de l'approximation numérique de la partie régulière de la solution est améliorée grâce à une technique appelée "propagation de la solution".

La méthode des fonctions spectrales est validée avec succès en comparant les coefficients de diffraction \acrshort{gtd} obtenus par celle-ci aux formules analytiques des coefficients de diffraction \acrshort{gtd} dérivés de la solution exacte exprimée par Sommerfeld. Les résultats obtenus avec la méthode des fonctions spectrales et ceux obtenus avec la \acrshort{gtd} issue de la formule exacte donnée par Sommerfeld sont identiques, si ce n'est parfois pour certaines directions d'observation proches des faces du dièdre.

\section[Résumé du chapitre 3]{Chapitre 3 : La méthode des fonctions spectrales pour la diffraction 2D d'une onde élastique par un dièdre à faces libres}

\hspace{2em}Dans le troisième chapitre du manuscrit, la méthode des fonctions spectrales est étendue au problème plus complexe de la diffraction des ondes élastiques par un dièdre d'angle arbitraire et à faces libres de contraintes.

Les principales étapes de la méthode sont les mêmes que dans le chapitre précédent, mais les calculs correspondants sont plus complexes, puisque les fonctions spectrales sont maintenant des vecteurs bidimensionnels et que les ondes incidentes, réfléchies et diffractées par les faces et l'arête peuvent être polarisées longitudinalement ou transversalement. Ces deux modes de propagation sont couplés par les conditions aux limites, ce qui signifie que des conversions de modes peuvent avoir lieu. Pour une configuration donnée, deux coefficients de diffraction sont donc calculés : un pour les ondes diffractées longitudinales et un pour les ondes diffractées transversales.

Pour les dièdres d'angle inférieur à $\pi$, les modules des coefficients de diffraction obtenus grâce à la méthode des fonctions spectrales sont comparés à ceux obtenus par la méthode de la Transformée de Laplace (\acrshort{lt} en anglais) et les résultats sont extrêmement proches. Cependant, le code \acrshort{lt} existant n'est valable que pour les dièdres d'angles inférieurs à $\pi$. Pour les dièdres d'angle supérieur à $\pi$, les modules des coefficients de diffraction calculés avec le code \acrshort{sf} sont comparés aux modules de coefficients de diffraction extraits d'un code éléments finis. Dans les zones où l'onde diffractée n'interfère pas avec d'autres ondes et où l'approximation \acrshort{gtd} est valable, les deux codes produisent des résultats très similaires. Enfin, les modules et phases des coefficients de diffraction calculés avec le code \acrshort{sf} sont validés expérimentalement, à l'aide de mesures préexistantes qui avaient été effectuées pour valider le code \acrshort{lt} et sont une fois de plus comparés aux résultats du code \acrshort{lt}. Les résultats des deux codes sont identiques, à l'exception d'un léger décrochage près des faces du dièdre dans un des cas testés, et sont très proches des mesures expérimentales.

\section[Résumé du chapitre 4]{Chapitre 4 : La méthode des fonctions spectrales pour la diffraction 3D d'une onde élastique par un dièdre à faces libres}

\hspace{2em}Dans le quatrième et dernier chapitre du manuscrit, la méthode des fonctions spectrales est étendue au cas 3D de la diffraction d'une onde élastique par un dièdre à faces libres, où le vecteur d'onde incident n'est pas nécessairement dans le plan normal à l'arête du dièdre. Dans ce cas, le rayon incident sur l'arête du dièdre produit une multitude de rayons diffractés pour chaque mode de propagation diffusé, formant des cônes appelés cônes de Keller. Le demi-angle au sommet de chacun de ces cônes est déterminé par la loi de Snell pour la diffraction. Selon cette loi, lorsque l'onde incidente est transversale et que l'angle d'obliquité du rayon incident (c'est-à-dire l'angle entre le vecteur d'onde incident et le plan perpendiculaire à l'arête du dièdre) est supérieur à un certain angle appelé angle critique, aucune onde longitudinale diffractée n'existe. Le champ diffracté présente alors des points de branchement imaginaires purs dont le traitement demande une attention particulière. 

La méthode des fonctions spectrales est détaillée pour les cas 3D, pour tous les types d'incidences et pour les angles de dièdre supérieurs et inférieurs à $\pi$. Une approximation numérique supplémentaire est proposée pour calculer la partie régulière des fonctions spectrales dans le cas d'une onde incidente transversale dont l'angle d'obliquité est supérieur à l'angle critique. Les valeurs des coefficients de diffraction obtenus avec cette approximation sont raisonnables, mais n'ont pas été testées expérimentalement ou numériquement. Le code des fonctions spectrales en 3D est testé avec succès dans plusieurs cas particuliers. Il produit des résultats identiques à ceux du code 2D dans les cas particuliers d'incidences 2D (l'angle d'obliquité vaut $0$) et à la \acrshort{gtd} issue de la solution exacte fournie par Sommerfeld dans le cas de la "limite acoustique" (les vitesses d'onde longitudinale et transversale sont fixées pour simuler la propagation des ondes acoustiques). Dans le cas d'un plan infini, la partie régulière est bien évaluée (l'angle du dièdre est égal à $\pi$ et aucune onde n'est diffractée), et ce notamment après l'angle critique lorsque l'approximation numérique mentionnée précédemment est appliquée.

\section[Conclusion et perspectives]{Conclusion et perspectives}

\hspace{2em}L'objectif de cette thèse est de développer et de valider un modèle générique et fiable de diffraction des ondes élastiques par les dièdres, valide pour tout angle de dièdre ainsi que pour les configurations 3D, en étendant une méthode appelée méthode des Fonctions Spectrales (\acrshort{sf} en anglais). Dans le premier chapitre, une revue bibliographique des modèles hautes fréquences existants est menée. Dans le deuxième chapitre, la méthode des fonctions spectrales est développée et validée numériquement pour le cas des ondes acoustiques. Dans le troisième chapitre, la méthode est étendue aux cas des ondes élastiques en 2D, puis validée numériquement et expérimentalement. Enfin, dans le quatrième et dernier chapitre, la méthode des fonctions spectrales est étendue aux cas des ondes élastiques en 3D et est validée numériquement pour certains cas particuliers.

Enfin, nous proposons certaines perspectives pour de futurs travaux. Ces perspectives sont :
\begin{itemize}
\item Dans le dernier chapitre de la thèse, les parties régulières des fonctions spectrales divergent dans le cas d'une onde transversale incidente avec une obliquité supérieure à l'angle critique. Il faut poursuivre les travaux afin de déterminer la cause de cette divergence et proposer une nouvelle méthode de calcul.
\item Évaluer ou modéliser la contribution asymptotique de l'onde longitudinale évanescente générée dans le cas d'une onde transversale incidente avec une obliquité supérieure à l'angle critique.
\item Mener une validation numérique et/ou expérimentale complète du code élastique 3D (travail actuellement en cours).
\item Au cours de sa thèse, Audrey Kamta-Djakou \cite{AKDthese} a développé le modèle \acrshort{utd} pour la diffraction des ondes élastiques par des dièdres, en utilisant l'algorithme de propagation des pôles de la méthode \acrshort{si}. Le modèle \acrshort{utd} doit donc être adapté à
l'algorithme \acrshort{sf} afin de pouvoir être appliqué aux incidences 3D. J'ai entamé les travaux d'intégration dans la plateforme logicielle CIVA du code \acrshort{sf} 3D avec un modèle \acrshort{utd}. Afin de gérer l'extension finie des arêtes dans CIVA, une possibilité est d'utiliser un modèle incrémental tel que la Théorie Incrémentale de la Diffraction (ITD en anglais) ou le modèle de Huygens que j'ai aidé à développer et valider en élastodynamique \cite{articleITD}.
\item Proposer une modélisation rigoureuse de la contribution aux coefficients de diffraction élastodynamiques des ondes de tête, dans la continuité des travaux de Fradkin et al. \cite{FradkinDarmon} et de ceux de Darmon \cite{HDRMichel}.
\item Étendre la méthode des fonctions spectrales au traitement des interfaces diédrales entre deux solides. Ceci s'inscrirait dans la continuité des travaux du stage de Lucien Rochery, encadré par Michel Darmon et moi-même. Au cours de ce stage, les développements théoriques concernant la diffraction des ondes acoustiques et élastiques par des dièdres impédants ont été initiés.
\item Dans la continuité des travaux de Kamotskii \cite{Kamotski2}, la méthode des fonctions spectrales pourrait être étendue au traitement des jonctions de dièdres (diffraction par deux dièdres adjacents).
\end{itemize}


Les résultats obtenus au cours de cette thèse ont mené à la publication de trois articles dans des journaux à comité de revue \cite{articleITD,article,articleelasto} ainsi qu'à deux communications dans des conférences internationales \cite{DD2018,AFPAC}.