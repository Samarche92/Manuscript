% Résumés (de 1700 caractères maximum, espaces compris) dans la
% langue principale (1re occurrence de l'environnement « abstract »)
% et, facultativement, dans la langue secondaire (2e occurrence de
% l'environnement « abstract »)
\begin{abstract}
This thesis falls into the framework of model development for simulation of ultrasonic non-destructive testing (NDT). The long-term goal is to develop, using ray methods, a complete simulation tool of specimen echoes (input, back-wall surfaces...) or echoes of inner structures of inspected parts. The thesis aims more specifically to integrate the phenomenon of diffraction by wedges to an existing model derived from geometrical acoustics, which only accounts for reflections on the wedge faces.

To this end, a method called the spectral functions method, which was initially developed for immersed wedges, is developed and validated as a first step in the case of acoustic waves with Dirichlet or Neumann boundary conditions. The method is then extended to elastic wave diffraction by infinite stress-free wedges of arbitrary angles, for 2D and 3D incidences. This method is semi-analytic since the unknown solutions are expressed as the sum of a singular function, determined analytically using a recursive algorithm, and a regular function which is approached numerically.

The corresponding codes are validated by comparison to an exact solution in the acoustic case and by comparison to other codes (semi-analytic and numerical) in the elastic case. Experimental validations of the elastodynamic model are also proposed.
\end{abstract}
\begin{abstract}
Le sujet de la thèse s’inscrit dans le cadre du développement de modèles pour la simulation du contrôle non-destructif (CND) par ultrasons. L'objectif à long terme est la mise au point, par une méthode de rayons, d’un outil complet de simulation des échos issus de la géométrie (surfaces d’entrée, de fond…) ou des structures internes des pièces inspectées. La thèse vise plus précisément à intégrer le phénomène de diffraction par les dièdres à un modèle existant dérivant de l’acoustique géométrique et qui prend uniquement en compte les réflexions sur les faces. 

Pour cela, la méthode dite des fonctions spectrales, développée initialement pour le cas d'un dièdre immergé, est développée et validée dans un premier temps dans le cas des ondes acoustiques pour des conditions aux limites de type Dirichlet ou Neumann. La méthode est ensuite étendue à la diffraction des ondes élastiques par des dièdres infinis à faces libres et d'angles quelconques, pour une incidence 2D puis pour une incidence 3D. Cette méthode est semi-analytique puisque les solutions recherchées s'écrivent sous la forme d'une somme d'une fonction singulière, qui est déterminée analytiquement à l'aide d'un algorithme récursif, et d'une fonction régulière, qui est approchée numériquement. 

Les codes correspondants sont validés par comparaison à une solution exacte dans le cas acoustique et par comparaison à d'autres codes (semi-analytiques et numériques) dans le cas élastique. Des validations expérimentales du modèle élastodynamique sont également proposées.
\end{abstract}
%
% Production de la page de résumés
\makeabstract
