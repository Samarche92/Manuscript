% Auteur de la thèse : prénom (1er argument obligatoire), nom (2e argument
% obligatoire) et éventuel courriel (argument optionnel). Les éventuels accents
% devront figurer et le nom /ne/ doit /pas/ être saisi en capitales
\author[samar.chehade.sc@gmail.com]{Samar}{Chehade}
%
% Titre de la thèse dans la langue principale (argument obligatoire) et dans la
% langue secondaire (argument optionnel)
\title[Modélisation de la diffusion 3D d'ondes élastiques par des structures complexes pour le calcul des échos de géométrie. Application à la simulation des CND par ultrasons.]{Modelling of the 3D scattering of elastic waves by complex structures for specimen echoes calculation. Application to ultrasonic NDT simulation.}
%
% (Facultatif) Sous-titre de la thèse dans la langue principale (argument
% obligatoire) et dans la langue secondaire (argument optionnel)
%\subtitle[Chaos' Laugh]{Le rire du chaos}
%
% Champ disciplinaire dans la langue principale (argument obligatoire) et dans
% la langue secondaire (argument optionnel)
\academicfield[Physique]{Physics}
%
% (Facultatif) Spécialité dans la langue principale (argument obligatoire) et
% dans la langue secondaire (argument optionnel)
\speciality[Acoustique]{Acoustics}
%
% Date de la soutenance, au format {jour}{mois}{année} donnés sous forme de
% nombres
\date{26}{9}{2019}
%
% (Facultatif) Date de la soumission, au format {jour}{mois}{année} donnés sous
% forme de nombres
%\submissiondate{1}{10}{2014}
%
% (Facultatif) Sujet pour les méta-données du PDF
\subject[Diffraction d'ondes élastiques]{Elastic wave diffraction}
%
% (Facultatif) Nom (argument obligatoire) de la ComUE
%\comue[logo=images/logo_uns_hd.jpg]{Université de Nice}
%
% Nom (argument obligatoire) de l'institut (principal en cas de cotutelle)
\institute[logo=images/edeobe.png]{Université Paris-Saclay}
%
% (Facultatif) En cas de cotutelle (normalement, seulement dans le cas de
% cotutelle internationale), nom (argument obligatoire) du second institut
%\coinstitute[logo=images/paris13,url=http://www.univ-paris13.fr/]{Université de Paris~13}
%
% (Facultatif) Nom (argument obligatoire) de l'école doctorale
\doctoralschool[url=https://www.universite-paris-saclay.fr/fr/formation/doctorat/electrical-optical-bio-physics-and-engineering-eobe]{EOBE}
%
% Nom (1er argument obligatoire) et adresse (2e argument obligatoire) du
% laboratoire (ou de l'unité) où la thèse a été préparée, à utiliser /autant de
% fois que nécessaire/
\laboratory[
logo=images/LOGOLIST2016.jpg,
logoheight=2.0cm,
telephone=+33 (0)1 69 08 08 00,
email=info-list@cea.fr,
url=http://www-list.cea.fr/
]{CEA LIST}{%
  CEA Saclay Digiteo Labs \\
Bât. 565\\
91191 Gif-sur-Yvette\\
  France}
%
% Directeur(s) de thèse et membres du jury, saisis au moyen des commandes
% \supervisor, \cosupervisor, \comonitor, \referee, \committeepresident,
% \examiner, \guest, à utiliser /autant de fois que nécessaire/ et /seulement
% si nécessaire/. Toutes basées sur le même modèle, ces commandes ont
% 2 arguments obligatoires, successivement les prénom et nom de chaque
% personne. Si besoin est, on peut apporter certaines précisions en argument
% optionnel, essentiellement au moyen des clés suivantes :
% - « professor », « seniorresearcher », « associateprofessor »,
%   « associateprofessor* », « juniorresearcher », « juniorresearcher* » (qui
%   peuvent ne pas prendre de valeur) pour stipuler le corps auquel appartient
%   la personne ;
% - « affiliation » pour stipuler l'institut auquel est affiliée la personne ;
% - « female » pour stipuler que la personne est une femme pour que certains
%   mots clés soient accordés en genre.
%
\supervisor[corps=expert CEA*,affiliation=CEA LIST]{Michel}{Darmon}
\cosupervisor[professor,affiliation=Université de Nice, Sofia-Antipolis]{Gilles}{Lebeau}
%\referee[professor,affiliation=IHP]{René}{Descartes}
%\referee[seniorresearcher,affiliation=CNRS]{Denis}{Diderot}
%\committeepresident[professor,affiliation=ENS Lyon]{Victor}{Hugo}
%\examiner[associateprofessor,affiliation=Université de Paris~13]{Sophie}{Germain}
%\examiner[juniorresearcher,affiliation=INRIA]{Joseph}{Fourier}
%\examiner[juniorresearcher*,affiliation=CNRS]{Paul}{Verlaine}
%\guest{George}{Sand}
%
% (Facultatif) Mention du numéro d'ordre de la thèse (s'il est connu, ce numéro
% est à spécifier en argument optionnel)
%\ordernumber[42]
%
% Préparation des mots clés dans la langue principale (1er argument) et dans la
% langue secondaire (2e argument)
%%%%%%%%%%%%%%%%%%%%%%%%%%%%%%%%%%%%%%%%%%%%%%%%%%%%%%%%%%%%%%%%%%%%%%%%%%%%%%%
\keywords{Elastodynamics, Diffraction, Asymptotic Methods}{Elastodynamique, Diffraction, Méthodes asymptotiques}
